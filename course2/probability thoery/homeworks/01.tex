\subsection{Домашка первая}

\i Давайте рассмотирм случай, когда ровно $0 \leq m \leq n$ программистов взяли свои ноутбуки. Так как все перемешивания ноутбуков равновероятны, то вероятность 
такого события равна $\frac{1}{n!}\binom{n}{m} (n-m-1)!$ (мы вибираем $m$ программистов, которые получают свои ноутбуки, тогда вариантов, когда из остальных 
свой не получил никто ровно $(n-m-1)!$). Конкретно в этом случае нас интересует только то, что случилось с $m$ программистами со своими ноутами. они 
потеряют их с вероятностью $p^m$. Осталось только разобраться с тем случаем, когда никто не получил свой ноут, вероятность таких случаев составляет
Осталось только просуммировать по $m$ и получить 
\[\sum_{m=0}^n \frac{(n-m-1)!}{n!}\binom{n}{m}p^m.\]

\i Для начала следуюет определитсья с вероятностным пространством. Для удобства предлагаю рассматривать в качестве элементарных исходов наборы, в которых
содержится информация о первом и втором (с учётом порядка) человеке, у которых брали тест и результаты этих тестов. Также, очевидно, что все упорядоченные
пары спортсменов имеют одинаковую вероятность быть выбранными (ровно $\frac{1}{12}$). За $P_i$ обозначим верочтность того, что $i$-тый спортсмен сдавал тест
и получил отрицатеьный результат, тогда легко убедиться в том, что 

\begin{enumerate}
    \item $P_A = 0{,}1 +  0{,}9 \cdot 0{,}05$;
    \item $P_A = 0{,}5 +  0{,}5 \cdot 0{,}05$;
    \item $P_A = 0{,}8 +  0{,}2 \cdot 0{,}05$;
    \item $P_A = 1$.
\end{enumerate}
Теперь за $p_i$ обозначим вероятность того, что $i$-тый спортсмен получил положительный результат (и вообще прошёл тест), если он оказался вторым в очереди на 
проверку. Для этого достаточно просуммировать желаемые вероятности по всем парам вида $(j, i)$, где $j$ --- номер спортсмена отличный от $i$. Таким образом 
получим, что 
\[p_i = \sum_{j \ne i} P_j (1-P_i).\]
После этого осталось учесть только то, что все выборы спортсменов для проверки равновероятны, и то, что нас интересует вероятность при условии, что у проверки 
в итоге оказался положительный результат, а значит искомые числа будут равны 
\[X_i = \frac{1}{12}P_i / \sum_{k \in \{A, B, C, D\}} \frac{P_k}{12} .\]
Опустимдолгие и тривиальные вычисления и просто скажем, что 
\begin{enumerate}
    \item $X_A \approx 0{,}613$;
    \item $X_B \approx 0{,}296$;
    \item $X_C \approx 0{,}091$;
    \item $X_D \approx 0$.
\end{enumerate}
