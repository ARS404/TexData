\subsection{Домашка третья}

\i Рассмотрим 2 пункта.
\par\pu $np(n) - \ln{n} \seek -\infty$. Попробуем применить неравенство Чебышева для оценки $P(X_n=0)$:
\begin{gather*}
    P(\abs{X_n - EX_n} \geq EX_n) \leq \frac{DX_n}{E^2X_n};\\ 
    P(X_n = 0) \leq \frac{DX_n}{E^2X_n}.
\end{gather*}
Теперь нам надо понять что-нибудь про правую часть неравенства. Для этого введём индикаторы $I_i$, равные 1 при условии, что $i$-тая вершина изолирована. Посчитаем $EX_n$:
    \[EX_n = E(\sum_{i=1}^n I_i) = \sum_{i=1}^n EI_i = \sum_{i=1}^n E=P(I_i = 1) =\footnote{Для того, чтобы вершина оказалась изолированной необходимо удалить все 
    рёбра из неё, а их ровно $n-1$} n(1-p)^{n-1}.\]
Посчитаем $DX_n$:
\begin{gather*}
    DX_n = E(X_n^2) - (EX_n)^2;\\ 
    E(X_n^2) = E(\sum_{i=1}^n I_i \times \sum_{i=1}^n I_i) = \sum_{i=1}^n P(I_i^2) + 2\sum_{i \ne j} P(I_i)P(I_j) = n(1-p)^{n-1} + n(n-1)(1-p)^{2n-3}.\\
    DX_n = n(1-p)^{n-1} + n(n-1)(1-p)^{2n-3} - n^2(1-p)^{2n-2}.
\end{gather*}
Тогда мы получили, что 
\begin{gather*}
    P(X_n = 0) \leq \frac{n(1-p)^{n-1} + n(n-1)(1-p)^{2n-3} - n^2(1-p)^{2n-2}}{n^2(1-p)^{2n-2}} = \frac{n(1-p)^{n-1} + n(n-1)(1-p)^{2n-3}}{n^2(1-p)^{2n-2}} - 1.
\end{gather*}
Видимо, вот такую штуку нам придётся оценивать, сразу скажем, что из того, что $np(n) - \ln{n} \seek -\infty$ следует, что $\limit{n}{\infty}p = 0$. Тогда давайте считать:
\begin{gather*}
    \limit{n}{\infty} \brackets{\frac{n(1-p)^{n-1} + n(n-1)(1-p)^{2n-3}}{n^2(1-p)^{2n-2}} - 1} = 
    \limit{n}{\infty} \brackets{\frac{1}{1-p} + \frac{1}{n(1-p)^{n-1}} - \frac{1}{n(1-p)} - 1} = \\ = 
    \limit{n}{\infty} \frac{1}{1-p} + \limit{n}{\infty} \frac{1}{n(1-p)^{n-1}} - \limit{n}{\infty} \frac{1}{n(1-p)^{n-1}} - 1 = 1 + 0 - 0 - 1 = 0.
\end{gather*}
Что и требовалось доказать!
\par\pu $np(n) - \ln{n} \seek \infty$. Снова воспользуемся неравенством Чебышева:
    \[P(X_n > 0) \leq EX_n = n(1-p)^{n-1}.\]
Снова считаем пределы:
    \[\limit{n}{\infty} n(1-p)^{n-1} = \limit{n}{\infty} e^{\ln{n} - (n-1)(p + \abs{O(p^2)})} =\footnote{Так как $0 \leq p(n) \leq 1$} \leq e^{\ln{n} - np + p} = 0.\]
Это победа, так как мы получили, что $P(X_n = 0) = 1 - P(X_n > 0) = 1$.


\i Покажем, что $Y_n$ --- марковская цепь. Это вполне очевидно, достаточно показать, что 
    \[P(Y_n = a_n | Y_{n-1} = a_{n-1}) = P(Y_n = a_n | Y_{n-1} = a_{n-1}, Y_{n-2} = a_{n-2}, \ldots, Y_1 = a_1).\]
При этом верно, что давайте рассмотрим $a_n\hatch$ равное множеству значений $X_n$ таких, что $Y_n = a_n$ тогда имеем:
    \[P(Y_n = a_n | Y_{n-1} = a_{n-1}) = P(X_n \in a_n\hatch | X_{n-1} \in a_{n-1}\hatch)\]
и 
    \[P(Y_n = a_n | Y_{n-1} = a_{n-1}, X_{n-2} = a_{n-2}, \ldots, Y_1 = a_1) = P(X_n \in a_n\hatch | X_{n-1} \in a_{n-1}\hatch, X_{n-2} \in a_{n-2}\hatch, \ldots, X_1 \in a_1\hatch)\]
В силу того, что $X_n$ по условию марковская цепь получаем равенство правых частей двух выражений выше, которое и влечёт желаемый результат.
Теперь посчитаем матрицу переходов:
\begin{itemize}
    \item $P(Y_n = 1 | Y_{n-1} = 1) = P(X_n = 1 | X_{n-1} = 1) = \frac{3}{7}$;
    \item $P(Y_n = 2 | Y_{n-1} = 2) = P(Y_n = 2 | X_{n-1} = 1) = P(X_n = 2 | X_{n-1} = 1) + P(X_n = 3 | X_{n-1} = 2) = \frac{4}{7}$;
    \item $P(Y_n = 1 | Y_{n-1} = 2) = P(X_n = 1 | Y_{n-1} = 2) = P(X_n = 1 | X_{n-1} = 2 \cup X_{n-1} = 3) = \\ =
        P(X_n = 1 | X_{n-1} = 2)P(X_{n-1} = 2 | Y_{n-1} = 2) + P(X_n = 1 | X_{n-1} = 3)P(X_{n-1} = 3| Y{n-1} = 2) = \frac{10}{11}$;
    \item $P(Y_n = 2 | Y_{n-1} = 2) = 1 - P(Y_n = 1 | Y_{n-1} = 2) = \frac{1}{10}$.
\end{itemize}


\i Для начала обратим внимание на то, что все $\eta_i$ положительные, а значит можно смело утверждать, что последовательность $T_n$ строго возрастает. Для того чтобы доказать, что 
$X_n$ является марковской цепью нам достаточно показать, что вероятность очередного её состояние зависит только от предыдущего, то есть 
    \[P(X_n = a_n | X_{n-1} = a_{n-1}) = P(X_n = a_n | X_{n-1} = a_{n-1}, X_{n-2} = a_{n-2}, \ldots, X_1 = a_1).\]
При этом
    \[P(X_n = a_n | X_i = a_i) = P(\sum_{i=T_i+1}^{T_n} \xi_i = a_n - a_i | X_i = a_i).\]
Осталось только заметить, что если мы распишем правую вероятность в желаемом равенстве пользуясь утверждением выше, то почти все суммы сократятся с теми, которые стоят в условии.
Теперь поймём, что у нас осталось только 
    \[P(\sum_{i=T_{n-1}+1}^{T_n} \xi_i = a_n - a_{n-1} | X_{n-1} = a_{n-1}).\]
А это именно то, что мы хотим, победа!