\subsection{Лекция вторая}

\begin{problem}[Задача о сумасшедшей старушке]
    Задача на условные вероятности (про старушку, которая заходит в автобус и рандомно садится), по индукции доказывается, что последний
    садится на своё место с вероятностью $\frac{1}{2}$, предпоследний --- $\frac{2}{3}$, а оба --- $frac{1}{6}$. Вообще, похоже, что события независимы
    (но это ещё надо доказать). 
\end{problem}

\begin{claim}[Формула Байеса]
    Пусть $D_1, D_2, \ldots, D_n$ --- разбиение $\Omega$, $A$ --- некоторое событие с вероятностью больше нуля, тогда
    \[P(D_j | A) = \frac{P(A | D_j)P(D_j)}{\sum_{t=0}^n P(A | D_t)P(D_t)}.\]
    Доказывается через подсчёт по определению (через формулы условной и полной вероятности). Работают и для счётного разбиения. 
\end{claim}

\begin{definition}[Независимость]
    События события $A$ и $B$ независимы тогда и только тогда 
    \[P(A \cap B) = P(A) \cdot P(B).\]
    события $A_i$ попарно незывисимы, если дял любых разлинчых $i,\ j$ события независимы.
\end{definition}

\begin{definition}[Совокупная независимость]
    Для любого подмножества $A_i$ верно, что вероятность пересечения равна пересечению вероятностей.
\end{definition}

\begin{remark}
    Считаем, что независимость == совокупная независимость. 
\end{remark}

\begin{definition}[Случайные величины в дискретных вероятностных пространствах]
    Пусть $(\Omega, P)$ --- дискретное вероятностное пространство. Случайной величиной (с.в.) на $(\Omega, P)$ называется отображение $\xi: \Omega -> \RR$.
    С.в. --- числовая характеристика случайного эксперимента.
\end{definition}

\begin{definition}
    Пусть с.в. $\xi$ принимает значения $\{x_1, x_2, \ldots, x_n\}$, обозначим 
    \[A_i = \{\omega : \xi(\omega) = x_1\} := P(\xi).\]
\end{definition}