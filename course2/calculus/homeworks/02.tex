\subsection{Домашка вторая}


\i 
\par\pu Посмотрим, как себя ведёт $A_i = \frac{1}{1+\frac{1}{\ln{\ln{n}}}}$. В силу того, что $\ln{\ln{n}}$ при $n \geq 3$ монотонно возрастает и не ограничен можно расписать 
следующее:
    \[\limit{n}{\infty} \frac{1}{1 + \frac{1}{\ln{\ln{n}}}} = \frac{1}{1 + \limit{n}{\infty} \frac{1}{\ln{\ln{n}}}} = \frac{1}{1 + 0} = 1.\]
Это означает, что с некоторого момента $N$ все слагаемые нашей суммы  больше, скажем, $\frac{1}{2}$, а значит желаемую сумму с некоторого момента можно оценить следующим образом:
    \[\sum_{n=3}^{N+k} \frac{1}{1+\frac{1}{\ln{\ln{n}}}} \geq \frac{k}{2} + X,\]
где $X$ --- сумма всех $A_i$ где $i < N$. Таким образом, мы ограничили нашу последовательность снизу другой последовательностью, которая, очевидно, расходится к $+\infty$, а значит
и исходная последовательность расходится к $+\infty$.

\par\pu Аналогично посмотрим на последовательность $A_i = \brackets{n^2 \sin{\frac{1}{n^2}}}^4$:
    \[\limit{n}{\infty} \brackets{n^2 \sin{\frac{1}{n^2}}}^4 = \brackets{\limit{n}{\infty} n^2 \sin{\frac{1}{n^2}}^4} =\footnote{Пользуемся первым замечательным пределом, 
    к которому приходим заменой $x = \frac{1}{n^2}$.} 1^4 = 1.\]
Дальнейшие рассуждения полностью повторяю предыдущий пункт, в итоге получаем, что рад расходится к $+\infty$.

\par\pu Давайте напишем следующее неравенство:
    \[\sum_{n=1}^{\infty} \frac{3^{2n}(2n)!}{n^n n!} = \sum_{n=1}^{\infty} \frac{3^{2n}\prod_{i=n+1}^{2n}i}{n^n} \geq \sum_{n=1}^{\infty} 3^{2n}.\]
Мы получили, что исходный ряд ограничен снизу рядом, который, очевидно, расходится в $+\infty$, а значит также делает и исходный.

\par\pu Для решения нам даже не понадобится восстанавливать строгую формулу $H_i$. Просто покажем, что при $i \geq 1$ верно, что $H_i \geq 2^{i-1}$. Сделаем это по индукции:
\begin{itemize}
    \item база: $i \in \{1, 2\}$ --- очевидно;
    \item переход: пусть наше предположение верно для всех $i \leq n$, покажем, что оно также верно и для $i = n+1$:
            \[H_{n+1} = 2H_n + H_{n-1} \geq 2 \cdot 2^{n-1} + 2^{n-2} \geq 2^n,\]
        что и требовалось доказать.
\end{itemize}
В таком случае, можно написать следующее неравенство:
    \[\sum_{i=2}^{\infty} \frac{1}{H_i} \leq \sum_{k=1}^{\infty} \frac{1}{2^k} = 1.\]
Осталось только сказать, что последовательность частичных сумм для изначального ряда монотонно возрастает и ограничена, а значит имеет предел.

\par\pu Для начала заметим, что $\lambda_i \in \left[\pi(i-1+\frac{1}{2}), \pi(i+\frac{1}{2})\right]$. Тогда просто возьмём, и ограничим снизу каждое слагаемое вида 
$\frac{1}{\lambda_i}$ числом $\frac{1}{\pi(i+1)}$. Тогда можно написать следующее неравенство:
    \[\sum_{i=1}^{\infty} \frac{1}{\lambda_i} \geq \sum_{i=1}^{\infty} \frac{1}{\pi(i+1)} = \frac{1}{\pi} \sum_{i=1}^{\infty} \frac{1}{i+1}.\]
При этом мы знаем, что гармонический рад расходится, а значит расходится и $\sum_{i=1}^{\infty} \frac{1}{i+1}$, а значит исходная сумма тоже расходится к $+\infty$.


\i Давайте рассмотрим ряд 
    \[b_n = \sqrt{\sum_{i=1}^n a_i} - \sqrt{\sum_{i=i}^{n-1} a_i} > 0.\]
Для начала докажем, что такой ряд разойдётся. Если мы распишем частичные суммы для полученного ряда, то получим:
    \[S_n = \sum_{m=1}^n \brackets{\sqrt{\sum_{i=1}^m a_i} - \sqrt{\sum_{i=i}^{m-1} a_i}} = \sqrt{\sum_{i=1}^n a_i}.\]
Желаемое мы получим из того, что последовательность частичных сумм для $\{a_i\}$ не ограничена.\\ 
Осталось только доказать, что $\limit{n}{\infty} \frac{b_n}{a_n} = 0$:
\begin{gather*}
    \limit{n}{\infty} \frac{b_n}{a_n} = \limit{n}{\infty} \frac{\sqrt{\sum_{i=1}^{n\phantom{1}} a_i} - \sqrt{\sum_{i=i}^{n-1} a_i}}{a_n} = \\ = 
    \limit{n}{\infty} \frac{1}{\sqrt{\sum_{i=1}^{n\phantom{1}} a_i} + \sqrt{\sum_{i=i}^{n-1} a_i}} = 0.
\end{gather*}
Последнее равенство очевидно в силу того, что частичные суммы для $\{a_i\}$ не ограничены. Победа!