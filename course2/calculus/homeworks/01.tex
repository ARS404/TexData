\subsection{Домашка первая}

\i В этой задаче мы хотим доказать, расходимость ряда по критерию Коши, то есть доказать, что 
    \[\exists \epsilon > 0\ \forall N \ \exists n \geq N, p: \ \abs{\sum_{i=n}^{n+p}} > \epsilon\]
Давайте просто рассмотрим $\epsilon = \frac{1}{4},\ n = p = N$, тогда мы получим, что 
    \[\abs{\sum_{i=n}^{n+p}} = \sum_{i = N}^{2N} \frac{1}{2i + 1} \geq \frac{N+1}{4N + 1} > \frac{1}{4}.\]
Таким образом, мы получили то, что хотели, победа!

\i 

\pu
\begin{gather*}
    \sum_{n=1}^{\infty}\brackets{2\sqrt{2n+1} - 3\sqrt{2n+3} - \sqrt{2n+5}} = 
    \sum_{n=1}^{\infty}\brackets{\brackets{2 - 3\sqrt{\frac{2n+3}{2n+1}} - \sqrt{\frac{2n+5}{2n+1}}} \cdot \sqrt{2n+1}}.
\end{gather*}
Теперь давайте посчитаем 
    \[\limit{n}{\infty} \brackets{2 - 3\sqrt{\frac{2n+3}{2n+1}} - \sqrt{\frac{2n+5}{2n+1}}} \cdot \sqrt{2n+1}.\]
Вполне очевидно, что дроби под корнем стремятся к 1, тогда получаем, что 
    \[\limit{n}{\infty} \brackets{2 - 3\sqrt{\frac{2n+3}{2n+1}} - \sqrt{\frac{2n+5}{2n+1}}} \cdot \sqrt{2n+1} = -2\sqrt{2n+1} = -\infty.\]
Таким образом, мы получили, что члены последовательности стремятся к $-\infty$, а значит сумма последовательности тоже сходится к $-\infty$.

\pu Для начала давайте по индукции докажем, что 
    \[\sum_{n = 1}^k \frac{1}{(n+1)!} = 1 - \frac{1}{(k+1)!}.\]
База $k = 1$ очевидна.\\ 
Переход от $k = m$ к $k = m+1$:
    \[\sum_{n = 1}^{m+1} \frac{1}{(n+1)!} = \sum_{n = 1}^m \frac{1}{(n+1)!} + \frac{1}{(m+2)!} = 1 - \frac{1}{(m+1)!} + \frac{1}{(m+2)!} = 1 - \frac{1}{(m+2)!}.\]
Таким образом мы получили, что частичные суммы по $k$-тый элемент имеют вид $1 - \frac{1}{(k+1)!}$, а значит, очевидно, последовательность частичных сумм сходится
к 1, а значит и сумма ряда равна 1.


\i 

\pu Заметим, что последовательность частичных сумм будет строго возрастаять, а значит, если мы докажем ограниченнгость суммы ряда сверху, то докажем и сходимость 
ряда, так что приступим:
    \[\sum_{i=1}^{\infty} \frac{\ln{i!}}{i^3} \leq \sum_{i=1}^{\infty} \frac{i\ln{i}}{i^3} = \sum_{i=1}^{\infty} \frac{\ln{i}}{i^2} \leq 
    \sum_{i=1}^{\infty}\frac{1}{i^{1{,}5}}.\]
Как известро, сумма вида $\sum_{i=1}^{\infty} \frac{1}{i^p}$ сходится тогда и только тогда, когда $p > 1$, что верно в нашем случае. Таким образом, мы доказали
сходимость исходного ряда.

\pu Попробуем доказать, что ряд расходится, для этого покажем, что при $n \to \infty$ его члены стремятся не к нулю. Давайте развлекаться:
\begin{gather*}
    \limit{n}{\infty} n^2\tg\brackets{\frac{n^2+1}{n^4+1}} = \limit{n}{\infty} \frac{n^2\sin\brackets{\frac{n^2+1}{n^4+1}}}{\cos\brackets{\frac{n^2+1}{n^4+1}}} = 
    \frac{\limit{n}{\infty} \brackets{n^2\sin\brackets{\frac{n^2+1}{n^4+1}}}}{\limit{n}{\infty} \cos\brackets{\frac{n^2+1}{n^4+1}}} = \\ = 
    \frac{\limit{n}{\infty} \brackets{n^2\sin\brackets{\frac{n^2+1}{n^4+1}}}}{1} = \frac{1}{1} = 1.
\end{gather*}
Таким образом, у нас получилось доказать, что ряд расходится.


\i Видимо, в условии задачи есть опечатка, к сожалению, ответа, как понимать условие, я не нашёл, поэтому выберу самый интересный вариант (по моему мнению): 
оригиральная последовательность расходится, а при возведении в степень сходится. Также буду считать, что $m > 1$. Очень хочется воспользоваться Дзета-функцией,
так и сделаем. Для этого достаточно взять $a_i = \frac{1}{i}$. Как известно, гармонический ряд расходится, также выполнено следующее неравенство при $m \geq 2$:
    \[\sum_{i=1}^{\infty} \frac{1}{i^m} \leq \sum_{i=1}^{\infty} \frac{1}{i^2} = \frac{\pi^2}{6}.\]
Так как сумма $\sum_{i=1}^{\infty} \frac{1}{i^m}$ ограничена сверху и монотонно возрастает, то она сходится.
Такой пример подойдёт для любого $m$, удовлетворяющего условию.


\i Воспользуемся подсказкой и обозначим за $S_{d, l}$ сумму всех членов $K_d$, знаменатель которых имеет длину $l$. Тогда будет справедливо равенство
    \[K_d = \sum_{i=1}^{\infty} S_{d, i}.\]
Рассмотрим подробнее $S_{d, l}$

в таком случае, каждое $S_{d, l}$ будет иметь не менее $9^{l}$ слагаемых\footnote{если $n = 0$, то на каждое место можно поставить 9 цифр, 
иначе не первое можно поставить только 8, а на остальные --- 9}, каждое из которых не больше, чем $10^{-l+1}$, а значит и 
    \[S_{d, l} \leq 9\brackets{\frac{9}{10}}^{l-1}.\]
Получается, можно написать следующее неравенство:
    \[K_d = \sum_{i=1}^{\infty} S_{d, i} \leq 9\sum_{i=1}^{\infty}\brackets{\frac{9}{10}}^{i-1} = 90.\]
Таким образом, мы получили, что сумма ряда ограничена сверху 90, а также все его слагаемые положительны, а значит рад сходится.

