\subsection{Первое ИДЗ}

\i
\begin{gather*}
    AB = \begin{pmatrix}
        2 & 3 & 5\\
        -3 & -4 & -7\\
        4 & -3 & 1\\
        3 & 1 & 4\\
        -2 & 4 & 2
    \end{pmatrix} \times \begin{pmatrix}
        1 & 0 & 0 & 1\\
        3 & 0 & 1 & 0\\
        -1 & 1 & 0 & 0
    \end{pmatrix} = \begin{pmatrix}
        6 & 5 & 3 & 2\\
        -8 & -7 & -4 & -3\\
        -6 & 1 & -3 & 4\\
        2 & 4 & 1 & 3\\
        8 & 2 & 4 & -2
    \end{pmatrix};\\
    ABx = \begin{pmatrix}
        6 & 5 & 3 & 2\\
        -8 & -7 & -4 & -3\\
        -6 & 1 & -3 & 4\\
        2 & 4 & 1 & 3\\
        8 & 2 & 4 & -2
    \end{pmatrix} \times x = 0;\\
    \begin{pmatrix}
        6 & 5 & 3 & 2\\
        -8 & -7 & -4 & -3\\
        -6 & 1 & -3 & 4\\
        2 & 4 & 1 & 3\\
        8 & 2 & 4 & -2
    \end{pmatrix} \approx \begin{pmatrix}
        2 & 4 & 1 & 3\\
        0 & 1 & 0 & 1\\
        0 & 0 & 0 & 0\\
        0 & 0 & 0 & 0\\
        0 & 0 & 0 & 0
    \end{pmatrix} => \\
    ABx = 0 <=> \begin{pmatrix}
        2 & 4 & 1 & 3\\
        0 & 1 & 0 & 1\\
        0 & 0 & 0 & 0\\
        0 & 0 & 0 & 0\\
        0 & 0 & 0 & 0
    \end{pmatrix} \times x = 0;\\
    => x_3, x_4 \in \RR, \quad x_2 = -x_4, \quad x_1 = \frac{x_4 - x_3}{2}.
\end{gather*}


\i
Рассмотрим 2 случая: $t = 4 \And t \ne 4$.
\begin{enumerate}
    \item $t = 4:$
        \begin{gather*}
            X = \begin{pmatrix}
                -4 & 0 & -4\\
                -5 & -4 & -17\\
                0 & 0 & 0
            \end{pmatrix}\\
            \chi(X, \lambda) = -\lambda^3 -8\lambda^2 - 16\lambda = -\lambda(\lambda + 4)^2;\\
            \mu(X) \  | \ \chi(X) => \end{gather*} 
            \begin{enumerate}
                \item $\mu = \lambda$ "--- очевидно, не подходит.
                \item $\mu = \lambda(\lambda + 4), \ \mu(X) = \begin{pmatrix}
                    0 & 0 & 0\\
                    20 & 0 & 20\\
                    0 & 0 & 0
                \end{pmatrix}$ "--- не подходит.
                \item $\mu = (\lambda + 4)^2, \ \mu(X) = \begin{pmatrix}
                    0 & 0 & -16\\
                    0 & 0 & -48\\
                    0 & 0 & 16
                \end{pmatrix}$ "--- не подходит.
                \item $\mu = \lambda(\lambda + 4)^2, \ \mu(X) = \begin{pmatrix}
                    0 & 0 & 0\\
                    0 & 0 & 0\\
                    0 & 0 & 0
                \end{pmatrix}$ "--- подходит.
            \end{enumerate}
            Таким образом, при $t = 4$ минимальный многочлен данной матрицы равен $X^3 + 8X^2 + 16X$.
    
    \item $t \ne 4:$
        \begin{gather*}
            X = \begin{pmatrix}
                -4 & 0 & -t\\
                -5 & -4 & -5-3t\\
                0 & 0 & -4+t
            \end{pmatrix} \approx \begin{pmatrix}
                1 & 4 & 13\\
                0 & 16 & 48\\
                0 & 0 & -4+t
            \end{pmatrix}\\
            \chi(X) = det(X - \lambda E) = det\begin{pmatrix}
                1- \lambda & 4 & 13\\
                0 & 16 - \lambda & 48\\
                0 & 0 & -4+t - \lambda
            \end{pmatrix} = (1 - \lambda)(16 - \lambda)(t-4-\lambda)\\
            \mu(X) \ | \chi(X) =>
        \end{gather*}
        \begin{enumerate}
            \item $\mu = (1 - \lambda)$ "--- очевидно, не подходит
            \item $\mu = (16 - \lambda)$ "--- очевидно, не подходит
            \item $\mu = (t-4-\lambda)$ "--- очевидно, не подходит
            \item $\mu = (16 - \lambda)(t-4-\lambda)$ "--- не подходит
            \item $\mu = (1 - \lambda)(t-4-\lambda)$ "--- не подходит
            \item $\mu = (1 - \lambda)(16 - \lambda)$ "--- не подходит
            \item $\mu = -(1 - \lambda)(16 - \lambda)(t-4-\lambda)$ "--- подходит.
        \end{enumerate}
        Таким образом, при $t \ne 4$ минимальный многочлен данной матрицы равен $-(1 - \lambda)(16 - \lambda)(t-4-\lambda)$.
\end{enumerate}


\i \begin{gather*}
    \letus A = \begin{pmatrix}
        a_{1, 1} & a_{1, 2} & a_{1, 3}\\
        a_{2, 1} & a_{2, 2} & a_{2, 3}
    \end{pmatrix} => \\
    => A\times\begin{pmatrix}
        -9\\-7\\-3
    \end{pmatrix} = 0 \And A\times\begin{pmatrix}
        -6\\3\\2
    \end{pmatrix} = 0 \And \exists x = \begin{pmatrix}
        x_1\\x_2\\x_3
    \end{pmatrix}: A\times x = \begin{pmatrix}
        1 \\ 3
    \end{pmatrix} => \\
    => \begin{cases}
        -9a_{1,1} - 7a_{1,2} - 3a_{1,3} = 0,\\
        -9a_{2,1} - 7a_{2,2} - 3a_{2,3} = 0;
    \end{cases} \And \begin{cases}
        -6a_{1,1} + 3a_{1,2} + 2a_{1,3} = 0,\\
        -6a_{2,1} + 3a_{2,2} + 2a_{2,3} = 0;
    \end{cases}\\
    \And \begin{cases}
        x_1a_{1,1} + x_2a{1,2} + x_3a_{1,3} = 1,\\
        x_1a_{2,1} + x_2a{2,2} + x_3a_{2,3} = 3;
    \end{cases}\\
    \text{из первых двух систем получаем:}\\
    A = \begin{pmatrix}
        \frac{5}{69}a_{1,3} & \frac{-12}{23}a_{1,3} & a_{1,3}\\
        \frac{5}{69}a_{2,3} & \frac{-12}{23}a_{2,3} & a_{2,3}
    \end{pmatrix}\\
    \letus a_{1,3} = 1 \And a_{2,3} = 3 =>\\
    A = \begin{pmatrix}
        \frac{5}{69} & \frac{-12}{23} & 1\\
        \frac{15}{69} & \frac{-36}{23} & 3
    \end{pmatrix}\\
    \letus x = \begin{pmatrix}
        0\\0\\1
    \end{pmatrix} => 
    Ax = \begin{pmatrix}
        1\\3
    \end{pmatrix}
\end{gather*}
Таким образом нам удалось привести пример такой матрицы, которая удовлетворяет всем необходимым требованиям, а значит такая существует.


\i 
\begin{gather*}
    A = \begin{pmatrix}
        -1 & 0 & -4\\
        5 & -1 & 7\\
        0 & 0 & -5
    \end{pmatrix}, \quad X = \begin{pmatrix}
        x_{1, 1} & x_{1, 2} & x_{1, 3}\\
        x_{2, 1} & x_{2, 2} & x_{2, 3}\\
        x_{3, 1} & x_{3, 2} & x_{3, 3}
    \end{pmatrix}\\
    AX = XA = B;\\
    \\
    B_{1, 1} = -1x_{1, 1} + 0x_{2, 1} -4x_{3, 1} = -1x_{1, 1} + 5x_{1, 2} + 0x_{1, 3} => \\
        => 5x_{1, 2} + 4x_{3,1} = 0,\\
    B_{1, 2} = -1x_{1, 2} + 0x_{2, 2} -4x_{3, 2} = 0x_{1, 1} -1x_{1, 2} + 0x_{1, 3} =>\\
        => 4x_{3, 2} = 0,\\
    B_{1, 3} = -1x_{1, 3} + 0x_{2, 3} -4x_{3, 3} = -4x_{1, 1} + 7x_{1, 2} - 5x_{1, 3} => \\
        => -4x_{1, 1} + 7x_{1, 2} -4x_{1, 3} + 4x_{3, 3} = 0,\\
    B_{2, 1} = 5x_{1, 1} - 1x_{2, 1} + 7x_{3, 1} = -1x_{2, 1} + 5x_{2, 2} + 0x_{2, 3} => \\
        => 5x_{2, 2} - 5x_{1, 1} - 7x_{3, 1} = 0,\\
    B_{2, 2} = 5x_{1, 2} - 1x_{2, 2} + 7x_{3, 2} = 0x_{2, 1} - 1x_{2, 2} + 0x_{2, 3} =>\\
        => -5x_{1, 2} + 7x_{3, 2} = 0,\\
    B_{2, 3} = 5x_{1, 3} - 1x_{2, 3} + 7x_{3, 3} = -4x_{2, 1} +  7x_{2, 2} - 5x_{2, 3} =>\\
        => 4x_{2, 1} + 7x_{2, 2} - 4x_{2, 3} - 7x_{3, 3} = 0,\\
    B_{3, 1} = 0x_{1, 1} + 0x_{2, 1} -5x_{3, 1} = -1x_{3, 1} + 5x_{3, 2} + 0x_{3, 3} =>\\\
        => 4x_{3, 1} + 5x_{3, 2} = 0,\\
    B_{3, 2} = 0x_{1, 2} + 0x_{2, 2} -5x_{3, 2} = 0x_{3, 1} - 1x_{3, 2} + 0x_{3, 3} =>\\
        => 4x_{3, 2} = 0,\\
    B_{3, 3} = 0x_{1, 3} + 0x_{2, 3} -5x_{3, 3} = -4x_{3, 1} +  7x_{3, 2} - 5x_{3, 3} => \\
        =>7x_{3, 2} - 4x_{3, 1} = 0;\\
    \\
    \begin{pmatrix}
        \left.\begin{matrix}
            0 & 5 & 0 & 0 & 0 & 0 & 4 & 0 & 0 &\\
            0 & 0 & 0 & 0 & 0 & 0 & 0 & 4 & 0 &\\
            -4 & 7 & -4 & 0 & 0 & 0 & 0 & 0 & 4 &\\
            -5 & 0 & 0 & 0 & 5 & 0 & -7 & 0 & 0 &\\
            0 & -5 & 0 & 0 & 0 & 0 & 0 & 7 & 0 &\\
            0 & 0 & 0 & 4 & 7 & -4 & 0 & 0 & -7 &\\
            0 & 0 & 0 & 0 & 0 & 0 & 4 & 5 & 0 &\\
            0 & 0 & 0 & 0 & 0 & 0 & 0 & 4 & 0 &\\
            0 & 0 & 0 & 0 & 0 & 0 & -4 & 7 & 0 &
        \end{matrix}\right|
        \begin{matrix}
            &0 \\ &0 \\ &0 \\ &0 \\ &0 \\ &0 \\ &0 \\ &0 \\ &0
        \end{matrix}
    \end{pmatrix} => \\
    => X = \begin{pmatrix}
        x_{2, 2}
        & 0
        & -x_{2, 2} + x_{3, 3}\\
        \frac{-7}{4}x_{2, 2} + x_{2, 3} + \frac{7}{4}x_{3, 3}
        & x_{2, 2}
        & x_{2, 3}\\
        0
        & 0
        & x_{3, 3}
    \end{pmatrix}
\end{gather*}


\i \begin{gather*}
    X = \begin{pmatrix}
        -7 & -4 & 8\\
        4 & -8 & 4\\
        4 & -2 & -1
    \end{pmatrix} = AB = \begin{pmatrix}
        -4 & a_1\\
        4 & a_2\\
        3 & a_3
    \end{pmatrix} \times \begin{pmatrix}
        b_1 & b_2 & b_3\\
        -1 & -4 & 4
    \end{pmatrix} => \\
    \begin{pmatrix}
        \left.\begin{matrix}
            -4b_1 - 1a_1\\
            -4b_2 - 4a_1\\
            -4b_3 + 4a_1\\
            4b_1 - 1a_2\\
            4b_2 - 4a_2\\
            4b_3 + 4a_2\\
            3b_1 - 1a_3\\
            3b_2 - 4a_3\\
            3b_3 + 4a_3
        \end{matrix}\right|
        \begin{matrix}
            -7\\
            -4\\
            8\\
            4\\
            -8\\
            4\\
            4\\
            -2\\
            -1
        \end{matrix}
    \end{pmatrix} => 
    A = \begin{pmatrix}
        -4 & 2+b_3\\
        4 & 1-b_3\\
        3 & \frac{-1}{4}-\frac{3}{4}b_3
    \end{pmatrix} \And
    B = \begin{pmatrix}
        \frac{5}{4}-\frac{1}{4}b_3 & -1-b_3 & b_3\\
        -1 & -4 & 4
    \end{pmatrix}\\
    \letus b_3 = 1:A = \begin{pmatrix}
        -4 & 3\\
        4 & 0\\
        3 & -1
    \end{pmatrix} \And
    B = \begin{pmatrix}
        1 & -2 & 1\\
        -1 & -4 & 4
    \end{pmatrix}\\
\end{gather*}
Расписав элементы центральной диагонали матрицы, полученной в результате умножения легко убедиться, что $tr(AB) = tr(BA)$ и $tr((AB)^k) = tr((BA)^k)$, тогда:
\begin{gather*}
    tr(X^2029) = tr((AB)^{2029}) = tr((BA)^{2029}) = tr\brackets{\begin{pmatrix}
        -9 & 2\\ 0 & -7
    \end{pmatrix}^{2029}}\\
    tr\brackets{\begin{pmatrix}
        -9 & 2\\ 0 & -7
    \end{pmatrix}^{2029}}
\end{gather*}
Обозначим $\begin{pmatrix}
               -9 & 2\\ 0 & -7
          \end{pmatrix}$ как $Y$. Тогда заметим, что по индукции легко доказать, что $Y^n$ имеет вид $\begin{pmatrix}
                   (-9)^n & k\\ 0 & (-7)^n
               \end{pmatrix}$, а значит и $tr(X^{2029}) = tr(Y^{2029})$ равен в точности $(-9)^{2029} + (-7)^{2029}$.