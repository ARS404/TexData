\subsubsection{Второе ИДЗ}

\textbf{Предисловие про прогу.} К домаше я прилагаю проект $PyCharm$ а так же ссылку на репозиторий $GitHab$ с ним "--- https://github.com/ARS404/Matrix, в котором содержится весь код, который я использовал для решения. Пару слов для работы с кодом:
\begin{itemize}
    \item Вся основная логика находится в файле $matrix.py$.
    \item Локальная логика для каждой задачи (где использовался код) содержится в отдельном файле с именем вида $problem\{\textit{номер задачи}\}.py$.
    \item Входные данные для каждой задачи также содержатся в отдельном файле с именем вида $input\{\textit{номер задачи}\}.txt$.
    \item Все пояснения и ограниечения содержатся в документации к $matrix.py$ а так же в локальных файлах каждой из задач.
\end{itemize}

\i
\begin{gather*}
    \sigma(1, 7, 5, 4, 6, 3, 8, 2)\sigma = \brackets{\brackets{\substack{1\ 2\ 3\ 4\ 5\ 6\ 7\ 8\\ 4\ 6\ 3\ 5\ 7\ 1\ 2\ 8}}^{-1}\brackets{\substack{1\ 2\ 3\ 4\ 5\ 6\ 7\ 8\\ 4\ 5\ 1\ 8\ 3\ 7\ 6\ 2}}^{11}}^{161} = t;\\
    t = \brackets{\brackets{\substack{1\ 2\ 3\ 4\ 5\ 6\ 7\ 8\\ 4\ 6\ 3\ 5\ 7\ 1\ 2\ 8}}^{-1}\brackets{\substack{1\ 2\ 3\ 4\ 5\ 6\ 7\ 8\\ 4\ 5\ 1\ 8\ 3\ 7\ 6\ 2}}^{11}}^{161} = \\
    = \brackets{\brackets{\substack{1\ 2\ 3\ 4\ 5\ 6\ 7\ 8\\ 6\ 7\ 3\ 1\ 4\ 2\ 5\ 8}}\brackets{\substack{1\ 2\ 3\ 4\ 5\ 6\ 7\ 8\\ 3\ 8\ 5\ 1\ 2\ 7\ 6\ 4}}}^{161} = \brackets{\substack{1\ 2\ 3\ 4\ 5\ 6\ 7\ 8\\ 3\ 8\ 4\ 6\ 7\ 5\ 2\ 1}}^{161} = \\
    = \brackets{\substack{1\ 2\ 3\ 4\ 5\ 6\ 7\ 8\\ 3\ 8\ 4\ 6\ 7\ 5\ 2\ 1}};\\
    \sigma\brackets{\substack{1\ 2\ 3\ 4\ 5\ 6\ 7\ 8\\ 7\ 1\ 8\ 6\ 4\ 3\ 5\ 2}}\sigma = \brackets{\substack{1\ 2\ 3\ 4\ 5\ 6\ 7\ 8\\ 3\ 8\ 4\ 6\ 7\ 5\ 2\ 1}};\\
    \sigma \in \left\{ \brackets{\substack{1\ 2\ 3\ 4\ 5\ 6\ 7\ 8\\ 1\ 7\ 5\ 4\ 8\ 6\ 3\ 2}}\right., 
    \brackets{\substack{1\ 2\ 3\ 4\ 5\ 6\ 7\ 8\\ 1\ 8\ 5\ 4\ 7\ 6\ 3\ 2}},
    \brackets{\substack{1\ 2\ 3\ 4\ 5\ 6\ 7\ 8\\ 4\ 7\ 2\ 1\ 8\ 3\ 6\ 5}},
    \brackets{\substack{1\ 2\ 3\ 4\ 5\ 6\ 7\ 8\\ 4\ 8\ 2\ 1\ 7\ 3\ 6\ 5}},\\
    \brackets{\substack{1\ 2\ 3\ 4\ 5\ 6\ 7\ 8\\ 5\ 7\ 1\ 3\ 8\ 2\ 4\ 6}},
    \brackets{\substack{1\ 2\ 3\ 4\ 5\ 6\ 7\ 8\\ 5\ 8\ 1\ 3\ 7\ 2\ 4\ 6}},
    \brackets{\substack{1\ 2\ 3\ 4\ 5\ 6\ 7\ 8\\ 6\ 7\ 3\ 2\ 8\ 1\ 5\ 4}},
    \left.\brackets{\substack{1\ 2\ 3\ 4\ 5\ 6\ 7\ 8\\ 6\ 8\ 3\ 2\ 7\ 1\ 5\ 4}}\right\}.
\end{gather*}
Код использовался для полного перебора вариантов $\sigma$.

\i
\begin{gather*}
    \letus A = \begin{bmatrix}
        1 & -3 & -3 & 2\\
        1 & 2 & -1 & -2\\
        -2 & -1 & 1 & 3\\
        3 & -2 & 1 & -2
    \end{bmatrix}, \quad
    B = \begin{bmatrix}
        -8 & -2 & -6 & -4\\
        -1 & -9 & 2 & 8\\
        -2 & -3 & 5 & -7\\
        2 & 8 & -4 & 1
    \end{bmatrix},\\
    C = \begin{bmatrix}
        2 & 1 & -2 & 1\\
        -1 & -3 & -3 & 3\\
        1 & 1 & 3 & -2\\
        2 & 2 & -2 & -1
    \end{bmatrix}, \quad
    D = \begin{bmatrix}
        1 & 1 & -1 & 1\\
        -1 & -2 & 2 & -2\\
        -1 & -2 & 1 & -1\\
        -1 & -2 & 3 & -2
    \end{bmatrix}:\\
    (X+B)^{-1} = A^{-1}DC^{-1} = \begin{bmatrix}
                                    \frac{-565}{96} & \frac{941}{192} & \frac{1091}{192} & \frac{695}{192}\\
                                    \frac{-187}{48} & \frac{307}{96} & \frac{349}{96} & \frac{233}{96}\\
                                    \frac{-57}{32} & \frac{275}{192} & \frac{317}{192} & \frac{67}{64}\\
                                    \frac{-487}{96} & \frac{815}{192} & \frac{929}{192} & \frac{605}{192}
                                 \end{bmatrix};\\
    X+B = \begin{bmatrix}
            2 & -3 & -12 & 4\\
            -15 & -17 & 7 & 28\\
            12 & 2 & -7 & -13\\
            5 & 15 & -18 & -11
          \end{bmatrix};\\
    X = \begin{bmatrix}
            10 & -1 & -6 & 8\\
            -14 & -8 & 5 & 20\\
            14 & 5 & -12 & -6\\
            3 & 7 & -14 & -12
        \end{bmatrix}.
\end{gather*}
Код использовался для всех вычислений.

\i
\begin{gather*}
    A = \begin{bmatrix}
        2 & -3 & -4 & -1\\
        4 & 2 & 3 & -1\\
        1 & -1 & -4 & 1\\
        0 & 0 & -5 & 3\\
    \end{bmatrix}\\
    \chi_A(\lambda) = \lambda^4 + (-3)\lambda^3 + (12)\lambda^2 + (2)\lambda^1 + (-14)\\
    X = (A^2 -3A + 2E)^{-2} = \brackets{(A^2 - 3A + 2E)^2}^{-1} = \\
    = \begin{bmatrix}
        134 & -14 & -54 & -10\\
        32 & 165 & -26 & 137\\
        16 & -23 & 48 & -39\\
        -50 & -55 & 90 & -71
      \end{bmatrix}^{-1} = \begin{bmatrix}
        \frac{10201}{450} & \frac{483}{50} & \frac{-8881}{150} & \frac{10793}{225}\\
        \frac{-112277}{900} & \frac{-1329}{25} & \frac{97787}{300} & \frac{-237647}{900}\\
        \frac{26837}{450} & \frac{1271}{50} & \frac{-11686}{75} & \frac{56807}{450}\\
        \frac{28129}{180} & \frac{333}{5} & \frac{-24499}{60} & \frac{59539}{180}
      \end{bmatrix};\\
      \det X = \frac{1}{3600}.
\end{gather*}
Код использовался для всех вычислений.

\i
\begin{gather*}
    A = 
    \begin{bmatrix}
        -3 & -1 & 4 & -5 & 1 & -8 & \textbf{x}\\
        4 & 4 & -6 & -8 & \textbf{x} & -9 & 6\\
        \textbf{x} & 3 & -2 & 8 & 9 & 4 & -2\\
        9 & -8 & -6 & 8 & 8 & \textbf{x} & -9\\
        -8 & -6 & 1 & 7 & 9 & 2 & \textbf{x}\\
        2 & \textbf{x} & -3 & \textbf{x} & 4 & -7 & 1\\
        -6 & 6 & \textbf{x} & 1 & -1 & -2 & 3
    \end{bmatrix}\\
    \det A = 0x^7 - 17x^6 + 54x^5 + 2928x^4 + 7958x^3 - 82454x^2 + 147893x + 1568098.\\
    \text{Таким образом, коэфицент при $x^5$ равен } 54.
\end{gather*}
Код использовался для всех вычислений.

\i
\begin{gather*}
    AB = \begin{bmatrix}
            4 & 1 & 1 & 1 & -5\\
            -8 & -2 & -2 & -2 & 10\\
            12 & 3 & 3 & 3 & -15\\
            -12 & -3 & -3 & -3 & 15\\
            -4 & -1 & -1 & -1 & 5
         \end{bmatrix}\\
    \chi_{AB}(\lambda) = \lambda^{5} + (-7)\lambda^4 + (0)\lambda^3 + (0)\lambda^2 + (0)\lambda^1 + (0) = \lambda^{5} -7\lambda^4
\end{gather*}
Код использовался для всех вычислений.