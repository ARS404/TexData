\subsection{Третье ИДЗ}

\i Для начала запишем данную систему уравнений в виде матрицы и приведём ё к более удобному виду:
\begin{gather*}
    \begin{pmatrix}
        2 & 6 & -1 & -4 & 4 & 3 \\
        1 & 3 & -1 & -3 & 3 & 2 \\
        1 & 3 & -1 & -3 & 3 & 2
    \end{pmatrix} \sim
    \begin{pmatrix}
        1 & 3 & -1 & -3 & 3 & 2 \\
        2 & 6 & -1 & -4 & 4 & 3 \\
        0 & 0 & 0 & 0 & 0 & 0
    \end{pmatrix} \sim 
    \begin{pmatrix}
        1 & 3 & -1 & -3 & 3 & 2 \\
        0 & 0 & 1 & 2 & -2 & -1 \\
        0 & 0 & 0 & 0 & 0 & 0
    \end{pmatrix}
\end{gather*}
Теперь подставим вектора $v_1$ и $v_2$ в полученную систему:
\begin{gather*}
    v_1 = (-7, 2, -1, 1, -1, 3):\\
    \begin{cases}
        1 \cdot (-7) + 3 \cdot 2 + (-1) \cdot (-1) + (-3) \cdot 1 + 3 \cdot (-1) + 2 \cdot 3 = 0,\\
        0 \cdot (-7) + 0 \cdot 2 + 1 \cdot (-1) + 2 \cdot 1 + (-1) \cdot (-1) + (-1) \cdot 3 = 0,\\
        0 = 0; 
    \end{cases}\\
    v_2 = (-8, 2, 3, -2, -1, 1):\\
    \begin{cases}
        1 \cdot (-8) + 3 \cdot 2 + (-1) \cdot 3 + (-3) \cdot (-2) + 3 \cdot (-1) + 2 \cdot 1 = 0,\\
        0 \cdot (-7) + 0 \cdot 2 + 1 \cdot 3 + 2 \cdot (-2) + (-1) \cdot (-1) + (-1) \cdot 1 = 0,\\
        0 = 0; 
    \end{cases}\\
\end{gather*}
Заметим, что в приведённой форме матрица имеет 2 главные переменные и 4 независимые, а значит размерность базиса решений равна 4. Таким образом, нам осталось найти 2 вектора, чтобы с двумя данными они образовывали линейно независимую систему. Выпишем общий вид решения данной системы:
$$x = (-3x_2 + x_4 - x_5 - x_6,\ x_2, x_6 + 2x_5 - 2x_4,\ x_4,\ x_5,\ x_6)$$
Распишем $v_1$ и $v_2$ по базису, состоящему из векторов $x_2,\ x_4,\ x_5,\ x_6$, получим:
$$v_1 \seek (3, -3, 3, 2),$$
$$v_2 \seek (0, 2, -1, -1);$$
Давайте найдём ещё 2 вектора, которые будут решениями, и их значения в независимых переменных будут линейно независимы от значений $v_1$ и $v_2$. На самом деле, легко убедится, что вектора $(0, 0, 1, 0)$ и $(0, 0, 0, 1)$ подойдут, давайте проверим:
\begin{gather*}
    \begin{pmatrix}
        3 & -3 & 3 & 2\\
        0 & 2 & -1 & -1\\
        0 & 0 & 1 & 0\\
        0 & 0 & 0 & 1
    \end{pmatrix} \sim
    \begin{pmatrix}
        3 & -3 & 0 & 0\\
        0 & 2 & 0 & 0\\
        0 & 0 & 1 & 0\\
        0 & 0 & 0 & 1
    \end{pmatrix} \sim
    \begin{pmatrix}
        1 & 0 & 0 & 0\\
        0 & 1 & 0 & 0\\
        0 & 0 & 1 & 0\\
        0 & 0 & 0 & 1
    \end{pmatrix}
\end{gather*}
Тогда искомыми векторами будут $v_3 = (-1, 0, 2, 0, 1, 0)$ и $v_4 = (-1, 0, 1, 0, 0, 1)$ (легкто подставить их в исходную систему и убедиться, что они подходят, также, по выше написанному понятно, что они образуют с $v_1$ и $v_2$ линейнонезависимую систему, а значит отбразуют базис системы решенй).

\i 
\begin{gather*}
    A = 
    \begin{pmatrix}
        -17 & 14 & 5 & -7\\
        -17 & 14 & 5 & -7\\
        17 & -14 & -5 & 7\\
        20 & -14 & 10 & 14
    \end{pmatrix}
\end{gather*}
Для начала приведём нашу матрицу к более удобному виду:
\begin{gather*}
    A \sim
    \begin{pmatrix}
        20 & -14 & 10 & 14\\
        17 & -14 & -5 & 7\\
        0 & 0 & 0 & 0\\
        0 & 0 & 0 & 0
    \end{pmatrix} \sim 
    \begin{pmatrix}
        -42 & 0 & -210 & -98\\
        0 & -42 & -270 & -98\\
        0 & 0 & 0 & 0\\
        0 & 0 & 0 & 0
    \end{pmatrix} \sim
    \begin{pmatrix}
        21 & 0 & 105 & 49\\
        0 & 21 & 135 & 49\\
        0 & 0 & 0 & 0\\
        0 & 0 & 0 & 0
    \end{pmatrix}
\end{gather*}
Вот теперь можно найти ФСР (которая также, очевидно, является базисом ядра). Всего у нас получилось 2 свободных переменных, а значит следующие 2 вектора являются ФСР (их линейная независимость очевидна):
\begin{gather*}
    v_1 = (-5, 0, 1, 0),\\
    v_2 = (0, -7, 0, 3);
\end{gather*}
Давайте найдём базис образа. Как известно, это бизс линейной оболочки столбцов. Мы уже знаем ранг нашей матрицы (это 2), а значит размерность образа также равна 2, таким образом, нам достаточно выбрать 2 линейнонезависимых столбца, очевидно, первый и второй нам подходят.\\
Найдём пересечение следующим образом: сперва запишем вектора обоих базисов по столбцам, тогда мы получим вектора, которые расписываются в обоих базисах, то есть те, которые принадлежат пересечению ядра и образа.
\begin{gather*}
    \begin{pmatrix}
        -5 & 0 & -17 & 14\\
        0 & -7 & -17 & 14\\
        1 & 0 & 17 & -14\\
        0 & 3 & 20 & -14
    \end{pmatrix} \sim
    \begin{pmatrix}
        -5 & 0 & -17 & 14\\
        -85 & 119 & 0 & 0\\
        68 & 0 & 0 & 0\\
        100 & -51 & 0 & -42
    \end{pmatrix} \sim 
    \begin{pmatrix}
        0 & 0 & 168 & 0\\
        0 & 0 & 0 & 168\\
        168 & 0 & 0 & 0\\
        0 & 168 & 0 & 0
    \end{pmatrix}
\end{gather*}
Таким образом, пересечение ядра и образа состоит только из нуля (то есть, базис пересечения пуст), а значит их сумма прямая, и её размерность составляет сумму размерностьей ядра и образа, то есть, равна 4. Таким образом, её базисом, очевидно, будет являться, например набор из векторов:
\begin{gather*}
    e_1 = (1, 0, 0, 0),\\
    e_2 = (0, 2, 0, 0),\\
    e_3 = (0, 0, 3, 0),\\
    e_4 = (0, 0, 0, 4).
\end{gather*}

\i По лемме о стабилизации $Im\phi^{2019} = Im\phi^4$, аналогично $ker\psi^{2019} = ker\psi^4$.
\begin{gather*}
    \phi^4 = A^4 = 
    \begin{pmatrix}
        32 & -16 & 0 & 16\\
        32 & -16 & 16 & 32\\
        0 & 0 & 32 & 32\\
        0 & 0 & -16 & -16
    \end{pmatrix}\\
    \psi^4 = (A^t)^4 = (A^4)^t = 
    \begin{pmatrix}
        32 & 32 & 0 & 0\\
        -16 & -16 & 0 & 0\\
        0 & 16 & 32 & -16\\
        16 & 32 & 32 & -16
    \end{pmatrix}
\end{gather*}
Легко понять, что ранг матрицы оператора $\phi^4$ равен 2, поэтому базис образа также будет иметь размерность 2, а значит в качестве него можно рассмотреть любые 2 линейно независимые столбца, например, 2 и 3. Чтобы найти базис ядра $\psi^4$ найдём ФСР соответствующей матрицы:
\begin{gather*}
    \begin{pmatrix}
        32 & 32 & 0 & 0\\
        -16 & -16 & 0 & 0\\
        0 & 16 & 32 & -16\\
        16 & 32 & 32 & -16
    \end{pmatrix} \sim
    \begin{pmatrix}
        2 & 2 & 0 & 0\\
        -1 & -1 & 0 & 0\\
        0 & 1 & 2 & -1\\
        1 & 2 & 2 & -1
    \end{pmatrix} \sim
    \begin{pmatrix}
        1 & 2 & 2 & -1\\
        0 & -1 & -2 & 1\\
        0 & 1 & 2 & -1\\
        0 & 1 & 2 & -1
    \end{pmatrix} \sim 
    \begin{pmatrix}
        1 & 2 & 2 & -1\\
        0 & 1 & 2 & -1\\
        0 & 0 & 0 & 0\\
        0 & 0 & 0 & 0
    \end{pmatrix}
\end{gather*}
Из этого очевидно следует, что вектора $(-1, 1, 0, 1)$ и $(2, -2, 1, 0)$ Являются ФСР. Теперь покажем, что базисы ядра и образа наших операторов линейнонезависимы:
\begin{gather*}
    \begin{pmatrix}
        -16 & -16 & 0 & 0\\
        0 & 16 & 32 & -16\\
        -1 & 1 & 0 & 1\\
        2 & -2 & 1 & 0
    \end{pmatrix} \sim 
    \begin{pmatrix}
        -1 & -1 & 0 & 0\\
        -1 & 1 & 0 & 1\\
        2 & -2 & 1 & 0\\
        0 & 1 & 2 & -1
    \end{pmatrix} \sim
    \begin{pmatrix}
        1 & 1 & 0 & 0\\
        0 & -2 & 0 & -1\\
        0 & -4 & 1 & 0\\
        0 & 1 & 2 & -1
    \end{pmatrix} \sim
    \begin{pmatrix}
        1 & 1 & 0 & 0\\
        0 & 1 & 2 & -1\\
        0 & 0 & 4 & -3\\
        0 & 0 & 9 & -4
    \end{pmatrix}
\end{gather*}
Отсюда уже видно, что бизисы действительно линейнонезависимы, тогда мы можем разложить $R^4$ в прямую суммы. Осталось разложить вектор $v$. Для этого решим систему, где столбцами будут наши базисные вектора, а в столбще свободных коэффицентов будет сам вектор $v$, тогда решением будут его координаты. Получим, что $$v = -4(-1, -1, 0, 0) + (-1)(0, 1, 2, -1) + 5(-1, 1, 0, 1) + 5(2, -2, 1, 0).$$

\i Заметим, что характеристический многочлен данной матрицы равен $(\lambda-1)^2\lambda^2$, то есть, $\lambda_1 = 1$ и $\lambda_2 = 0$. Теперь рассмотрим разложение нашего знуляющего многочлена на 2 взаимнопростых: $p = (\lambda - 1)^2$ и $q = \lambda^2$. Тогда можно применить известный факт, что $Im\ q(\phi) = ker\ p(\phi)$. При этом, $ker\ p(\phi) = V^1$ (потому что $ker\ p(\phi)$ стабилизируется со втогорой степени) и $Im\ q = Im \phi^2$. Таким образом, очевидно, что $\psi = \phi^2$ подходит.\\
*Для подсчёта характеристического многочлена я использовл тот же код, который прикладывал к прошлому БДЗ (ссылка на $GitHab$ с ним "--- https://github.com/ARS404/Matrix).

\i Пусть наш оператор задан матрицей $A_{\phi}\hatch$, тогда мы заем, что его представление в новом базисе имеет вид $C^{-1}A_{\phi}C$, где $C$ "--- матрица перехода к новому базису. Для начала подставим в это равенство первый базис (матрица перехода к нему, очевидно, равна <<склеенным>> столбцам из самого базиса):
\begin{gather*}
    A_{\phi}\hatch = C^{-1}AC;\\
    CA_{\phi}\hatch = AC.
\end{gather*}
Тогда можно подставить в это равенство произвольный вектор (домножить на него справа). Давайте в качестве него возьмём вектор $v = \begin{pmatrix} 1\\ 0\\ 0 \end{pmatrix}$, тогда столбцы с ошибками занулятся, получим:
\begin{gather*}
    CA_{\phi}\hatch v = A_{\phi}Cv;\\
    \begin{pmatrix}
        2 & 2 & 5\\
        1 & 2 & -1\\
        1 & 1 & 1
    \end{pmatrix} \times A_{\phi}\hatch \times \begin{pmatrix} 1\\ 0\\ 0 \end{pmatrix} = A_{\phi} \times \begin{pmatrix}
        2 & 2 & 5\\
        1 & 2 & -1\\
        1 & 1 & 1
    \end{pmatrix} \times \begin{pmatrix} 1\\ 0\\ 0 \end{pmatrix};\\
    \begin{pmatrix}
        2 & 2 & 5\\
        1 & 2 & -1\\
        1 & 1 & 1
    \end{pmatrix} \times A_{\phi}\hatch \times \begin{pmatrix} 1\\ 0\\ 0 \end{pmatrix} = A_{\phi} \times \begin{pmatrix}
        2\\ 1\\ 1
    \end{pmatrix};\\
    \begin{pmatrix}
        2 & 2 & 5\\
        1 & 2 & -1\\
        1 & 1 & 1
    \end{pmatrix} \times \begin{pmatrix} -2\\ 3\\ 1 \end{pmatrix} = A_{\phi} \times \begin{pmatrix}
        2\\ 1\\ 1
    \end{pmatrix};\\
    \begin{pmatrix} 7\\ 3\\ 2 \end{pmatrix} = A_{\phi} \times \begin{pmatrix} 2\\ 1\\ 1\end{pmatrix};
\end{gather*}
Осталось повторить эту операцию для второго базиса (сперва с $v = (0, 1, 0)$, затем с $v = (0, 0, 1)$):
\begin{gather*}
    \begin{pmatrix}
        5 & 1 & 2\\
        3 & 2 & 3\\
        -1 & 0 & 0
    \end{pmatrix} \times A_{\phi}\hatch\hatch \times \begin{pmatrix} 0\\ 1\\ 0\end{pmatrix} = A_{\phi} \times \begin{pmatrix} 1\\ 2\\ 0 \end{pmatrix};\\
    \begin{pmatrix} -12\\ 1\\ 4 \end{pmatrix} = A_{\phi} \times \begin{pmatrix} 1\\ 2\\ 0\end{pmatrix};\\ \\
    \begin{pmatrix}
        5 & 1 & 2\\
        3 & 2 & 3\\
        -1 & 0 & 0
    \end{pmatrix} \times A_{\phi}\hatch\hatch \times \begin{pmatrix} 0\\ 0\\ 1\end{pmatrix} = A_{\phi} \times \begin{pmatrix} 2\\ 3\\ 0\end{pmatrix};\\
    \begin{pmatrix} -15\\ -3\\ 4\end{pmatrix} = A_{\phi} \times \begin{pmatrix} 2\\ 3\\ 0\end{pmatrix};
\end{gather*}
Теперь склеим всё это дело в одну матрицу, получим:
\begin{gather*}
    A_{\phi} \times \begin{pmatrix}
        2 & 1 & 2\\
        1 & 2 & 3\\
        1 & 0 & 0
    \end{pmatrix} = \begin{pmatrix}
        7 & -12 & -15\\
        3 & 1 & -3\\
        2 & 4 & 4
    \end{pmatrix};\\
    A_{\phi} = \begin{pmatrix}
        6 & -9 & 4\\
        -9 & 5 & 16\\
        -4 & 4 & 16
    \end{pmatrix}.
\end{gather*}
Таким образом, мы нашли искомую матрицу $A_{\phi}$.\\
*Для подсчёта характеристического многочлена я использовл тот же код, который прикладывал к прошлому БДЗ (ссылка на $GitHab$ с ним "--- https://github.com/ARS404/Matrix).