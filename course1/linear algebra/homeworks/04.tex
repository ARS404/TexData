\subsection{Четвёртое ИДЗ}

\textit{Для начала стоит сказать, что для всех операций с матрицами (арифметические действия, поиск решений и ФРС, нахождение хар. многочлена, взятие ранга и определителя и тд.) я использовал свой код, вот ссылка на гитхаб с ним}\\
\url{https://github.com/ARS404/Matrix}
\\

\i Сразу скажем, что у матрицы будут задавать один и тот же линейный оператор тогда и только тогда, когда их ЖНФ совпадают. Для поиска ЖНФ мы хотим воспользоваться известным нам алгоритмом (номер 32), поэтому проверим, что у всех 3 матриц только одно собственное значение. посчитаем их характеристические многочлены:
\begin{gather*}
    \chi_A(\lambda) = (\lambda + 4)^4;\\
    \chi_B(\lambda) = (\lambda + 4)^4;\\
    \chi_C(\lambda) = (\lambda + 4)^4.
\end{gather*}
Таким образом, мы можем применить известный нам алгоритм и получить:
\begin{gather*}
    \begin{pmatrix}
        -5 & -5 & -3 & 1\\
        1 & 4 & 5 & -2\\
        -1 & -8 & -9 & 2\\
        1 & 8 & 5 & -6
    \end{pmatrix} \sim \begin{pmatrix}
        -4 & 0 & 0 & 0\\
        0 & -4 & 1 & 0\\
        0 & 0 & -4 & 1\\
        0 & 0 & 0 & -4
    \end{pmatrix} \\ 
    \begin{pmatrix}
        -5 & -5 & 1 & -3\\
        1 & 2 & -1 & 4\\
        1 & 6 & -5 & 4\\
        -1 & -6 & 1 & -8
    \end{pmatrix} \sim \begin{pmatrix}
        -4 & 0 & 0 & 0\\
        0 & -4 & 1 & 0\\
        0 & 0 & -4 & 1\\
        0 & 0 & 0 & -4
    \end{pmatrix}\\
    \begin{pmatrix}
        -7 & 2 & 2 & 3\\
        5 & -6 & -3 & -4\\
        1 & 2 & -4 & 1\\
        -7 & 2 & 4 & 1
    \end{pmatrix} \sim \begin{pmatrix}
        -4 & 1 & 0 & 0\\
        0 & -4 & 0 & 0\\
        0 & 0 & -4 & 1\\
        0 & 0 & 0 & -4
    \end{pmatrix}
\end{gather*}
Таким образом, матрицы $A$ и $B$ задают один и тот же линейный оператор, а $C$ "--- другой.\\


\i Воспользуеся общим алгоритмом поиска ЖНФ (номер 27). Дня начала найдём собственные значения (хар. многочлен посчитали прогой):
$$\chi_A(\lambda) = \lambda^6 -22\lambda^5 + 201\lambda^4 - 976\lambda^3 + 2656\lambda^2 -3840\lambda + 2304.$$
Тогда собственные значения это:\\
$\lambda_1 = 4$ (кратности 4);\\
$\lambda_3 = 3$ (кратности 2).\\
Осталось посчитать количество клеток каждого размера для обоих собственных значений. Получим, что для 4 у нас 2 клетки размера 2, а для 3 "--- одна клетка размера 2, таким образом, ЖНФ имеет вид
$$\begin{pmatrix}
    3 & 1 & 0 & 0 & 0 & 0\\
    0 & 3 & 0 & 0 & 0 & 0\\
    0 & 0 & 4 & 1 & 0 & 0\\
    0 & 0 & 0 & 4 & 0 & 0\\
    0 & 0 & 0 & 0 & 4 & 1\\
    0 & 0 & 0 & 0 & 0 & 4
\end{pmatrix}$$
Теперь будем искать жорданов базис (страдать). Пусть он состоит из вектор-столбцов $e_1, \ldots, e_6$. Для начала для первой клетки (с тройкой). Для начала мы хотим найти такой $e_2$, что $(A - 3E)^2e_2 = 0$, и $(A - 3E)e_2 \ne 0$. Поищем $e_2$ среди решений $(A - 3E)^2$:
\begin{gather*}
    (A - 3E)^2\begin{pmatrix}
        -1\\1\\1\\1\\2\\0
    \end{pmatrix} = \begin{pmatrix}
        0\\0\\0\\0\\0\\0
    \end{pmatrix},\quad
    (A - 3E)\begin{pmatrix}
        -1\\1\\1\\1\\2\\0
    \end{pmatrix} = \begin{pmatrix}
        0\\0\\0\\0\\0\\1
    \end{pmatrix}\\
\end{gather*}
Таким образом, в качестве $e_2$ и $e_1$ можно взять $\begin{pmatrix}-1\\1\\1\\1\\2\\0\end{pmatrix}$ и $\begin{pmatrix}0\\0\\0\\0\\0\\1\end{pmatrix}$ соответственно.
К $e_4$ и $e_6$ мы видвинем аналогичные требования (только с собственным значением 4).
\begin{gather*}
    (A - 4E)^2\begin{pmatrix}
        -2\\4\\8\\0\\4\\0
    \end{pmatrix} = \begin{pmatrix}
        0\\0\\0\\0\\0\\0
    \end{pmatrix},\quad
    (A - 4E)\begin{pmatrix}
        -2\\4\\8\\0\\4\\0
    \end{pmatrix} = \begin{pmatrix}
        -4\\0\\0\\16\\16\\16
    \end{pmatrix}\\
    (A - 4E)^2\begin{pmatrix}
        -2\\4\\8\\0\\0\\4
    \end{pmatrix} = \begin{pmatrix}
        0\\0\\0\\0\\0\\0
    \end{pmatrix},\quad
    (A - 4E)\begin{pmatrix}
        -2\\4\\8\\0\\0\\4
    \end{pmatrix} = \begin{pmatrix}
        -4\\0\\0\\40\\24\\20
    \end{pmatrix}\\
\end{gather*}
Таким образом мы нашли оставшиеся базисные вектора. Осталось лишь проверить, что эти вектора реально линейно независимы. Для этого склеим их в матрицу и посчитаем её ранг. Склеили, посчитали, порадовались. Таким образом, искомый базис равен
\begin{gather*}
    \begin{pmatrix}
        0\\0\\0\\0\\0\\1
    \end{pmatrix},
    \begin{pmatrix}
        -1\\1\\1\\1\\2\\0
    \end{pmatrix},
    \begin{pmatrix}
        -4\\0\\0\\16\\16\\16
    \end{pmatrix},
    \begin{pmatrix}
        -2\\4\\8\\0\\4\\0
    \end{pmatrix},
    \begin{pmatrix}
        -4\\0\\0\\20\\24\\20
    \end{pmatrix},
    \begin{pmatrix}
        -2\\4\\8\\0\\0\\4
    \end{pmatrix}.
\end{gather*}\\


\i Пусть $G = (g_1 | g_2 | g_3)$ и $F = (f_1 | f_2 | f_3)$, тогда $A_\phi = G^{-1}AG$ (в исходном базисе). Аналогично $B_\psi = F^{-1}BF$. Давайте посчитаем эти матрицы:
\begin{gather*}
    A_\phi = \begin{pmatrix}
        33 & 75 & -102\\
        -30 & -75 & 105\\
        -11 & -30 & 43
    \end{pmatrix},\quad 
    B_\psi = \begin{pmatrix}
        -33 & 75 & -75\\
        -28 & 62 & -60\\
        -14 & 31 & -30
    \end{pmatrix}
\end{gather*}
Матрица композиции это произведение матриц, тогда матрица $\phi \circ \psi$ в стандартном базисе равна $A_\phi B_\psi$ или $\begin{pmatrix}
    -1761 & 3963 & -3915\\
    1620 & -3645 & 3600\\
    601 & -1352 & 1335
\end{pmatrix}$
Давайте теперь найдём ядра наших операторов. Как известно, базис ядра совпадает с ФСР:\\
для $A_\phi$ это $\begin{pmatrix}
    -5\\9\\5
\end{pmatrix}$,\\
для $B_\psi$ это $\begin{pmatrix}
    25\\20\\9
\end{pmatrix}$.
Очевидно, что эти вектора линейно независимы, а значит и ядра $A_\phi$ и $B_\psi$ также линейно независимы, то есть базис суммы ядер это объединение базисов ядер, например
$$\begin{pmatrix}
    -5\\9\\5
\end{pmatrix},
\begin{pmatrix}
    25\\20\\9
\end{pmatrix}.$$
Мы уже понял, что ранги наших матриц равны 2, а значит количество линейно независимых столбцов у обоих матриц равно 2, тогда образы первого и второго оператора равны соответственно
$$<\begin{pmatrix}
    33\\-30\\-11
\end{pmatrix},
\begin{pmatrix}
    75\\-75\\-30
\end{pmatrix}>, \quad
<\begin{pmatrix}
    -33\\-28\\-14
\end{pmatrix},
\begin{pmatrix}
    75\\62\\31
\end{pmatrix}>$$
Теперь воспользуемся алгоритмом для нахождения базиса пересечения (номер 11). Для начала найдем ФСР <<склееных>> базисов наших образов. Получим $\begin{pmatrix}
    135\\-72\\2960\\1315
\end{pmatrix}$. Тогда искомый базис равен
\begin{gather*}
    \begin{pmatrix}
        -97680\\-82880\\-41440
    \end{pmatrix},\quad
    \begin{pmatrix}
        98625\\81530\\40765
    \end{pmatrix}.
\end{gather*}\\

\i Для начала поймём, что при сопряжении $A_t$ произвольной невыражденной матрицей $B$ множество коммутирующих с ней матриц не меняется. В самом деле:
\begin{gather*}
    A_t\hatch = B^{-1}A_tB;\\
    A_t\hatch X\hatch = X\hatch A_t\hatch <=> B^{-1}ABX\hatch = X\hatch B^{-1}AB <=>\\
    <=> A_t\brackets{B^ X\hatch B^{-1}} = \brackets{B X\hatch B^{-1}}A_t;\\
    B X\hatch B^{-1} \in L_t => \letus B X\hatch B^{-1} = X:\\
    <=> A_t X = X A_t.
\end{gather*}
При этом $X$ диагонализуема тогда и только тогда, когда диагонализуема $X\hatch$ так как при сопряжении диагонализуемость не меняется. Таким образом, при сопряжении $A_t$ произвольной матрицей $B$ все матрицы из $L_t$ будут сопряжены $B$.\\
Теперь воспользуемся алгоритмом, чтобы найти ЖНФ $A_t$.
\begin{gather*}
    \chi_{A_t}(\lambda) = (\lambda + 3)^2(2 - \lambda)^3
\end{gather*}
Таким образом, собственные значения это -3 (кратности 2) и 2 (кратности 3). Теперь рассмотрим ЖНФ $A_t$ в зависимости от значений параметра $t$.
\begin{itemize}
    \item $t = 0$ "--- $\begin{pmatrix}
        -3 & 1 & 0 & 0 & 0\\
        0 & -3 & 0 & 0 & 0\\
        0 & 0 & 2 & 1 & 0 \\
        0 & 0 & 0 & 2 & 0\\
        0 & 0 & 0 & 0 & 2
    \end{pmatrix}$
    
    \item $t = -1$ "--- $\begin{pmatrix}
        -3 & 1 & 0 & 0 & 0\\
        0 & -3 & 0 & 0 & 0\\
        0 & 0 & 2 & 1 & 0 \\
        0 & 0 & 0 & 2 & 1\\
        0 & 0 & 0 & 0 & 2
    \end{pmatrix}$
    
    \item $t = 1$ "--- $\begin{pmatrix}
        -3 & 1 & 0 & 0 & 0\\
        0 & -3 & 0 & 0 & 0\\
        0 & 0 & 2 & 1 & 0 \\
        0 & 0 & 0 & 2 & 1\\
        0 & 0 & 0 & 0 & 2
    \end{pmatrix}$
    
    \item $t \not\in \{-1, 0, 1\}$.\\
    Тут интереснее, придётся ручками считать ранги нужных матриц (мой код не умеет в матрицы с переменными), так что начнём.\\
    Для начала определимся с количествок клеток размера 3 для двойки, оно равно $rk(A_t - 2E)^4 + rk(A_t - 2E)^2 - 2rk(A_t - 2E)^3$ или
    \begin{gather*}
        rk \begin{pmatrix}
            0 & 0 & 0 & 0 & 0\\
            0 & 125 & 0 & 0 & -100\\
            0 & -25(t+1) & 0 & 0 & 15(t+1)\\
            0 & 5(t+1) & 0 & 0 & 23-27t\\
            0 & 0 & 0 & 0 & 125
        \end{pmatrix} + rk\begin{pmatrix}
            0 & 0 & 0 & 0 & 0\\
            0 & 25 & 0 & 0 & -10\\
            0 & -5(t+1) & 0 & 0 & t + 1\\
            t & t+1 & 0 & 0 & -5(1-t)\\
            0 & 0 & 0 & 0 & 25
        \end{pmatrix} - \\ - 2rk \begin{pmatrix}
            0 & 0 & 0 & 0 & 0\\
            0 & -125 & 0 & 0 & 75\\
            0 & 25(t+1) & 0 & 0 & -10(t+1)\\
            0 & -5(t+1) & 0 & 0 & 26t-24\\
            0 & 0 & 0 & 0 & -125
        \end{pmatrix} = 2 + 3 - 4 = 1.
    \end{gather*}
    Таким образом, все для 2 есть только одна клетка размера 3.\\
    разберёмся с клетками размера 1 для 3. Их количество равно $rk(A_t+3E)^2 + rk(A_t+3E)^0 - 2rk(A_t+3E)^1$ или
    \begin{gather*}
        rk \begin{pmatrix}
            25 & 0 & 0 & 0 & 0\\
            0 & 0 & 0 & 0 & 0\\
            10t & 5(t+1) & 25 & 0 & t+1\\
            t & t+1 & 10 & 25 & 5(1-t)\\
            0 & 0 & 0 & 0 & 0
        \end{pmatrix} + 5 - 2rk \begin{pmatrix}
            5 & 0 & 0 & 0 & 0\\
            0 & 0 & 0 & 0 & 1\\
            t & t+1 & 5 & 0 & 0\\
            0 & 0 & 1 & 5 & t-1\\
            0 & 0 & 0 & 0 & 0
        \end{pmatrix} = \\ = 3 + 5 - 8 = 0.
    \end{gather*}
    Это значит, что клеток размера 1 для -3 нет, тогда все клетки размера 2. В итоге мы получили, что в этом случае ЖНФ имеет вид $\begin{pmatrix}
        -3 & 1 & 0 & 0 & 0\\
        0 & -3 & 0 & 0 & 0\\
        0 & 0 & 2 & 1 & 0\\
        0 & 0 & 0 & 2 & 1\\
        0 & 0 & 0 & 0 & 2
    \end{pmatrix}$
\end{itemize}
Таким образом, у нас есть 2 вариатна того, как может выглядеть ЖНФ от $A_t$. Осталось разобраться с тем, какие размерности будут у пространств, которые пораждаются каждой из возможных форм.
\begin{itemize}
    \item Для начала рассмотрим вариант $t = 0$. Найдём все такие матрицы $X$, что они коммутируют с $A_t$ (тоже самое, что коммутируют с её ЖНФ (дялее $A\hatch$)):
    \begin{gather*}
        \letus X = \begin{pmatrix}
            x_{11} & x_{12} & x_{13} & x_{14} & x_{15}\\
            x_{21} & x_{22} & x_{23} & x_{24} & x_{25}\\
            x_{31} & x_{32} & x_{33} & x_{34} & x_{35}\\
            x_{41} & x_{42} & x_{43} & x_{44} & x_{45}\\
            x_{51} & x_{52} & x_{53} & x_{54} & x_{55}
        \end{pmatrix}:\\
        A\hatch X - X A\hatch = 0 = \\ =
        \begin{pmatrix}
            x_{21} & x_{22}-x_{11} & x_{23}-5x_{13} & -x_{13}-5x_{14}+x_{24} & x_{25}-5x_{15}\\
            0 & -x_{21} & -5x_{23} & -x_{23}-5x_{24} & -5x_{25}\\
            5x_{31}+x_{41} & -x_{31}+5x_{32}+x_{42} & x_{43} & x_{44}-x_{33} & x_{45}\\
            5x_{41} & 5x_{42}-x_{41} & 0 & -x_{43} & 0\\
            5x_{51} & 5x_{52}-x_{51} & 0 & -x_{53} & 0
        \end{pmatrix};
        \intertext{так как эта матрица должна равняться нулевой, то все $x_{ij}$, которые в какой-то клетке стоят в одиночестве автоматом приравниваются к нулю, тогда получим:}
        \begin{pmatrix}
            0 & x_{22}-x_{11} & 0-5x_{13} & -x_{13}-5x_{14}+x_{24} & 0-5x_{15}\\
            0 & 0 & 0 & 0-5x_{24} & 0\\
            5x_{31}+0 & -x_{31}+5x_{32}+x_{42} & 0 & x_{44}-x_{33} & 0\\
            0 & 5x_{42}-0 & 0 & 0 & 0\\
            0 & 5x_{52}-0 & 0 & 0 & 0
        \end{pmatrix};
        \intertext{для <<исправленной>> матрицы можно проделать тоже самое:}
        \begin{pmatrix}
            0 & x_{22}-x_{11} & 0 & 0-5x_{14}+0 & 0\\
            0 & 0 & 0 & 0 & 0\\
            0 & 0+5x_{32}+0 & 0 & x_{44}-x_{33} & 0\\
            0 & 0 & 0 & 0 & 0\\
            0 & 0 & 0 & 0 & 0
        \end{pmatrix};
        \intertext{ну всё, осталось совсем немного:}
        \begin{pmatrix}
            0 & x_{22}-x_{11} & 0 & 0 & 0\\
            0 & 0 & 0 & 0 & 0\\
            0 & 0 & 0 & x_{44}-x_{33} & 0\\
            0 & 0 & 0 & 0 & 0\\
            0 & 0 & 0 & 0 & 0
        \end{pmatrix}.
    \end{gather*}
    Однако, стоит заметить, что переменные $x_{12}, x_{34}, x_{35}, x_{54}, x_{55}$ сократились при подсчёте коммутатора, а значит они могут быть любыми.\\
    Значит для того, чтобы $X$ коммутировала с $A_t$ необходимо и достаточное, чтобы $X$ имела вид
    $$\begin{pmatrix}
        a & b & 0 & 0 & 0\\
        0 & a & 0 & 0 & 0\\
        0 & 0 & c & d & e\\
        0 & 0 & 0 & c & 0\\
        0 & 0 & 0 & f & h\\
    \end{pmatrix}.$$
    получился какой-то кошмар, но делать нечего, воспользуемся алгоритмом 26, чтобы понять, при каких значениях переменных данная матрица диагонализируема.
    \begin{enumerate}
        \item $\chi_X(\lambda) = (a-\lambda)^2(c-\lambda)^2(h-\lambda)$;
        \item характеристический многочлен удалось разложить на линейные множители, так что можно продолжать;
        \item теперь надо сделать оценки на размеры нужных ФСР:
        \begin{itemize}
            \item для $h$: $X - hE = \begin{pmatrix}
                a-h & b & 0 & 0 & 0\\
                0 & a-h & 0 & 0 & 0\\
                0 & 0 & c-h & d & e\\
                0 & 0 & 0 & c-h & 0\\
                0 & 0 & 0 & f & 0\\
            \end{pmatrix}$, при этом мы хотим, чтобы ранг такой матрицы не превосходи 4 (иначе $X$ не диагонализируема), вполне очевидно, что это условие выполнено всегда (или мы можем сократить 4 и 5 строчку, или она из них уже нулевая).
            \item для $c$: $X - cE = \begin{pmatrix}
                a-c & b & 0 & 0 & 0\\
                0 & a-c & 0 & 0 & 0\\
                0 & 0 & 0 & d & e\\
                0 & 0 & 0 & 0 & 0\\
                0 & 0 & 0 & f & h-c\\
            \end{pmatrix}$, при этом ранг не должен превосходить 3. Заметим, что для этого достаточно, чтобы $b = d = e = f = 0$.
            \item для $a$: $X - aE = \begin{pmatrix}
                0 & b & 0 & 0 & 0\\
                0 & 0 & 0 & 0 & 0\\
                0 & 0 & c-a & d & e\\
                0 & 0 & 0 & c-a & 0\\
                0 & 0 & 0 & f & h-a\\
            \end{pmatrix}$, аналогично, для диагонализируемости необходимо, чтобы ранг не превосходил 3, а для этого достаточно, чтобы $b = d = e = f = 0$.
        \end{itemize}
    \end{enumerate}
    Таким образом, мы получили, что искомое пространство содержит в себе пространство матриц вида $\begin{pmatrix}
        a & 0 & 0 & 0 & 0\\
        0 & a & 0 & 0 & 0\\
        0 & 0 & b & 0 & 0\\
        0 & 0 & 0 & b & 0\\
        0 & 0 & 0 & 0 & c
    \end{pmatrix}$\\
    Размерность этого подпространства, очевидно, равна 3, а значит и размерность всего искомого пространства при $t = 0$ хотя бы 3.
    \item Настало время случая $t \ne 0$. Введём аналогичные обозначения, и получим:
    \begin{gather*}
        A\hatch X - X A\hatch = 0 = \\
        \begin{pmatrix}
            x_{21} & x_{22}-x_{11} & x_{23}-5x_{13} & -x_{13}-5x_{14}+x_{24} & -x_{14}-5x_{15}+x_{25}\\
            0 & -x_{21} & -5x_{23} & -x_{23}-5x_{24} & -x_{24}-5x_{25}\\
            5x_{31}+x_{41} & -x_{31}+5x_{32}+x_{42} & x_{43} & x_{44}-x_{33} & x_{45}-x_{34}\\
            5x_{41}+x_{51} & -x_{41}+5x_{42}+x_{52} & x_{53} & x_{54}-x_{43} & x_{55}-x_{44}\\
            5x_{51} & 5x_{52}-x_{51} & 0 & -x_{53} & -x_{54}
        \end{pmatrix};
        \intertext{вновь займёмся зануление одиноких переменных:}
        \begin{pmatrix}
            0 & x_{22}-x_{11} & 0-5x_{13} & -x_{13}-5x_{14}+x_{24} & -x_{14}-5x_{15}+x_{25}\\
            0 & 0 & 0 & 0-5x_{24} & -x_{24}-5x_{25}\\
            5x_{31}+x_{41} & -x_{31}+5x_{32}+x_{42} & 0 & x_{44}-x_{33} & x_{45}-x_{34}\\
            5x_{41}+0 & -x_{41}+5x_{42}+x_{52} & 0 & 0-0 & x_{55}-x_{44}\\
            0 & 5x_{52}-0 & 0 & 0 & 0
        \end{pmatrix};
        \intertext{возникли новые одинокие переменные, занулим их:}
        \begin{pmatrix}
            0 & x_{22}-x_{11} & 5x_{13} & -x_{13}-5x_{14}+0 & -x_{14}-5x_{15}+x_{25}\\
            0 & 0 & 0 & 0 & 0-5x_{25}\\
            5x_{31}+0 & -x_{31}+5x_{32}+x_{42} & 0 & x_{44}-x_{33} & x_{45}-x_{34}\\
            0 & 0+5x_{42}+0 & 0 & 0 & x_{55}-x_{44}\\
            0 & 0 & 0 & 0 & 0
        \end{pmatrix};\\
        \begin{pmatrix}
            0 & x_{22}-x_{11} & 0 & 0-5x_{14} & -x_{14}-5x_{15}+0\\
            0 & 0 & 0 & 0 & 0\\
            0 & 0+5x_{32}+0 & 0 & x_{44}-x_{33} & x_{45}-x_{34}\\
            0 & 0 & 0 & 0 & x_{55}-x_{44}\\
            0 & 0 & 0 & 0 & 0
        \end{pmatrix};\\
        \begin{pmatrix}
            0 & x_{22}-x_{11} & 0 & 0 & 0\\
            0 & 0 & 0 & 0 & 0\\
            0 & 0 & 0 & x_{44}-x_{33} & x_{45}-x_{34}\\
            0 & 0 & 0 & 0 & x_{55}-x_{44}\\
            0 & 0 & 0 & 0 & 0
        \end{pmatrix};
    \end{gather*}
    В этом случае некоторые переменные также сократились при подсчёте коммутатора, а именно $x_{12}, x_{35}$, а это значит, что для того, чтобы $X$ коммутировла с $A_t$ при $t \ne 0$ необходимо и достаточно, чтобы $X$ имела вид
    $$\begin{pmatrix}
        a & d & 0 & 0 & 0\\
        0 & a & 0 & 0 & 0\\
        0 & 0 & b & c & e\\
        0 & 0 & 0 & b & c\\
        0 & 0 & 0 & 0 & b
    \end{pmatrix}.$$
    Тут нам снова пригодится алгоритм 26.
    \begin{enumerate}
        \item $\chi_X(\lambda) = (a-\lambda)^2(b-\lambda)^3$;
        \item к сожалению, хар. многочлен разложился на линейные множители, так что придётся продолжать;
        \item сделаем оценки на размеры нужных ФСР:
        \begin{enumerate}
            \item сразу заметим, что если $a = b$, то необходимо, чтобы все остальные переменные равнялись 0 (так как при $a = b = d$ хар. многочлен будет иметь вид $(d-\lambda)^5$, а это значит, что ранг $X - dE$ должен равняться 0, из чего следует, что все остальные переменные должны быть равны 0). Далее считаем, что $a \ne b$;
            \item для $a$: $X - aE = \begin{pmatrix}
                0 & d & 0 & 0 & 0\\
                0 & 0 & 0 & 0 & 0\\
                0 & 0 & b-a & c & e\\
                0 & 0 & 0 & b-a & c\\
                0 & 0 & 0 & 0 & b-a
            \end{pmatrix}$, при этом нам нужно, чтобы ранг не превосходил 3. Это возможно только при $d = 0$.
            \item для $b$: $X - bE = \begin{pmatrix}
                a-b & d & 0 & 0 & 0\\
                0 & a-b & 0 & 0 & 0\\
                0 & 0 & 0 & c & e\\
                0 & 0 & 0 & 0 & c\\
                0 & 0 & 0 & 0 & 0
            \end{pmatrix}$, при этом ранг не должен превосходить 2, что возможно только при $c = e = 0$.
        \end{enumerate}
    \end{enumerate}
    Так мы получили необходимое и достаточное условие на то, чтобы $X$ была диагонализируема ($d = c = e = 0$), осталось оценить размерность пространства таких матриц. Вполне очевидно, что она равна 2.
\end{itemize}
В итоге мы получили, что размерность искомого пространства превосходит 2 только при $t = 0$, а значит максимум достигается только при этом значении.
\\


\i Начнём с того, что найдём базис левого ортогонального дополнения к $<$$v$$>$. Согласно алгоритму 36 это можно сделать найдя ФСР системы $v^t\beta^t x = 0$, сделаем это:
\begin{gather*}
    \begin{pmatrix}
        5 & 3 & -1 & -2
    \end{pmatrix} \begin{pmatrix}
        0 & 0 & 5 & -3\\
        0 & 0 & -3 & 2\\
        5 & -3 & 5 & 6\\
        -3 & 2 & 6 & -8
    \end{pmatrix} x = 0;\\
    \begin{pmatrix}
        1 & -1 & -1 & 1
    \end{pmatrix} x = 0.
\end{gather*}
Тогда ФСР (которая будет являться базисом левого ортогонального дополнения) будет иметь вид
\begin{gather*}
    f_1 = \begin{pmatrix}
        -1\\0\\0\\1
    \end{pmatrix},
    f_2 = \begin{pmatrix}
        1\\0\\1\\0
    \end{pmatrix},
    f_3 = \begin{pmatrix}
        1\\1\\0\\0
    \end{pmatrix}.
\end{gather*}
Давайте теперь рассмотрим базис равный $\brackets{v, f_1, f_2, f_3}$ (в том, линейной независимости этого множества очень легко убедиться), тогда матрицей перехода к этому базису будет $A = \brackets{v | f_1 | f_2 | f_3}$. Найдём вид $B\hatch$ матрицы $B$ в этом базисе:
\begin{gather*}
    B\hatch = A^tBA = \begin{pmatrix}
        6 & 4 & -2 & -3\\
        -21 & -14 & 4 & 8\\
        41 & 18 & 5 & -13\\
        1 & -1 & 2 & 0
    \end{pmatrix}
\end{gather*}
Теперь симметричным методом Гаусса приведём $B\hatch$ к диагональному виду:
\begin{gather*}
    \coolMatrix{6 & 4 & -2 & -3\\
            -21 & -14 & 4 & 8\\
            41 & 18 & 5 & -13\\
            1 & -1 & 2 & 0}
            {1 & 0 & 0 & 0\\
            0 & 1 & 0 & 0\\
            0 & 0 & 1 & 0\\
            0 & 0 & 0 & 1} \sim 
    \coolMatrix{
            1 & -1 & 2 & 0\\
            0 & 10 & -14 & -3\\
            -21 & -14 & 4 & 8\\
            41 & 18 & 5 & -13}
            {0 & 0 & 0 & 1\\
            1 & 0 & 0 & -6\\
            0 & 1 & 0 & 0\\
            0 & 0 & 1 & 0} \sim \\ \sim
    \coolMatrix{
            1 & -1 & 2 & 0\\
            0 & 10 & -14 & -3\\
            -21 & -14 & 4 & 8\\
            -1 & -10 & 13 & 3}
            {0 & 0 & 0 & 1\\
            1 & 0 & 0 & -6\\
            0 & 1 & 0 & 0\\
            0 & 2 & 1 & 0} \sim 
    \coolMatrix{
            1 & -1 & 2 & 0\\
            0 & 10 & -14 & -3\\
            -21 & -14 & 4 & 8\\
            0 & -11 & 15 & 3}
            {0 & 0 & 0 & 1\\
            1 & 0 & 0 & -6\\
            0 & 1 & 0 & 0\\
            0 & 2 & 1 & 1} \sim \\ \sim
    \coolMatrix{
            1 & -1 & 2 & 0\\
            0 & 10 & -14 & -3\\
            0 & -35 & 46 & 8\\
            0 & -11 & 15 & 3}
            {0 & 0 & 0 & 1\\
            1 & 0 & 0 & -6\\
            0 & 1 & 0 & 21\\
            0 & 2 & 1 & 1} \sim
    \coolMatrix{
            1 & -1 & 2 & 0\\
            0 & 10 & -14 & -3\\
            0 & -5 & 4 & -1\\
            0 & -1 & 1 & 0}
            {0 & 0 & 0 & 1\\
            1 & 0 & 0 & -6\\
            3 & 1 & 0 & 3\\
            1 & 2 & 1 & -5} \sim \\ \sim
    \coolMatrix{
            1 & -1 & 2 & 0\\
            0 & 0 & -6 & -5\\
            0 & -5 & 4 & -1\\
            0 & -1 & 1 & 0}
            {0 & 0 & 0 & 1\\
            7 & 2 & 0 & 0\\
            3 & 1 & 0 & 3\\
            1 & 2 & 1 & -5} \sim
    \coolMatrix{
            1 & -1 & 2 & 0\\
            0 & -1 & 1 & 0\\
            0 & 0 & -1 & -1\\
            0 & 0 & -6 & -5}
            {0 & 0 & 0 & 1\\
            1 & 2 & 1 & -5\\
            -2 & -9 & -5 & 28\\
            7 & 2 & 0 & 0} \sim \\ \sim
    \coolMatrix{
            1 & -1 & 2 & 0\\
            0 & -1 & 1 & 0\\
            0 & 0 & -1 & -1\\
            0 & 0 & 0 & 1}
            {0 & 0 & 0 & 1\\
            1 & 2 & 1 & -5\\
            -2 & -9 & -5 & 28\\
            19 & 56 & 30 & -168} \sim
    \coolMatrix{
            1 & -1 & 2 & 0\\
            0 & -1 & 1 & 0\\
            0 & 0 & -1 & 0\\
            0 & 0 & 0 & 1}
            {0 & 0 & 0 & 1\\
            1 & 2 & 1 & -5\\
            17 & 47 & 25 & -140\\
            19 & 56 & 30 & -168} \sim \\ \sim
    \coolMatrix{
            1 & -1 & 2 & 0\\
            0 & -1 & 0 & 0\\
            0 & 0 & -1 & 0\\
            0 & 0 & 0 & 1}
            {0 & 0 & 0 & 1\\
            18 & 49 & 26 & -145\\
            17 & 47 & 25 & -140\\
            19 & 56 & 30 & -168} \sim
    \coolMatrix{
            1 & -1 & 0 & 0\\
            0 & -1 & 0 & 0\\
            0 & 0 & -1 & 0\\
            0 & 0 & 0 & 1}
            {34 & 94 & 50 & -279\\
            18 & 49 & 26 & -145\\
            17 & 47 & 25 & -140\\
            19 & 56 & 30 & -168} \sim \\ \sim
    \coolMatrix{
            1 & 0 & 0 & 0\\
            0 & -1 & 0 & 0\\
            0 & 0 & -1 & 0\\
            0 & 0 & 0 & 1}
            {16 & 45 & 24 & -134\\
            18 & 49 & 26 & -145\\
            17 & 47 & 25 & -140\\
            19 & 56 & 30 & -168}
\end{gather*}
Таким образом, диагональный вид будет равен левой части полученной матрицы, а склеенные базис-столбцы будут равны произведению $A$ и правой части полученной матрицы, то есть
$$\begin{pmatrix}
    98 & 279 & 149 & -833\\
    67 & 191 & 102 & -570\\
    1 & 2 & 1 & -6\\
    -14 & -41 & -22 & 123
\end{pmatrix}$$