\subsection{Седьмое ДЗ}


\i Для начала заметим, что из рефлективности $\leq$ следует, что $x \leq x => x \sim x$, то есть, отношение $\sim$ также рефлективно.\\
Заметим, что $x \sim y <=> x \leq y \And y \leq x <=> y \leq x \And x \leq y <=> y \sim x$. Таким образом отношение $\sim$ симметрично.\\
Осталось показать транзитивность $\sim$. $\letus a \sim b \And b \sim c <=> a \leq b,\  b \leq c \And c \leq b,\  b \leq a$. Из транзитивность $\leq$ верно, что $a \leq c \And c \leq a <=> a \sim c$. Что и требовалось доказть.

\i Давайте рассматривать частичные порядки на множествах челых чисел, а отношением порядка будет кратность одного числа на другое. Иными словами, будем говорить, что $a \preccurlyeq b <=> a | b$.
Покажем, что ответ может принимать все значения от 0 до 6.
\begin{itemize}
    \item \textbf{0.} Пусть наши числа это различные степени двойки, тогда очевидно, что полученный порядок явяется линейным.
    \item \textbf{1.} Пусть 2 из наших чисел "--- степени различные, ненулевые степени 6, а оставшиеся 2 "--- 2 и 3 соответственно, тогда существует только одна пара несравнимых элементов.
    \item \textbf{3.} Пусть наши числа это 2, 3, 5, 30, тогда пар несравнимых элементов ровно 3.
    \item \textbf{4.} Пусть наши числа это 2, 4, 3, 9, тогда пар несравнимых "--- 4.
    \item \textbf{5.} Пусть наши числа это 2, 4, 3, 5, тогда существует только 1 пара сравнимых, а значит оставшиеся 5 пар "--- пары несравнимых.
    \item \textbf{6.} Пусть наши числа это 2, 3, 5, 7, тогда они все попарно несравнимы.
\end{itemize}

\i Давайте докажем, что эти порядки изоморфны. Заметим, что $42 = 2 \times 3 \times 7$. Тогда каждому делителю этого числа сопроставим элемент $f(a): \{x: x|42\} \mapsto \{0, 1\}^3$, где каждое число равно степени вхождения 2, 3, 7 соответственно. Тогда верно, что $a | b <=> f(a) \subseteq f(b)$, получим, что $f$ "--- изоморфизм порядков. Рассмотрим функцию $g(a): \{x: x\subseteq \{1, 2, 3\}\} \mapsto \{0, 1\}^3$. Легко убедиться, что $a \subseteq b <=> g(a) \subseteq g(b)$, следовательно $g$ "--- также является изоморфизмом прядков. Таким образом, мы получили, что оба данных порядка изоморфны некоторому третьему, следовательно они изморфны между собой. Что и требовалось доказать.

\i Предположим, что они изоморфны, тогда сеществует некоторый изоморфизм порядков $f: \ZZ + \NN \mapsto \ZZ + \ZZ$. Рассмотрим $f(1\hatch)$.
\begin{itemize}
    \item $f(1\hatch) = a$ (без штриха). Тогда рассмотрим прообраз числа $0\hatch$. Очевидно, что это некоторое число $k\hatch$, иначе нарушается порядок. С другой стороны, на порядке первом порядке отрезок $[1\hatch, k\hatch]$ был конечным, а вот его образ, $[a, 0\hatch]$, очевидно, бесконечен. Тогда такой вариант невозможен.
    \item $f(1\hatch) = a\hatch$ (со штрихом). Тогда рассмотрим прообраз $(a-1)\hatch$. Очевидно, что это некоторое число без штриха (иначе нарушится порядок между $1\hatch$ и этим числом), обозначим его за $b$, а прообраз $0$ обозначим за $c$ (тоже без штриха, по понтным причинам), при этом $c < b$, так как $f$ сохраняет порядок. В исходном порядке, отрезок $[b, c]$ был конечным, а его образ "--- бесконечный, следовательно, такой вариант тоже невозможен.
\end{itemize}
Таким образом, искомого изоморфизма не существует.

\i Предположим, что данные порядки изоморфны, тогда чуществует некоторый изоморфизм порядков $f: \NN \times \ZZ \mapsto \ZZ \times \ZZ$. Рассмотрим образ $(1, 0)$, пусть он равен $(a, b)$. Элемент $(a-1, b-1) < (a, b) <=> f^{-1}((a-1, b-1)) < f^{-1}((a, b))$, тогда прообраз $(a-1, b-1)$ имеет вид $(1, x), \ x < 0$. Тогда отрезок $[(1, 0), (1, x)]$ конечен, а его образ "--- нет, следовательно искомого изморфизма не существует.

\i Докажем, что нанные порядки не изоморфны. Предположим обратное, тогда существует некоторый изоморфизм порядков $f: \QQ \mapsto \NN \times \QQ$.  Рассмотрим поорбазы точек $(1, 0)$ и $(2, 0)$, пусть это некоторые числа $a$ и $b$. Займёмся полвинным делением. На очередной итерации будем брать точку $m = \frac{a+b}{2}$, которая, очевидно, рациональна и перебозначать за отрезок $[a, b]$ отрезок $[a, m]$ или $[m, b]$. Давайте поясним, как мы будем выбирать.\\
Заметим, что образ точки $m$ может иметь вид $(1, x)$, где $x$ больше соответствующей координаты точки $a$ (в этом случае выберем отрезок $[m, b]$), или вид $(2, y)$, где $y$ меньше соответствуюей координаты точки $b$ (в этом случае выберем отрезок $[a, m]$).\\
Заметим, что наш алгоритм приводит к тому, что в пределе на первом порядке мы получим одно число, или пустое множество, а на втором "--- бесконечный отрезок (так как после каждой итерации алгоритма он остаётся таковым). Следовательно искомого изоморфизма не существует.

\i Предположим, что данное множество не является фундированным, тогда верно, что в нём найдется некоторое подмножество $A$, в котором существует бесконечная убывающая последоавтельность $S$. Рассмотрим эту последоваельность $S$. Легко заметить, что множество первых элементов всех последовательностей из $s$ ограничено снизу (все они больше 0), а значит, найдётся наименьший элемент, а так как $S$, по нашему предположению бесконечная последовательность, то начиная с некоторого момента, все последовательности в $S$ начинаются с одного элемента. Аналогичное рассуждение верно и для всех последующих элементов в последовательностях из $S$. Тогда, для каждого индекса верно, что начиная с некоторого момента, все последоваетсльности в $S$ совпадают с ним. Тогда для каждого индекса верно, что элемент с его номером после того, Как он стабилизируется верно, что он либо равен, либо меньше пред идущего, следовательно $S$ не может быть бесконечной, так как тогда, начиная с какого-то момента верно, что все элементы в $S$, начиная с какого-то индекса, тожедественно равны некоторому числу, а значит совпадают. Значит, $S$, обрезанная до какого-то элемента, конечна, а значит мы пришли к противоречию, и данное множество фундированное. Что и требовалось доказть.