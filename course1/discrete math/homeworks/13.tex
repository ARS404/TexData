\subsection{Тринадцатое ДЗ}


\i Пусть событие $A$ это то, что человек оказался голубоглазым, а $B$ "--- то, что он блондин. Тогда, как известно $P(B|A) = \frac{P(A|B)\cdot P(B)}{P(A)} => \frac{P(B|A)}{P(A|B)} = \frac{P(B)}{P(A)} = 0{,}5$.

\i Пусть событие $A$ это то, что пациент болен, а $B$ "--- то, что его тест оказался положительным. Заметим, что так как события независимы, то $P(A|B) = P(A)$ и $P(\overline{A}|B) = P(\overline{A})$. Тогда $\frac{P(A|B)}{P(\overline{A}|B)} = \frac{P(A)}{P(\overline{A})} = \frac{10^{-5}}{1-10^{-5}} \approx 10^{-5}$.

\i Путь событие $A$ это то, что 1 оказалась на своём месте, а $B$ "--- то, что на всоём месте оказалась двойка. Заметим, что $P(A) = \frac{1}{n}$, так как мы можем поставить 1 на $n$ мест и для каждого из выборов будет ровно $(n-1)!$ вариантов расстановок. Теперь скажем, что $P(A|B) = \frac{1}{n-1}$ так как при условии, что 2 стоит на своём месте выборов положения 1 ровно $n-1$ и все они равновероятны. Таким образом, $P(A) \ne P(A|B)$, следовательно события не независимы.

\i Всего вариантов чисел, кратных 2 ровно 50. Заметим, что из них на 3 делятся только те, которые делятся на 6, и их всего $\lfloor\frac{100}{6}\rfloor = 16$. Таким образом, вероятность того, что случайное число от 1 до 100 делится на 3, при условии, что оно делится на 2, равна $\frac{16}{50} = 0{,}32$.

\i Рассмотрим все возможные варианты:
\begin{enumerate}
    \item Первый принял верное решение ($\frac{1}{p}$)
    \begin{enumerate}
        \item Второй принял верное решение ($\frac{1}{p}$)
        \begin{enumerate}
            \item Третий принял верное решение ($\frac{1}{2}$) "--- Этот вариант нас устраивает. Вероятность равна $\frac{1}{2p^2}$.
            \item Третий принял неверное решение ($\frac{1}{2}$) "--- Этот вариант нас устраивает. Вероятность равна $\frac{1}{2p^2}$.
        \end{enumerate}
        \item Второй принял неверное решение ($\frac{p - 1}{p}$)
        \begin{enumerate}
            \item Третий принял верное решение ($\frac{1}{2}$) "--- Этот вариант нас устраивает. Вероятность равна $\frac{p-1}{2p^2}$.
            \item Третий принял неверное решение ($\frac{1}{2}$) "--- Этот вариант нас не устраивает.
        \end{enumerate}
    \end{enumerate}
    \item Вервый принял неверное решение ($\frac{p - 1}{p}$)
    \begin{enumerate}
        \item Второй принял верное решение ($\frac{1}{p}$)
        \begin{enumerate}
            \item Третий принял верное решение ($\frac{1}{2}$) "--- этот вариант нас устраивает. Вероятность равна $\frac{p-1}{2o^2}$.
            \item Третий принял неверное решение ($\frac{1}{2}$) "--- этот вариант нас не устраивает.
        \end{enumerate}
        \item Второй принял неверное решение ($\frac{p - 1}{p}$)
        \begin{enumerate}
            \item Третий принял верное решение ($\frac{1}{2}$)  "--- этот вариант нас не устраивает.
            \item Третий принял неверное решение ($\frac{1}{2}$)  "--- этот вариант нас не устраивает.
        \end{enumerate}
    \end{enumerate}
\end{enumerate}
Так как конечные события (выбор всех 3 жюри) независимы, то вероятоность правильного выбора составляет $\frac{1}{2p^2} + \frac{1}{2p^2} + \frac{p-1}{2p^2} + \frac{p-1}{2p^2} = \frac{1}{p}$.

\i Заметим, что вероятность выбора конкретного шара составляет $\frac{1}{2x}$, где $x$ "--- количество шаров в его коробке (вероятность именно такая потому, что с вероятностью $\frac{1}{2}$ король берёт нужную коробку и с вероятностью $\frac{1}{x}$ берёт из неё нужный шар). Также $\frac{1}{2x} \geq \frac{1}{38}$, так как в коробке по условию не может лежать больше 19 шаров. Тогда вероятность выборы чётного шара не меньше, чем $10 \cdot \frac{1}{38}$. Осталось заметить, что это достижимо при условии, что в одной коробке лежит один белый шар, а все осталбные лежат в другой коробке.

\i Для начала заметим, что нас интересуют только $n+2$ яйца, которые участвовали в первых $n+1$ раунде. Мы знаем, что все их процности различны, и все упорядочивания по прочности равновероятны. Заметим, что количество перестановок, где яйцо из первого раунда выигрывает следующие $n-1$ раунд равно $2(n! + (n+1)!)$ так как перестановок, где в первой паре есть самое сильное яйцо равно $2(n+1)!$, а таких, где в первой паре стоит второе по силе яйцо, а самое сильное стоит последним ровно $2n!$. При этом тех, которые нас устраивают всего $2(n+1)!$, так как в оставшихся $2n!$ выиграет последнее яйцо. Тогда вероятность того, что в $(n+1)$-ом раунде будет достигнута победа при уловии, что она достигнута во всех предыдущих, равна $\frac{(n+1)!}{(n+1)! + n!} = \frac{n+1}{n+2}$.