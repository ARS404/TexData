\subsection{Второе ДЗ}

\i Для начала заметим, что по условию чётные и нечётные числа между собой строго упрядоченны, а значит, стоится биекция между возможжными расстановками чисел в ряду и указанием на позиции, на которых стоят чётные числа. Посчитать количество вариантов второго совсем просто. По очевидным причинам это в точности $\binom{10}{5}$.

\i Для начала скажем, что всего вариантов раскладки колоды существует $36!$. Способов выбрать 2 карты одного цвета сцществует ровно $\binom{18}{2}$, а значит вариантов выбрать 2 пары разных цветов всего $\binom{18}{2}^2$. Тогда раскладок колоды, в которых на верху лежат 2 пары карт разных цветов существует $\binom{18}{2}^2 \cdot 4! \cdot 32!$. А значит, вероятность того, что раскладка удовлетворяет нашим запросам составляет
$$\frac{\binom{18}{2}^2 \cdot 4! \cdot 32!}{36!} = \frac{153}{385} = 0{,}3(974025).$$

\i Для начала подсчитаем, сколько всего существует 7--значных чисел, в которых ровно 2 чётные цифры, и которые начинаются с нечётной цифры. Легко убедиться, что их $\binom{6}{2} \cdot 5^7$, так как всего существует ровно $\binom{6}{2}$ способов выбрать позиции для чётных чисел, и на каждую позицию в числе можно выбрать ровно 5 цифр. При этом, способов выбрать 2 подряд идущие позиции для чётных чисел ровно 5 (так как у нас ровно 5 способов выбрать место для первой из них), и на каждый из них мы можем составить $5^7$ чисел, следовательно мы среди $\binom{6}{2} \cdot 5^7$ посчитали ровно $5^8$ чисел, которые нас не устраивают. Тогда 7--значных чисел, в записиси которых ровно 2 чётные цифры, и перед каждой из них стоит нечётная всего $\binom{6}{2} \cdot 5^7 - 5^8 = 2 \cdot 5^8.$

\i Данная задача, по очевидным причинам эквавалентна задаче о разбиении числа в сумму фиксированного числа слагаемых (то есть задаче о шарах и перегородках, которая разбиралась на лекции). Тогда ответ будет равен $\binom{8+7}{7} = \binom{15}{7} = 6435$.

\i Для начала скажем, что общее число разбиений детей на 3 хоровода ровно в точности $3^9$. В то же время, число разбиений детей на группы составляет 
$$\binom{9}{3} \times \binom{6}{3} \times \binom{3}{3} = 84 \cdot 20 \cdot 1 = 1680,$$
так как мы сперва выбираем 3 из 9 детей на первый хоровод, затем 3 из 6 "--- на второй, а остальных "--- на третий. Таким образом, вероятность того, что вокруг какой-то ёлки не получится хоровода составляет $\frac{3^9 - 1680}{3^9} = \frac{6001}{6561}.$

\i Давайте доказывать данную задачу по индукции по $n$.\\
\textbf{База.} $n = 0, n = 1$ "---  очевидно.\\
\textbf{Переход.} Предположим, что задача доказана для всех $n \leq k$, докажем её для $n = k+1$.
\begin{gather*}
    \sum\limits_{i=0}^{(k+1)/2}\binom{(k+1)-i}{i} = 
    \intertext{нам известно тождество вида:}
    \binom{n}{k} = \binom{n-1}{k} + \binom{n-1}{k-1};
    \intertext{давайте применим его к каждому слагаемому:}
    = \sum\limits_{i=0}^{(k+1)/2}\brackets{\binom{k-i}{i} + \binom{k-i}{i-1}} = \\
    = \sum\limits_{i=0}^{(k+1)/2}\binom{k-i}{i} + \sum\limits_{i=0}^{(k+1)/2}\binom{k-i}{i-1} = 
    \intertext{Давайте считать, биномиальные коэфиценты, один (или оба) из агрументов которых меньше нуля, или такие, что в них $n < k$, равными нулю (при этом, тождесво которым мы пользуемся сохранить свою вернсть. Тогда получим:}
    = \sum\limits_{i=0}^{(k)/2}\binom{k-i}{i} + \sum\limits_{i=0}^{(k-1)/2}\binom{k-i}{i} = \\
    = F_{k} + F_{k-1} = F_{k+1}.
\end{gather*}
Что и требовалось доказать.

\i Сперва скажем, что мы будем называть строку треуголькика Паскаля прикольной, если все, кроме крайних, числа в ней чётные. Докажем, что если строка с номером $n$ прикольная, то и строка с номером $2n$ "--- тоже. Утверждение о том, что строка с номером $n$ прикольная равносильно тому, что $(x+1)^n = x^n + 1 + 2 f(x)$, где $f(x)$ "--- многочлен с целыми коэфицентами (так можно скажать потому, чтовсе коэфиценты $f(x)$ будут биномиальными каэфицентами, из которых состоит $n$--тая строка треугольника Паскаля, в которой по нашему предполажению все числа чётные). Тогда давайте возведём это равенство в квадрат и убедимся, что строка с номером $n+1$ тоже прикольная:
$$(x+1)^{2n} = x^{2n} + 1 + 4f^2(x) + 2x^n + 4x^nf(x) + 4f(x) = x^{2n} + 1 + 2f_1(x).$$
Таким образом, строка с номером $2n$ тоже прикольная. Давайте теперь докажем, что прикольными могут быть срока только с номерами вида $2^k$. Будем делать это по идукции. База очевидна. Предположим, что наше утверждение верно для всех строк с номерами не больше, чем $2^k$, докажем его для всех строк с номерами не больше, чем $2^{k+1}$. заметим, что в строке с номером $2^k$ в центре есть <<отрезок>> из $2^k - 1$ чётного числа, тогда очевидно, что в каждой следующей строке в центре стоит <<отрезок>> длины на 1 меньше. А значит, строка со всем нечётными возникнет не раньше, чем через $2^k-1$ строку и будет иметь номер не меньше, чем $2^{k+1}-1$. С другой стороны, мы знаем, что строка с номером $2^{k+1}$ точно прикольная, а значит в строке с номером $2^{k+1}$ все числа нечётные. Таким образм мы доказали, что все прикольные строки имеют номера вида $2^k$, и все такие строки прикольные. Это в свою очередь равносильно тому, что все строки с номерами вида $2^k - 1$ состоят целиком из нечётных чисел, и никакие дргуие этому условию не удовлетворяют.

\i Давайте сперва посмотрим, сколько существует способов расставить на доске $100 \times 100$ ровно $20$ различных фигур, это $\frac{(100^2)!}{(100^2-20)!} = (100^2-19) \cdot (100^2-18) \cdot \ldots \cdot 100^2$. При этом мы посчитали в том числе и <<плохие>> расстановки, в которых есть побитые фигуры. Очень грубо оценим количество таких расстановок. Для этого скажем, что в каждой из них найдётся пара фигур такая, что первая бьёт вторую. Их не более чем $(100^2 \cdot 20) \times (20 \cdot 19)$, где первая скобка "--- количество способов выбрать и поставить первую фигуру, а вторая "--- вторую. При этом, мы посчитали каждую <<плохую>> расстановку несколько раз (но намдостаточно того, что мы посчитали каждую). Осталось убедиться, что общее количество расстановок строго больше, чем количество <<плохих>>, что очевидно. Таким образом, все расстановки не могут быть <<плохими>>, а значит найдётся такая, в которой расставлены 20 фигур так, что они не бьют друг друга.