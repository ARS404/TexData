\subsection{Шестое ДЗ}


\i Предположим, что множество таких последовательнстей счётно, тогда каждое его подмножество или конечно, или счётно. Зная это, рассмотрим последовательности, где пара элементов с номерами $2k, 2k+1$ всегда имеет вида $01$ или $10$. Легко убедиться, что множество таких последовательностей является подмножеством описанного в условии множества двоичных последовательностей. Очевидно, что оно бесконечно, тогда, из нашего предположения оно должно быть счётно, а значит, мы сумеем пронумеровать все его элементы. Докажем, что так мы пронумеровали не все последовательности нашего множества. Для этого каждой исходной последовательности сопоставим новую, в которой каждый элемент будет виды пары соседних элементов в исходной последовательности. Таким образом, мы получим новую воичную последовательность, и по каждой возможной такой последовательности мы сымеем восстановить исходную (у нас получилась биекция), а множество таких двоичных последовательностей, как известно, не счётно. Получилось противоречие, что и требовалось доказать.

\i Для начала сопоставим каждому отрезку пару вещественных чисел, равных его началу и концу (будем считать все отрезки ориентированными). Тогда мы хотим доказать, что $\RR \times \RR \sim \RR$, так как мощность множества вещественных числе равна континуум. С другой стороны, мы знаем, что $\RR$ равномощно множеству бесконечных двоичных последовательностей. Теперь давайте построим биекцию из $f: \RR \times \RR \mapsto \RR$ по следующему правилу:
\begin{itemize}
    \item каждому числу наша функция сопоставляет какую-то двоичную последовательность бесконечной длины (это сопоставление биективно),
    \item для каждой пары последовательностей она сторит третью, как результат их <<сливания>> по следующему правилу:\\
    на позицию с номером $2k+1$ втсаёт символ с номером $k+1$ из первой последовательности, а на $2k$ "--- с номером $k$ из второй.
    \item после $f$ сопоставляет полученной последовательности вещественной число (по той же биекции, что и до этого).
\end{itemize}
Так же легко убедиться, что полученная функция является биекцией, а значит мы доказали, что $\RR^2 \sim \RR$. Более того, из этого утверждения индукцией по степени можно доказать, что $\RR^n \sim \RR$.\\
Таким образом, мы доказали, что множество отрезов на прямй имеет мощность $\mathfrak{C}$.

\i Для начала обозначим множество последовательностей из условия за $\Sigma$. Как известно, $\{0, 1, 2\}^{\aleph_0} \sim \RR \ $, а $\ \Sigma \subset \{0, 1, 2\}^{\aleph_0} => \abs{\Sigma} \leq \abs{\RR}$\\
Теперь рассмотрим $P \subset \Sigma = \{S\}$, где верно, что $s_{2k} = 1, s_{2k+1} \in \{0, 2\}$. Рассмотрев каждый элемент последовательности по модулю 2, мы получим что $\abs{P} \geq \abs{\{0, 1\}^{\aleph_0}} => \abs{\Sigma} \geq \abs{\{0, 1\}^{\aleph_0}} = \abs{\RR} => \Sigma \sim \RR$.

\i Обозначим множество описанных прогрессий за $\Phi$. Покажем, что $\abs{\Phi} = \aleph_0$.\\
Рассмотрим все прогрессии, начинающиеся с числа $n$, обозначим их множество за $\Phi_n$. Построим
из $\Phi_n$ инъекцию в $\NN_{+\{-1, 0\}}^n$ по следующему правилу:\\
для прогрессии $A \in \Phi_n$ в последовательности $S(A)$ на $i$-том месте будем писать, сколько раз в $A$ встречается число $i$, если же, начиная с какого-то мометна, все числа в последовательности равны, то в соответствующем месте $S(A)$ напишем $-1$. Таким образом, $\abs{\Phi_n} \leq \abs{\NN_{+\{-1, 0\}}^n}$.\\
Пусть $\NN_{+\{-1, 0\}}^\ast = \{\NN_{+\{-1, 0\}}^n| n\in\NN\}$. Покажем, что $\abs{\NN_{+\{-1, 0\}}^\ast} = \aleph_0$.  Для начала земетим, что для элементов $\NN_{+\{-1, 0\}}^n$ можно ввести лексекографический порядок (что очевидно) и пронумеровать их, а значит $\abs{\NN_{+\{-1, 0\}}^n} = \aleph_0$. Мы уже много раз доказывали, что объединение счётного числа счётных множеств счётно, а значит и мощность $\NN_{+\{-1, 0\}}^\ast$ равно $\aleph_0$, а так как $\abs{\Phi} \leq \abs{\NN_{+\{-1, 0\}}^\ast}$, и $\Phi$, очевидно, бесконечно, то оно счётно. Что и требовалось доказать.

\i Для начала выберем некоторую биекцию $F_0$, и убдем каждому рациональному чслу сопоставлять его пару в этой биекции. Таким образом нам необходимо почситать количество биекций из $\NN$ в $\NN$, то есть количество перестановок натуральных чисел. Обозначим множество всех перестановок на $\NN$ за $\Theta$. Покажем, что $\abs{\Theta} > \aleph_0$. Предположим обратное, тогда все перестановки можно пронумеровать, приведём пример перестаноки, которая не получила свой номер. Выберем ей значения так, чтобы в $i$-том символе она отличалась от $i$-той перестановки (очевидно, что так можно сделать), таким образом, $\abs{\Theta} > \aleph_0$. 

\i Да можно, для доказательства просто предъявим пример размещения континуум единиц на плоскости.\\
Для начала скажем, что все единицы будут повёрнуты так, что их <<короткая>> строна будет направлена вдоль оси $OX$ в положительную сторону (предварительно мы ввели декартову систему координат), то есть единицы будут выглядеть примерно так: $\angle$. Далее рассмотрим прямую, задаваемую уравнением $y = x$. Легко убедиться, что, если разместить единицы так, чтобы их вершины лежали на этой прямой, то они не будут пересекаться (что очевидно). Тогда давайте расположим единицы так, что в каждой точке прямой $y = x$ будет распологаться рвершина какой-то единицы. Тогда мощность множество единиц совпадает с мощностью $\RR$, то есть равна $\mathfrak{C}$, и при этом единицы не пересекаются. Что и требовалось доказать.

\i Для начала введём на плоскости декартову систему координат. Обозначим за $S$ множество всех точек, обе координаты которых рацианальны. Как мы знаем, $\abs{S} = \aleph_0$. Для любого креста, отрезки которого не паралленльны осям координат верно, что на нём найдется уникальная точка из $S$ (что очевидно). Осталось показать, что множество крестов, с отрезками параллельными осям координат также счётно, и тогда мы докажем, что всё множество крестов счётно. Для этого оставим на плоскости только эти кресты и введём новые ося координат так, чтобы они не были параллельны сторонам оставшихся крестов. Повторив аналогичные рассуждения м ыполучим, что их множество счётно, а значит любое множество крестов, которое можно нарисовать или конечно, или счётно. Что и требовалось доказать.

\i $\letus B\hatch = B \backslash A$, тогда $A \cup B = A \cup B\hatch$. Докажем, если $A \cup B\hatch$ континуально, то и одно из исходных множеств также континуально (тогда мы докажем исходную заачу).\\
Так как $\abs{A \cup B\hatch} = \mathfrak{C}$, то $A \cup B\hatch \sim [0, 1]^2$. С другой стороны, рассмотрим Квадрат $1 \times 1$, как объенидение $\mathfrak{C}$ множеств $[0, 1]$. Предположим, что хотя бы одно из множеств $A, B\hatch$ п омодулю меньше, чем $\mathfrak{C}$. Тогда среди рассматриваемых множеств $[0, 1]$ найдётся хотя бы одно, в котором нет элементов из этого множества ($A$ или $B\hatch$), Следовательно все его элементы лежат в оставшемся, которое, в таком случае континуально. Что и требовалось доказать.
