\subsection{Четырнадцатое ДЗ}


\i Предположим обратное, тогда матожидание выигрыша составляет хотя бы $5000 \times Pr[\text{выигрыш} \geq 5000] \geq 5000 \times 0{,}01 = 50$. Заметим, что это больше, чем сумма, которая идёт на выигрышь с одного билета, а значит наше предположение неверно, и вероятность выигрыша не менее $5000$ меньше одного процена.

\i Давайте докажем, что $E[x] \geq 2{,}5$. В самом деле, $E[x] \geq Pr[x \geq 5] \times 5 = 2{,}5$. Теперь покажем, что любое значение $E[x] \geq 2{,}5$ достижимо. Пусть $Pr[x = 0] = \frac{1}{2}$ и $Pr[x = 5+c] = \frac{1}{2}$, вероятность всех остальных значений, соответствено, равна нулю. Заметим, что такие вероятности удовлетворяют условию задачи, а $E[x]$ равно $2{,}5 + \frac{c}{2}$, где $c \geq 0$, а значит, $E[x]$ может принимать все возможные значения не меньше $2{,}5$.

\i Для начала разобьём все возможные наборы на пары. Для этого каждому набору $(a_1, \ldots a_{10})$ сопоставим набор $(29-a_1, \ldots 29-a_{10})$. Заметим, что это однозначное сопоставление. Пусть $c$ вероятность выбора конкретного набора (наборы выбираются равновероятно). Тогда $E[\sum a_i] = \sum\limits_{(a_1, \ldots, a_{10})}\frac{1}{2}\brackets{c(a_1 + \ldots + a_{10}) + c(29-a_1 + \ldots + 29-a_{10})} = \frac{c}{2}(29\times 10)\times \frac{1}{c} = 5\times29$.

\i Посчитаем матожидание того, что произвольная пара букв в нашем слове окажется <<хорошей>> (то есть $ab$). Заметим, что нас устраивает только 1 вариант из 4, а значит и матожидание такого события равно $\frac{1}{4}$. Осталось просуммировать его по всем парам, которых 19. Таким образом матожидание количества подслов вида $ab$ составляет $\frac{19}{4}$.

\i Предположим обратое, то есть, $Pr[x \geq 6] \geq \frac{1}{10}$. Тогда $E[2^x] \geq 2^6 \times Pr[x \geq 6] \geq \frac{64}{10} > 5$ (так как $2^x > 0$), противоречие, а значит наше предположение неверно $=> Pr[x \geq 6] < \frac{1}{10}$. 

\i Для начала обозначим данные в условии множества за $A, B$ соответственно. Заметим, что вариантов, когда $f: A \mapsto B$ инъективна ровно $\frac{\abs{B}!}{(\abs{B}-\abs{A})!}$ так как для первого элемента $A$ мы можем выбрать значение $\abs{B}$ способами, для второго "--- $\abs{B}-1$, \ldots, для последнего "--- $\abs{B} - \abs{A} + 1$ способом. При этом всего функций $f: A \mapsto B$ существует ровно $\abs{B}^{\abs{A}}$. Подставим $n$, тогда инъективных функций всего $\frac{(100n)!}{(99n)!}$, а произвольных "--- $(100n)^{n}$. То есть, нас просят доказать, что $\frac{\frac{(100n)!}{(99n)!}}{(100n)^{n}} \mapsto 0$ при $n \mapsto \infty$. По формула Стирлинга:
\begin{gather*}
    \limit{n}{\infty} \frac{\frac{(100n)!}{(99n)!}}{(100n)^{n}} = \limit{n}{\infty} \frac{\frac{\sqrt{2\pi 100n}\brackets{\frac{100n}{e}}^{100n}}{\sqrt{2\pi 99n}\brackets{\frac{99n}{e}}^{99n}}}{(100n)^n} = \limit{n}{\infty} \frac{\sqrt{2\pi 100n}\brackets{\frac{100n}{e}}^{100n}}{\sqrt{2\pi 99n}\brackets{\frac{99n}{e}}^{99n}(100n)^n} = \\
    = \limit{n}{\infty} \frac{\sqrt{100}\brackets{\frac{100}{e}}^{100n}}{\sqrt{99}\brackets{\frac{99}{e}}^{99n}100^n} = \limit{n}{\infty} \frac{10 \cdot 100^{99n}}{\sqrt{99}\cdot e^n \cdot 99^{99n}} = \frac{10}{\sqrt{99}} \limit{n}{\infty} \brackets{\frac{100^{99}}{e\cdot99^{99}}}^n.
\end{gather*}
Заметим, что $\frac{100^{99}}{e\cdot99^{99}} < 1 => \limit{n}{\infty} \frac{\frac{(100n)!}{(99n)!}}{(100n)^{n}} = 0.$ Что и требовалось доказать.