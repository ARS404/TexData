\subsubsection{Семнадцатое ДЗ}

\i $\brackets{\frac{F}{G}}\hatch = \brackets{FG^{-1}}\hatch = F\hatch/G + F(G^{-1})\hatch = F\hatch/G -\frac{G\hatch}{G^2}F = \frac{F\hatch G - FG\hatch}{G^2}.$

\i Будем доказывать, что для произвольной производящей функции $F$, у которой коэффициент при $x^1$ отличен от нуля, из рассматриваемого класса существует искомая. Подставим $G$ в $F$, получим:
$$R(x) = F(G(x)) = f_1G(x) + f_2G^2(x) + \ldots = \sum_{k=1}^{\infty} f_kG^k(x).$$
При этом мы хотим, чтобы $r_0 = 0$ (что, очевидно, верно) и $f_1 g_1 = 1$ (то есть, $g_1 = \frac{1}{f_1}$, отсюда очевидно становится, что все варианты, когда $f_1 = 0$ не подходят). Также положим, что $g_0 = 0$. Запишем $r_2$:
$$r_2 = f_1 g_2 + f_2 g_1 = 0.$$
Тогда $g_2 = \frac{-f_2 g_1}{f_1}$. Далее по индукции докажем, что искомаю производящая функция $G(x)$ существует, для этого поочерёдно посчитаем её коэффициенты (первые 3 уже восстановили, это наша база). Предположим, что мы восстановили первые $k$, тогда $(k+1)$-ый можно представить в следующем виде:
$$r_{k+1} = f_1 g_{k+1} + f_2\sum_{t_1 + t_2 = k+1}g_{t_1}g_{t_2} + \ldots + f_k\sum_{t_1 + t_2 + \cdot + t_k = k+1} \prod_{i = 1}^{k} g_{t_i} + f_{k+1}\sum_{t_1 + t_2 + \cdot + t_{k+1} = k+1} \prod_{i = 1}^{k+1} g_{t_i} = 0.$$
Таким образом, можно однозначно восстановить значение $g_{k+1}$ так как все слагаемые, кроме первого нам уже известны (так как по индукции мы восстановили все значения $g_i, i \leq k$, и $g_{k+1}$ присутствует только в одном слагаемом), а значит, из него мы можем выразить $g_{k+1}$:
$$g_{k+1} = -\frac{f_2\displaystyle\sum_{t_1 + t_2 = k+1}g_{t_1}g_{t_2} + \ldots + f_k\sum_{t_1 + t_2 + \cdot + t_k = k+1} \prod_{i = 1}^{k} g_{t_i} + f_{k+1}\sum_{t_1 + t_2 + \cdot + t_{k+1} = k+1} \prod_{i = 1}^{k+1} g_{t_i}}{f_1}$$.
Таким образом, мы однозначно определяем все занчения $g_i$ то, чтобы $F(G(x)) = x$ тогда и только тогда, когда $f_1 \ne 0$ (при условии, что $f_0 = 0$).

\i Рассмотрим функцию $(1+x)^n$, её производящая функция это $1 + \binom{n}{1}x + \binom{n}{2}x^2 + \ldots + \binom{n}{n}x^n$, так как выполняется равенство:
$$(1+x)^n = 1 + \binom{n}{1}x + \binom{n}{2}x^2 + \ldots + \binom{n}{n}x^n.$$
От обоих частей этого равенства возьмём производную (то, что так можно делать доказывали на лекции), получим:
$$n(1+x)^{n-1} = 1\binom{n}{1} + 2\binom{n}{2}x + \ldots + n\binom{n}{n}x^{n-1}.$$
При $x = 1$ левая часть равенства превращается в искомую величину, а правая легко считается, получим:
$$1\binom{n}{1} + 2\binom{n}{2} + \ldots n\binom{n}{n} = n(1+1)^{n-1} = n\cdot 2^{n-1}.$$

\i Давайте найдём производящую функцию $F(x) = \displaystyle \sum_{k=0}^{\infty} (k+1)x^k$. Заметим, что $F(x)\cdot x = \displaystyle \sum_{k=1}^{\infty} k x^k$, следовательно:
$$F(x) \cdot (1-x) = \sum_{k=1}^{\infty} x^k.$$
Из материалов лекции мы знаем левую часть, а значит можем переписать равенство следующим образом:
$$F(x) = \frac{1}{(1-x)^2} = \sum_{k=0}^{\infty} (k+1)x^k.$$
Теперь рассмотрим последовательность частичных сумм вида 
$$G(x) = \sum_{k=0}^{\infty}\brackets{x^k\sum_{i=0}^{k}i(i+1)}.$$
По материалам лекции мы заем, что $G(x)$ можно представить в следующем виде:
$$G(x) = F(x)\cdot(1+x+x^2+\ldots) = \frac{1}{(1-x)^2} \cdot \frac{1}{1-x} = \frac{1}{(1-x)^3}.$$
Таким образом, $(n-1)$-ый коэффициент (который нас и интересует) мы можем посчитать как 
$$\frac{G^{(n)}(0)}{(n-1)!} = \brackets{\frac{1}{(1-x)^3}}^{(n)} \cdot \frac{1}{(n-1)!} = \frac{(n+1)!}{3(n-1)!} = \frac{(n-1)(n)(n+1)}{3}$$

\i Рассмотрим выражение вида $F_{100}(x) = \displaystyle\prod_{k=1}^{100} (x+\frac{1}{k})$. Раскрыв в нём скобки, мы получим, что все его коэффиуиенты "--- числа, обратные к произведению элементов некоторого подмножества $\{1, 3, \ldots, 100\}$ (что очевидно), однако при этом мы также посчитали и пустое подмножество (коэффициент при $x^{100}$). Тогда искомая сумма равна $F_{100}(1)-1$. Рассмотрим $F_i(x)\displaystyle\prod_{k=1}^{i} (x+\frac{1}{k})$ по индукции докажем, что $F_i(1) = i+1$.\\
\textit{База.} $i=1, F_1(1) = 1 + \frac{1}{1} = 2$. База верна.\\
\textit{Переход.} Пусть равенство докажено для всех $i < n$, докажем его для $i = n$.
$$F_n(1) = F_{n-1}(1) \cdot (1 + \frac{1}{n}) = n \cdot \frac{n+1}{n} = n+1.$$
Таким образом переход доказан, а значит $F_{100}(1) - 1 = 100$, что равняется искомой сумме.

\i Рассмотрим производящую функцию $A(x) = \frac{1}{(1-x^2)(1-x^5)}$, тогда верно, что:
$$A(x) = \frac{1}{(1-x^2)} \cdot \frac{1}{(1-x^5)} = \brackets{\sum_{k=0}^{\infty}x^{2k}} \brackets{\sum_{k=0}^{\infty}x^{5k}}.$$
Тогда, раскрыв скобки, получим, что коэффициент $A(x)$ при $x^n$ равен числу способов разложить $n$ в сумму пятёрок и двоек (что очевидно). Тогда коэффициенты при $x^n$ $A(x)\cdot x, A(x)\cdot x^2, A(x)\cdot x^3, A(x)\cdot x^4$ равны числу способов представить $n$ как сумму некоторого числа двоек и пятёрок и одной, двух, трёх и четырёх единиц соответственно. А значит, коэффициент при $x^n$ в $F(x) = A(x) + A(x)\cdot x + A(x)\cdot x^2 +  A(x)\cdot x^3 +  A(x)\cdot x^4$ равен числу способов представить $n$ как сумму двоек и пятёрок и не более чем 4 единиц. Таким образом, искомая производящая функция равна $\displaystyle\frac{1 + x + x^2 + x^3 + x^4}{(1-x^2)(1-x^5)}$.