\subsection{Восьмое ДЗ}


\i Заметим, что любые цепь и антицепь пересекаются не более, чем по одному элементу, что очевидно. Тогда мы получаем оценку на $n-k+1$, осталось привести пример.\\
Дававйте рассматривать прядок на натуральныйх числах, где отношением порядка будет делимость. Пусть цепь представляет из себя $k$ различных степеней $2$, начиная с первой, а все остальные числа "--- различные нечётные простые. В таком случае, масимальная антицепь, очевино содержит ровно $n-k+1$ число (достаточно просто взять все простые числа).

\i Обозначим данное множество за $S$ и предположим обратное: все цепи и антицепи в $S$ конечны.\\
Тогда для каждого $A \subset S$ за $A_0$ обозначим множество наименьших жлементов в $A$. Так как $A_0$ является антицепью в $S$ (что очевидно), то мощность $A_0$ для любого $A$ конечна. Также легко заметить, что $\forall a \in A \exists a_0 \in A_0: a_0 \leq a$ (так как вякая цепь в $A$ является цепью в $S$, а значит конечна и имеет наименьший элемент в $A$, который гарантированно лежит в $A_0$).
$\forall s \in S$ обозначим множество элементов больших $s$ за $M(s)$. Заметим, что $M(s)_0$ конечно по вышедоказанному. Рассмотрим такое множество $X$, которое состоит из $s \in S: M(s)$ бесконечно. Тогда для любого $x$ из $X$ верно, что $M(x)_0$ конечно.\\
Заметим, что $M(s) = M(s)_0 \cup \bigcup\limits_{s_0 \in S_0} M(s_0)$. В силу конечности $M(s)_0$ мы получим, что для каждого элемента $h$ можновыбрать некоторый другой $f(h)$, такой что $h < f(h)$, что равносильно тому, что мы сумели найти бесконечную возростающую последовательность. Что и требовалось доказать.

\i Рассмотрим граф, где верины "--- прямые из условия, а рёбра имеют 2 цвета, при том белое проводим, если 2 прямые пересекаются не ниже $OX$, иначе проведём чёрное ребро. По условию, любые 2 прямые пересекаются, поэтому граф получится полным. Тогда применим для начего графа теорему Рамсея. Известно, что $r(3, 5) = 14$, следовательно в нашем графе найдётся $K_3$ на чёрном цвете, или $K_5$ на белом (возможно и то и то), а это ровно то, что мы хотим. Что и требовалось доказать.

\i Нет, для этого покажем, что есть пример, в котором $x$ и $f(x)$ не сравнимы при любом $x$.\\
Рассмотрим произвольное множество, в котором никакие 2 элемента не сравнимы, легко убедиться, что такое множество фундированное, пусть $f(x)$ "--- произвольное преобразование, которое на всех элементах, кроме $a$ и $b$ тожедественно, а эти 2 меняет местами. Очевидно, что такой пример удовлетворяет условию, и при этом не верно, что $f(x) \geq x$ для $x = a$.

\i Предположим обратное, тогда найдётся такое $x_0 \in A: f(x_0) = x_1 \in A, x_1 < x_0$, тогда из монотонности $f(x)$, получим, что $f(x_1) = x_2, f(x_0) > f(x_1) => x_1 > x_2$. Аналогичное верно для произвольного $i: f(x_i) = x_{i+1}, f(x_{i-1}) > f(x_i) => x_i > x_{i+1}$. В таком случае, мы получили бесконечную убывающую последовательности $\{x_i\}$, а значит исходное множество не может быть фундированным, простиворечие. 

\i Для начала преобразуем нашу систему отрезков так, что если отрезок $a$ вложен в отрезок $b$, то выкинем отрезок $b$. Покажем, что при таких операциях числа $n$ и $k$ не изменились.\\
Во-первых, для лбого <<протыкания>> полученной системы верно, что будет проткнута и исходная система, а для любое <<протыкание>> исходной, также удовлетворяет и полученной, поэтому число $k$ не изменилось. Во-вторых, если у нас получилось найти $n$ попрарно непересекающихся отрезков в исходной системе, то каждый из них остался в полученной или же остался некоторый отрезок, вложенный в него, поэтому число $n$ неуменьшилось, а так как $n$, очевидно, не могло увеличиться, то оно осталось неизменным. Таким образом, мы можем считать, что в нашей системе отрезков любые 2 или не пересекаются, или же пересекаются, но не вложенны (обозначим полученную систему за $\Phi$).\\
Введём на прямой с отрезками линейную систему координат. Рассмотрим множество систем непересекащихся систем отрезков (обозначим за $P$). Для каждой системы $A \in P$ упорядочим отрезки в порядке возрастания координаты их левой границы, и системе $A$ сопоставим $X(A) = \{x_1, x_2, \ldots, x_n\}$, где $x_i$ "--- координата левого конца $i$-того отрезка. Теперь на $P$ введём порядок, где отношением $A, B \in P$ будет отношение $X(A), X(B)$ в лексекографичексом порядке. В силу конечности $P$ найдётся некоторый наименьший элемент, а так как наш порядок линейный, то этот элемент будет минимальным, обозначим его за $S = \{s_1, s_2, \ldots s_n\}$, где $s_i$ "--- $i$-тый отрезок системы $S$.\\
Далее будем доказывать, что, ставя точку в правй конец $s_i$ (начиная с первого), верно, что мы попали хотя бы одной точкой в каждый отрезок из $\Phi$, левый конец которого лежит не раньше, чем правый конец $s_i$. После первого действия, это свойство, очевидно, выполняется, иначе найдётся больше $n$ попарно непересекающихся отрезков. ПУсть наше предположение оказалось неверным после $k$-того действия. Тогда мы нашли отрезок $l$, который не пересекается с $s_j$, при $j < k$, а его левый конец лежит раньше, чем левый конец $s_k$ (иначе $s_k \subset a$). А тогда $S$ "--- не наименьший элемент $P$. Также заметим, что после $n$-той операции все отрезки, по нашему предположению, будут содежрать хотя бы одну точку, иначе можно выбрать больше, чем $n$ попарно непересекающихся отрезков.\\
Таким образом, мы попали не менее, чем одной точкой в каждый отрезок из $\Phi$, а по вышедоказанному это равносильно тому, что для любой исходной системы верно, что $n = k$. Что и требовалось доказть.

\i Индукцией по $k$ докажем, что бесконечной антицепи не найдётся.\\
База. $k=1$, очевидно, так как порядок на $\NN$ линейный.\\
Переход. Предположим, что задача решена при $k = n$, покажем, что она также решена при $k = n+1$.
Предположим обратное, и нашлась какая-то бесконечная антицепь, выберем в ней произвольный элемент $A$. Для любого элемента антицепи $B \ne A$ верно, что найдётся координата, в которой $B < A$ (иначе $A$ и $B$ сравнимы, и $A < B$). Разобьём все элементы антицепи, отличные от $A$ на $n+1$ группу $X_1, X_2, \ldots, X_{n+1}$, помещая каждый элемент антицепи в группу, номер которой совпадает с номером координаты, в которой этот элемент меньше $A$ (если элемент можно поместить в несколько групп, то выберем произвольную). Заметим, что при этом каждый элемент (кроме $A$) попал в какую-то группу, число которых конечно, тогда, если мы докажем, что в каждой группе число элементов так же конечно, то и мощность всей антицепи также конечна.\\
Рассмотрим произвольную группу $X_i$, заметим, что у всех элементов $X_i$ $i$-тая координата ограницена некотороым числам $m$, где $m$ "--- значение $i$-той координаты $A$. Тогда разобьём $X_i$ на $Y_1, Y_2, \ldots, Y_{m-1}$, где каждый элемент будем помещать в ту группу, номер которой совпадает с его значением в $i$-той координате.
Так как $Y_j$ "--- подмножество некоторой антицепи, то $Y_j$ также является антицепью. Заметим, что если из всех элементов $Y_j$ <<выкинуть>> $i$-тую координату, то полученной множество $Z_j$ будет вложено в $\NN^k$, и, очевидно, будет являться антицепью, тогда по предположению индукции $Z_j$ конечно, следовательно, $Y_j$ конечно, а тогда и $X_i$ конечно, а это ровно то, что мы хотели доказать. Таким образом, бесконечной антицепи в $\NN^k$ не может существовать.