\subsection{Двенадцатое ДЗ}

\i Введём вероятностное пространство, где событиями будут упорядоченные пары выпавших значений на кубиках. Всего возможных исходов ровно $6^2 = 36$, при этом одинаковые значения на кубиках выпадают в $6$ из нах, а значит вероятность выпадения 2 одинаковых составляет ровно $\frac{1}{6}$.

\i Введём вероятностное пространство, где событиями будут возможные выборы числа. Всего таких выборов ровно 100, а чисел, сумма цифр которых равна 8, среди них ровно $8$ ($17, 26, 35, 44, 53, 62, 71, 80$), а значит вероятность выбора подходящего числа составляет $0{,}08$.

\i Рассмотрим вероятностное пространство, гда событиями будут возможные последовательности из результатов 4 бросков кубика. Все события равнорероятны. Всего событий ровно $6^4$. Рассмотрим все строго возрастающие последовательности из 4 чисел не превосходящих 6. Для удобства подсчётаБ будем считать, что все эти последовательности начинаются с 0 и содержит 5 чисел. Разница между соседними или всегда равна 1 (и тогда мы получим мы получим последовательность $(0, 1, 2, 3, 4) <-> (1, 2, 3, 4)$), или ровно 1 разность равна 2, таких последовательностей 4, или 2 разности равно 2, и тогда таких последовательностей 6, или 1 разность равна 3, таких последовательностей тоже 4. Таким образом, подходящих последовательностей $1+4+6+4 = 15$. Тогда вероятность того, что значения на кубиках будут совпадать составляет ровно $\frac{15}{6^4} \approx 0,011$.

\i рассмотрим вероятностное пространство, где событиями будут выборы 8 клеток для уже зафиксированного билета. Всего таких выборов ровно $\binom{64}{8}$. Всего подходящих выборов ровно $\binom{8}{4}\binom{56}{4}$, так как необходимо выбрать 4 клетки из 8 отмеченных и 4 из оставшихся. Таким образом ответ равен $\frac{\binom{8}{4}\binom{56}{4}}{\binom{64}{8}} \approx 0{,}00580871$.

\i Изобразим турнир как бинарное дерево. Заметим, что самая сильная и вторая команда встретятся в вершина, которая является их наименьшим общим предком (так как они не проигрывают никому другому). Это будет последний раунд турнира (то есть корень рассматриваемого дерева) тогда и только тогда, когда интересующие нас команды в начале турнира расположены в разных половинах уровня с листьями. Теперь рассмотрим вероятностное пространство, где событиями будут варианты расположение интересующих нас команд на уровне лисьев в турнирном дереве. Всего возможных исходов ровно $64 \times 63$, а интересующих нас $64 \times 32$. Тогда вероятность встречи самых сильных команд в последнем раунда составяет ровно $\frac{64\times32}{64\times63} \approx 0{,}507936$.

\i Рассмотрим вероятностное пространство, в котором событиями будут являться пары из возможных последовательностей из результатов бросков Пети и Васи. Заметим, что все события равновероятны. Всего возможных событий ровно $2^{21}$ так как все броски упорядочены и для каждого есть ровно 2 варианта. Если у Пети орёл выпал $k$ раз, то количество случаев, когда у Васи орлов выпало меньше равно $\sum\limits_{i=1}^{k-1} \binom{10}{i}$, так как ровно столько вариатнов, когда у него $0, 1, \ldots, k-1$ монета, то есть все случаи, когда у него меньше. Тогда всего случаев, когда у Пети $k$ монет, а у Васи меньше, ровно $\binom{11}{k}\sum\limits_{i=0}^{k-1}\binom{10}{i}$. Значит всего устраивающих нас вариантов $\sum\limits_{k=0}^{k\leq11}\brackets{\binom{11}{k}\sum\limits_{i=0}^{k-1}\binom{10}{i}}$. Таким образом искомая вероятность равна $\frac{\sum\limits_{k=0}^{k\leq11}\brackets{\binom{11}{k}\sum\limits_{i=0}^{k-1}\binom{10}{i}}}{2^{21}} = \frac{1}{2}$.

\i Рассмотрим все полные турниры на $n$ командах. Будем считать, что результат каждого матча определяется случайно и равновероятно (то есть, $0{,}5$ что победит первая команда и $0{,}5$, что вторая). Пусть $N = \frac{n(n-1)}{2}$, тогда всего возможных турниров ровно $2^N$. При этом веротность того, что случился конкретный турнир равна произведению вероятностей каждого из матчей в нём, то есть равна $2^{-N}$. Таким образом все возможные турниры равновероятны. Будем называть турнир плохим, если в нём нашлось 10 команд, для которых не нашлось противника, которому они все проиграли (такую 10-ку будем называть плохой). Заметим, что по формуле включений-исключений верно, что вероятность того, что турнир плохой меньше или равна вероятности того, что 10-ка оказалось плохой на количество 10 в турнире (модуль объединения не больше, чем сумма модулей). Тогда вероятрость того, что турнир на $n$ командах оказался плохим не больше, чем $\frac{\binom{n}{10}\times P_1}{2^N}$, где $P_1$ вероятность того, что случайная 10-ка в случайном турнире оказалась плохой. Заметим, что $P_1 \leq 1$ и $\binom{n}{10} \leq n^{10}$. Следовательно вероятность того, что турнир оказался плохим не больше чем $\frac{\binom{n}{10}\times P_1}{2^N}$, что не больше чем $\frac{n^{10}}{2^{\frac{n(n-1)}{2}}} \leq \frac{n^{10}}{2^n}$, что меньше 1 при достаточно больших $n$ так как $n^{10}$ асимптотически меньше, чем $2^n$ Таким образом, Для достаточно большого $n$ вероятность того, что произвольный турнир оказался плохим меньше 1, а значит найдётся и турнир, в котором для любой 10-ки команд найдётся такая, которая выиграла у всех из них.