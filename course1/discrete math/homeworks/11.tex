\subsection{Одиннадцатое ДЗ}


\i Предположим обратное, тогда существует некоторое множевтво $X = S_i$, которое содержится в $S_1 \cup S_2 \cup \ldots \cup S_l, \ i > l$. В таком случае рассмотрим объединение $S_1 \cup S_2 \cup \ldots \cup S_l \cup S_i$. В таком объединении по условию содержится ровно $l$ элементов, но при этом оэто объединение состоит из $l+1$ множества, значит не выполняется условие разнообразия, получается противоречие. Что и требовалось доказать.

\i \label{prob.2} Давайте назовём искомое паросочетание $P$. Сразу же скажем, что в $P$ мы включим $M \cap M\hatch$, и в дальнейшем решении будем считать, что паросочетания $M$ и $M\hatch$ не пересекаются. Если некоторое ребро из $M \cup M\hatch$ не перескается ни с каким другим из этого же пересечения, то мы включаем его в $P$ и выкидываем из $M$ и $M\hatch$ (из того, в котором оно лежало). Если мы сумеем обработать так все рёбра, то мы победим. Если же м в какой-то момент остановились, то у нас осталось несколько путей, в которых чередуются рёбра из $M$ и $M\hatch$. Рассмотрим произвольный недополнимый из них. Пусть он, без ограничения общности начинается в первой доле ребром из $M$. Возьмём из него только рёбра из паросочетания $M$. Заметим, что они покрывают все вершины из первой доли и из второй (те вершины, которые покрывало хотя бы одно ребро из рассматриваемого пути). Таким образом мы смогли выбрать набор рёбер такой, что он покрывает все вершины из $S \cup T$, осталось понять, почему это паросочетание. Это очевидно, так как те рёбра, которые были включены в одно паросочетание изначально не пересекаются, а не, которые были в разных обрабатываются одновременно со всеми, с которыми могут пересечься, и мы выбираем их так, чтобы паресечений не было. таким образом получится искомое паросочетание.

\i Для начала рассмотрим наименьшее вершинное покрытие в $G$. В силу того, что каждая вершины имеет степень не больше $d$, наименьшее вершинное покрытие должно содержать хотя бы $\frac{l}{d}$ вершин, а тогда существует паросочетание, такого же размера. Что и требовалось доказать.

\i Давайте рассмотрим двудольный граф, где вершинами первой доли будут подмножества мощности $k$, а второй "--- $k+1$, ребро будем проводить, если подмножество из первй доли включено в подмножество из второй. В первой доле будет ровно $\binom{n}{k}$, в во второй "--- $\binom{n}{k+1}$. Также, как известно, при ограничен иях на $k$ из задачи верно, что $\binom{n}{k} \leq \binom{n}{k+1}$. Полученный граф будет регулярным, то есть, степени всех вершин в каждой из долей равны между собой (что очевидно), тогда легко убедиться, что выполнено условие разнообразия, а значин найдётся паросочетание мощности $\binom{n}{k}$, а это в точности то, что мы и хотим (каждому подмножеству из первой доли сопоставим его пару в паросочетании из второй). Что и требовалось доказть.

\i Давайте решать эту задачу по индукции по $d$.\\
База. $d = 1$ "--- произвольный граф с вершинами, степени не больше, чем 1. Очевидно, что из себя он представляет набор отдельных рёбер, а значит и разбивается на 1 паросочетание.\\
Переход. Пусть задача решена для $d = k$, докажем её для $d = k+1$. Для этого рассмотрим первую долю графа ($A$). Оставим в ней только те вершины, степень которых в точности равна $k+1$ ($S$). В силу того, что степени вершин во второй доле после этой операции так же не превышают $k+1$ (мы не рассматриваем рёбра, содержащие вершины из первой доли степени $k+1$), верно условие ранообразия, а значит мы можем применить лемуу Холла. Таким образом мы найдём паросочетание на всех вершинах степени $k+1$ из первой доли ($M$). После этого выделим во второй доле ($B$) все вершины степени $k+1$ ($T$) и снова применим лемму Холла, получим новое паросочетание $M\hatch$. Осталось применить для полученных множеств и паросочетаний задачу \ref{prob.2}. Тогда мы получим паросочетание, которое покрывает все вершины степени $k+1$. Выделим его и применим предположение индукции для графа без этих рёбер и $d=k$. Таким образом мы совершили переход и доказали задачу для произвольного $d$.

\i Построим двудольный граф, в котором вершинами первой доли будут подмножества $A_i$, а второй "--- $B_i$. Между двумя вершинами будем проводить ребро цвета $s$, если соответствующие подмножества пересекаются по элементу $s$ (при этом мы разрешаем кратные рёбра). Тогда степень каждой вершины в точности $k$ иполученный граф является регулярным, и при этом степени вершин в первой доле не меньше, чем во второй, а значит выполнено условие разнообразия. Таким образом по лемме Холла найдётся паросочетание размера $p$. Рассмотрим цвета рёбер, вошедших в это паросочетание, пусть это $s_1, s_2, \ldots s_p$. Заметим, что соответствующие элементы будут образовывать систему различных представителей, как для $A_i$, так и для $B_i$. Что и требовалось доказать.

\i Пусть $A = \{A_0, A_1, A_2, \ldots\}, B = \{B_1, B_2, \ldots\}$. $A_0$ соединим рёбрами со всеми вершинами доли $B$, а любую вершину $A_i, \ i \ne 0$, соединим с $B_i$. Докажем, что для любого $S \subset A$ верно, что $\abs{S} \leq \abs{N(S)}$. Если $S$ содержит $A_0$, то это очевидно, а если нет, то для любого $A_i \in S$ найдётся уникальное $B_i$, связное с ним. Таким образом, очевидно, условие разнообразия выполнено. Осталось понять, почему в нашшем графе не может быть паросочетания на всех вершинах $A$. Предположим, что такое нашлось, тогда пусть $A_0$ по ребру паросочетания связано с некоторой вершиной $B_k$, заметим, что в таком случае никакое ребро не паросочетания не покрывает вершину $A_k$, следовательно паросочетания, покрывающего все вершины $A$ не существует. Таким образом лемма Холла не всегда верна для бесконечных графов.