\subsection{Пятое ДЗ}


\i В любом бесконечном множестве можно выделить счётное подмножество, зная это рассмотрим счётное $C \subset A \backslash B$. Пользуясь тем, что $C, B$ "--- счётные множества мы можем пронумеровать их элементы и записать $B = \{b_1, b_2, \ldots \}, \ C = \{c_1, c_2, \ldots \}$.\\
Теперь построим биекцию $f A \mapsto A \backslash B$ по следующему правилу:
\begin{itemize}
    \item $f(x) = x, x \in (A \backslash B) \backslash C$;
    \item $f(x) = c_{2i-1}, x \in C \And x = c_i$;
    \item $f(x) = c_{2i}, x \in B \And x = b_i$.
\end{itemize}
Осталось только понять, почему эта функция является биекцией. Абсолютно очевидно, что для любого $x$ наша функция определена, и при том однозначно. Также заметим, что при разных $x\ f(x)$ принимат различные значения. Осталось лиш понять, почему для любого $y \in A \backslash B$ найдется $x \in A: f(x) = y$. Если $y \ in (A \backslash B) \backslash C$, то $x = y$, если $y \in C \And y = c_{2i}, x = b_i$, если $y \in C \And y = c_{2i-1}, x = c_i$. Что и требовалось доказать. Таким образом $A$ равномощно $A \backslash B$. 

\i Давайте доказывать, что это не верно, для этого просто приведйм пример.\\
$\letus A = \NN \backslash \{2, 4, 781\}, B = \NN$. Тогда, очевидно, что $A, B$ счётны, а $A \backslash B = \{2, 4, 781\}$ "--- конечно. Таким образом мы привели пример, когда $A$ и $A \backslash B$ не равнмощны.

\i Для начала поймём, что нам достаточно построить биекцию между интервалом и отрезком, которая оставляет середину на места. Тогда применив эту биекцию ко всем диаметрам круга без границы получим биекцию, которая переводит круг без границы в круг с границей. Построим искомую биекцию $f$.\\
Для начала поймём, что нам достаточно построить $f: (0, 1) \mapsto [0, 1]$, так как все остальные случаи сводятся к этому гомотетией. Определим $f(x)$ следующим образом:\\
Для начала выбедим на $(0, 1)$ счётное множество $A = \{x: x = \frac{1}{n}, n \in \NN\}$, тогда $f(x) = x, x \notin A,\ f(\frac{1}{a}) = \frac{1}{a-3}, a\notin \{2, 3, 4\},\ f(\frac{1}{2}) = \frac{1}{2},\ f(\frac{1}{3}) = 0, f(\frac{1}{4}) = 1$. Вполне очевидно, что описанная функция является биекцией, а значит задача решена.

\i Заметим, что у любого многочлена конечное число корней, а значит, если мы сумеем пронумеровать все многочлены, то потом просто будем нумеровать их корни в соответствии с порядком нумерации многочленов (сперва все корни первого, затем второго, и так далее), если какое-то число повторится, то просто пропустим его. Замечательно, давайте предбявим алгоритм, как пронумеровать многочлены с целыми коэфицентами (то есть построить биекцию с $\NN$).\\
Для начала пронумеруем все многочлены фиксированной степени $d$. Для этого воспользуемся тем, что мы можем пронумеровать все целые числа. Тогда каждому многочлену сопоставим последовательность из номеров его коэфицентов и будем нумеровать эти последовательности. Заметим, что множество этих последовательностей равномощно $\NN^d$, что как известно счётно (это утверждение равносильно тому, что $\NN^2 \sim \NN$ (что равносильно $\QQ \sim \NN$), а дальше можно доказать по индукции по степени ($\NN^k \sim \NN \times \NN^{k-1} \sim \NN^2 \times \NN^{k-1} \sim \NN^{k+1}$)). Таким образом, все многочлены можно пронумеровать.\\
После этого на $i$-той итерации нашего алгоритма будем присваивать номера с $\frac{i(i-1)}{2}+1$ по $\frac{i(i+1)}{2}$ многочленам со стеенями от 1 до $i$, а конектернее, для каждой степени будем присваивать номер многочлену, с наименьшим номером во <<внутренней>> нумерации (нумерации среди многочленов той же степени), у которого еще нет номера в <<общей>> нумерации (среди всех многочленов).\\
Таким образом, каждый многочленн получит свой номер, а по вышедоказанному это равносильно тому, что свой номер получат и все алгебраические числа (при этом очевидно, что каждому номеру будет сопоставлено какое-то число), таким образом мы построим биекцию между $\NN$ и множеством алгебраических чисел, следовательно множество алгебраических чисел счётно. Что и требовалось доказать.

\i Как известно, множество рациональных чисел всюду плотно на множестве вещественных, а значит и на любом отрезке прямой (на которой мы строим биекцию с $\RR$) найдётся рациональное число (это просто условие всюду плотности). После этого, по аксиоме выбора, мы сможем каждому отрезку сопоставить рациональное число, которое на нём лежит, при этом все выбранные числа оказутся одинаковыми, так как отрезку по условию не пересекаются.\\
Теперь воспользуемся тем, что $\QQ$ счётно, а значит мы можем пронумеровать все рациональные числа. Осталось только упорядочить все отрезки по номеру наименьшего рационального числа, включённого в этот отрезок и пронумеровать, что равносильно тому, что рассматриваемое множество конечно или счётно.

\i Известно, что в любом бесконечном множестве можно выделить счётное подмножество, давайте этим воспользуемся, и выделим в данном в условии множестве счётное подмножество $A$. Докажем, что $A$ содержит бесконечное множество непересекающихся счётных подмножеств. Просто предъявим алгоритм их построения.\\
Пронумеруем элементы $A$, после этого будем <<раскидывать>> их по множествам $A_1, A_2, \ldots$. На $i$-той итерации нашего алгоритма будем <<раскидаем>> элементы с номерами с $\frac{i(i-1)}{2} + 1$ по $\frac{i(i+1)}{2}$ в множества $A_1, A_2, \ldots, A_i$ по одному (что возможно, так как количество множеств на каждой итерации конечно и равно количеству элементов). Таким образом мы получим систему непересекающихся подмножеств множества $A$ счётной мощности, где каждое подмножество также будет счётным. Что и требовалось доказать.

\i Для начала пронумеруем всё целые числа (это возможно, так как, как известно, $\ZZ$ счётно). Далее будем <<назначать>> значения биекций в целых числах в порядке их нумерации.\\
На $i$-той итерации рассмотрим целое число $x_i$ далеевыберем <<главную>> биекцию: если $i \modeq{3} 0$, то это $g_3$, если $i \modeq{3} 1$ "--- то $g_1$, во всех остальных случаях"--- $g_2$. Далее <<главной>> биекции <<назначим>> в точке $x_i$ наименьшее по номеру значение, которое она еще не принимает, о оставшимся двум "--- некоторые значение $a$ и $b$, такие, что оба по модулю больше, чем любое значение, уже присвоенное нашим биекциям, и при этом выполняется равенство $f(x_i) = g_1(x_i) + a + b => f(x_i) - g_1(x_i) = a + b$, что, очевидно, имеет сколь угодно большие по модулю решения. Таким образом $g_i$ будут определены на всех целых числах, и в разных точках их значения будут отличаться. Осталось понять, почему для любых $x, i$ существует $k$ такое, что $g_i(x) = k$. Пусть в нашей нумерации целых чисел $k$ имеет номер $r$, тогда не позднее, чем за $3r$ итераций нашего алгоритма будет найдено искомое $x$, а значит $g_i$ "--- биекция. Что и требовалось доказать.