\subsection{Двадцать первое ДЗ}

\i Для начала заметим, что $t(x) = 1$ тогда и только тогда, когда $x = x_0 = \{1, \ldots, 1\}$. При всех $x \ne x_0\ t(x) = 0$, и желаемое неравенство очевидно (и для $f$ и для $g$), осталось понять, что из условия следует, что при $x=x_0\ 1 = t(x_0) \leq f(x_0) + g(x_0)$, а значит или $g(x_0) = 1$ (и тогда желаемое неравенство достигается для $g$), или $f(x_0) = 1$ (тогда желаемое неравенство достигается для $f$). Что и требовалось доказать.

\i Будем говорить, что набор из 3 значений переменных подходит для дизъюнкта, когда на нём этот дизъюнкт принимает значение 1, иначе будем говорить, что набор не подходит. Заметим, что в любой дизъюнкт можно подставить 2 наборов значений, при этом только один из них не будет подходить. Теперь предположим, что у нас получилось использовать $n$ переменных и $m$ дизъюнктов, чтобы получить КНФ, которая не имеет решений. Это означает, что при любой подстановке значений (которых всего $2^n$) хотя бы один дизъюнкт обнуляется. Осталось заметить, что любой дизъюнкт обнуляется тогда и только тогда, когда значения переменных, которые в него входят фиксированные, и все остальные произвольные, тогда любой дизъюнкт обнуляется не более, чем в $2^{n-3}$ подстановках. Отсюда уже очевидно, что для получению желаемой КНФ понадобится не меньше 8 дизъюнктов.\\
Осталось привести пример, в качестве него можно взять все возможные 8 дизъюнктов на 3 переменных (то есть $2^3$ подстановок к ним отрицания). Тогда тот дизъюнкт, в котором все истинные переменные стоят с отрицанием, а ложные "--- без него, занулится, и при этом очевидно, что такой будет.

\i Начнём с того, что ранжируем нашу схему, подвесив её за вершину, соответствующую выводу. Это возможно так как данная структура представляет из себя ориентированное дерево. После этого, на очередном раше будем рассматривать отрицание, расположенное на самом высоком уровне, и при этом такое, что его потомок не является листом (так как такое отрицание уже представляет из себя отрицание переменной), и по закону Де Моргона заменять $\neg(a \vee b)$ на $\neg a \wedge \neg b$ и $\neg(a \wedge b)$ на $\neg a \vee \neg b$ (то есть, отрицания теперь будут расположены на том же уровне, на котором до этого распологалось утверждение, от которого бралось отрицание). Таким образом, мы <<сдвинем>> отрицания на один уровень вниз. В силу того, что схема имеет конечную глубину, такой алгоритм рано или поздно закончится, то есть все отрицания будут браться от листьев. На самом деле, в самом начале ещё стоит избавится от всех многократных отрицаний, и только после этого ранжировать схему (очевидно, что после этого мы не увеличим число вершин, а значит оценка для новой схемы так же подойдёт и для старой). Осталось понять, какова сложность новой схемы. В силу того, что на каждой операции мы заменяли конъюнкцию на дизъюнкцию или наоборот и делали что-то с отрицаниями, то число вершин соответствующих не отрицаниям не изменилось. При этом у нас исчезло сколько-то отрицаний, и добавилось не больше, чем число листьев, то есть сложность формулы. Таким образом, новая сложность не превосходит $2s$.

\i Начало будет аналогично предыдущей задаче. Снова ранжируем схему, подвесив её за вывод. Далее снова пойдём по схеме <<сверху -- вниз>> Заметим, что, если значение некоторой ячейки <<скармливается>> только отрицаниям, то можно провернуть аналогичный фокус (применить закон Де Моргона как в прошлой задаче), если же значение ячейки так же переходит и в некоторую другую, которая не является отрицанием, то давайте создадим её копию (копию той, которая <<скармливается>> отрицанию) и все выводы не в отрицания из оригинала перекинем в неё, а вводы просто продублируем. Тогда с тем, что осталось снова можно применить закон Де Моргона. Таким образом, у нас опять количество отрицаний на уровне, на котором мы стоим строго убывает при каждой очередной операции, а количество уровней не меняется, а значит предъявленный алгоритм конечен. Аналогично прошлой задаче мы каждой итерацией добавляем не больше одной вершины, однако в новом случае у каждой вершины может дополнительно появится до 1 клона, а значит новую сложность можно оценить как $3s$.

\i Предположим, что существует такая КНФ, которая удовлетворяет условиям, но при этом невыполнима. Тогда рассмотрим случайные значения переменных в этой КНФ. Так как по нашему предполажению она невыполнима, то какой-то дизъюнкт ($A$) даёт 0. Попробуем исправить его так, чтобы не испортить другие. Для этого рассмотрим произвольную переменную, входящую в $A$, после смены её значения выбранный дизъюнкт станет верным (что очевидно), однако могут испорититься другие. Предположим, что для любого выбора переменной из $A$ хотя бы один другой всегда становится ложным (из тех, которые до этого были верны). Это в свою очередь означает, что для любой переменной этого дизъюнкта существует хотя бы один другой дизъюнкт, в который она входит с другим <<знаком>> (то есть с отрицанием или без него). Тогда рассмотрим произвольную переменную $x$ из $A$ и произвольный дизъюнкт который она <<портит>> при замене ($B$). По условию в $A$ и $B$ есть ещё хотя бы одна переменная $y$, входящая в оба с разными знаками, что в свою очередь означает, что или $A$ на самом деле принимает истинное значение, или $B$ нельзя испортить заменой $x$, так как в первом случае значение $y$ удовлетворяет $A$, а во втором "--- $B$, и одно из этих двух условий обязательно выполнено. Таким образом, мы всегда сможем исправить неверный дизъюнкт, а значит рассматриваемая КНФ всегда выполнима. Что и требовалось доказать.