\subsubsection{Девятое ДЗ}


\i Предположим, что такой граф существует, тогда <<выкинув>> из него вершину степени 1 (вместе с ребром из неё), мы получим граф на 7 вершинах с 22 рёбрами, однако, в графе на 7 вершинах не может быть больше, чем $\frac{7 \cdot 6}{2} = 21$ ребро, а значит, такое невозможно.

\i Заметим, что на трёх вершинах степени 4 уже есть 12 рёбер, при том мы могли посчитать дважды не более, чем 3 из них (по одному на каждую пару). Однако, если мы посчитали дважды 3 ребра, то они уже образуют цикл, а если мы посчитали дважды не более 2 рёбер, то всего в графе на 10 вершинах их не менн 10, а значит он в любом случае не может быть деревом.

\i Подвесим дерево за его корень, тогда по условию в нём $n$ уровней (не считая сам корень), при этом на уровне с номером $k$ ровно $2^k$ вершин. Также, очевидно, что наибольшее расстояние не превосходит $2n$ (так как растояние до любой вершины до корня не больше $n$). Также заметим, что есть вершины, расстояния между которыми в точности равны $2n$ "--- это листья, которые являются потомками разных детей корня (так как нам гарантированно придётся пройти через корень, а расстояние до него от каждой из рассматриваемых вершин в точности равно $n$). Всего таких пар, очевидно, $\brackets{2^{n-1}}^2$. Покажем, что никакие другоие пары не подходят.\\
Заметим, что обе решины должны быть листьями, иначе, мы доберёмся до корня от первой и до второй от корня меньше, чем за $2n$, а при этом, такой путь не меньше расстояния между вершинами. Если же оба листа являются потомками одного ребёнка корня, то нам достаточно дойти до этого ребёнка, а не до корня, и тогда расстояние будет не больше, чем $2(n-1)$.\\
Таким образом, мы показали, что всего подходящих пар ровно $\brackets{2^{n-1}}^2$. Осталось заметить, что для каждой пары существует ровно 2 варианта порядка, а значит, существует ровно 2 пути. Следовательно ответ "--- $2(2^{n-1})^2$.

\i Давайте построим граф, где вершинами будут являться области, на которые исходные кубики внутри большого куба разбивают пространство (всего $(n-2)^3$ маленьких внутренних кубиков и $1$ область, полученная сливанием всех кубиков, соприкосающихся с внешники гранями большого куба), и будем соединять их ребром тогда и только тогда, когда в исходной картинке 2 области были смежными по перегородке, а после этого перегородка была стёрта. Заметим, что наше условие о том, что м ысможем добраться до <<внешнего>> кубика равносильно тому, что мы нарисуем связный граф, а значит мы провели не менее, чем $(n-2)^3 + 1 - 1$ ребро, то есть стёрли ровно $(n-2)^3$ перегородок. Осталось показать, что такого числа перегородок достаточно. Зафиксируем неоторое направление, перпендикулярное направлению кубиков, а после этого в каждом <<внутреннем>> кубике (в том, что лежит в центральном кубе $(n-2)^3$) сотрём перегородку в выбранном направлении, тогда, очевидно, что полученная фигура удовлетворяет условию, и мы стёрли ровно $(n-2)^3$ перегородок.

\i Давайте решать эту задачу индукцией по $n$ (для каждого $k$ отдельно).\\
База. $n = k + 1$. Очевидно, что $K_n$ удовлетворяет условию.\\
Переход.\\
Если $k$ чётно, то предположим, что задача решена для $n = a$, докажем её для $n = a + 1$.  <<Забудем>> про одну из вершин и по предположению индукции посторим необходимый граф для $n - 1$ и $k$ по предположению индукции. Затем <<вернём>> забытую вершину (назовём её $X$) и выберем произвольные $k/2$ рёбер (обозначим их за $A_iB_i$) (заметим, что $k$ рёбер гарантированно найдутся). Далее заменим рёбра $A_iB_i$ на $A_iX$ и $XB_i$. Заметим, что полученный граф удовлетвряет условию.\\
Если $k$ нечётно, то предположим, что задача решена для $n = a$, докажем её для $n = a + 2$. <<Забудём>> про пару вершин (обозначим их за $X$ и $Y$). По предположению индукции на оставшихся вершинах мы смогли построить граф, удовлетворяющий условию для $n - 2$ и $k$. <<Вернём>> забытую пару и выберем $k-1$ рёбер (обозначим их за $A_iB_i$) (заметим, что $k-1$ рёбер гарантированно найдутся). Далее заменим рёбра $A_iB_i$ на $A_iX$ и $YB_i$ и проведём ребро $XY$.Заметим, что полученный граф удовлетвряет условию.

\i Предположим обратное, тогда нашлось некоторое множество рёбер мощности меньше $n$ такое, что после их удаления граф расался на несколько компонент связности. Заметим, что какая-то из них будет состоять из не более, чем $n$ вершин. Обозначим число вершин в ней за $k$. Рассмотрим произвольную вершину из этой компоненты, заметим, что <<наружу>> этой компоненты в исходном графе шло хотя бы $n - (k-1)$ ребро, и таких вершин ровно $k$, следовательно, в исходном графе межды этой компонентой и другими было хотя бы $k(n - (k-1))$ рёбер. Рассмотрим эту формулу, как функцию от $k$. Заметим, что получится парабола с ветвями вниз, что является выпуглой вверх функцией, а значит минимум достигается на концах области определения (при этом, очевидно, $k \in \{1, 2, \ldots, n\}$). Таким образом, минимальное число рёбер, которые надо удалить, чтобы изолировать рассматриваемую компоненту хотя бы $min(1(n - (1-1)), n(n - (n-1)) = n$. Противоречие, так как мы удалили строго меньше $n$ рёбер.

\i Давайте называть галочкой пару рёбер вида $AB$, $BC$, где $A, C$ "--- концы галочки, а $B$ "--- центр. Для начала заметим, что каждой паре вершин соответствует ровно 2 галочки, и каждая галочка соответствует какой-то паре (соответствие: пара вершин = концы галочки). Таким образом галочек в точности $\frac{n(n-1)}{2} \times 2 = n(n-1)$.\\
Рассмотрим произвольную вершину $A$, тогда для любой вершины $B$ найдётся пара вершин $U, V$, таких, что обе связаны в $A$. Также, легко понять, что для любого выбора $B$ полученная пара $U, V$ будет отличаться (иначе нарушится условие для вершин $U, V$). Таким образом, легко понять, что $A$ фвляется центром хотя бы $n-1$ галочки, а значит все вершины являются центрами строго $n-1$ галочки (иначе их больше, чем $n(n-1)$).\\
Таким образом, у каждой вершины одинаковое количество галочек, а значит у них одинаковые степени (так как количество галочек равно $\frac{v(v-1)}{2}, v$ "--- степень вершины). Что и требовалось доказть.
