\subsection{Первое ДЗ}

\i Давайте вести индукцию по $n$.\\
\textbf{База.} $n=1: \ \frac{1}{1+1} > \frac{13}{24}$, занчит база индукции верна.\\
\textbf{Переход.} Пусть наше неравенство верно для $n = k$, докажем, что оно верно для $n = k + 1$:
\begin{gather*}
    \frac{1}{k+2} + \frac{1}{k+3} + \ldots + \frac{1}{2k} + \frac{1}{2k+1} + \frac{1}{2k+2} > \frac{13}{24};\\
    \brackets{\frac{1}{k+1} + \frac{1}{k+2} + \ldots + \frac{1}{2k}} + \brackets{\frac{1}{2k+1} + \frac{1}{2k+2} - \frac{1}{k+1}} > \frac{13}{24};\\
    \frac{1}{2k+1} + \frac{1}{2k+2} - \frac{1}{k+1} > 0;\\
    \frac{1}{2k+1} + \frac{1}{2k+2} > \frac{1}{k+1};\\
    (2k+2)(k+1) + (2k+1)(k+1) > (2k+1)(2k+2);\\
    4k^2 + 3k + 3 > 2k^2 + 3k + 2;\\
    1 > 0.
\end{gather*}
Таким образом переход индукции доказан и задача решена.

\i Давайте вести индукцию по $n$.\\
\textbf{База.} $n = 1: \frac{1-\frac{1}{2}}{1+\frac{1}{3}} = 3$, значит база индукции верна.\\ 
\textbf{Переход.} Пусть наше неравенство верно для $n = k$, докажем, что оно верно для $n = k + 1$:\\
для начала давайте подставим вместо $x_k$ $(x_k + x_{k+1})$, тогда сумма $x_i$ не изменилась и можно применить предположение индукции:
\begin{gather*}
    \frac{1-x_1}{1+x_1} + \frac{1-x_2}{1+x_2} + \ldots + \frac{1-x_k-x_{k+1}}{1+x_k+x_{k+1}} \geqslant \frac{1}{3}.\\
    \text{В таком случае, если мы докажем следующее неравенство, мы решим задачу:}\\
    \frac{1-x_1}{1+x_1} + \frac{1-x_2}{1+x_2} + \ldots + \frac{1-x_k}{1+x_k} + \frac{1-x_{k+1}}{1+x_{k-1}} \geqslant \frac{1-x_1}{1+x_1} + \frac{1-x_2}{1+x_2} + \ldots + \frac{1-x_k-x_{k+1}}{1+x_k+x_{k+1}};\\
    \frac{1-x_k}{1+x_k} + \frac{1-x_{k+1}}{1+x_{k-1}} \geqslant \frac{1-x_k-x_{k+1}}{1+x_k+x_{k+1}};\\
    \text{давайте для удобства записи выполним следующую замену} m+c = x_k, m-c = x_{k+1};\\
    \frac{1-(m+c)}{1+(m+c)} + \frac{1-(m-c)}{1+(m-c)} \geqslant \frac{1-2m}{1+2m};\\
    \frac{2+2c^2-2m^2}{1+2m+m^2-c^2} \geqslant \frac{1-2m}{1+2m}.
    \intertext{Лекго заметить, что при $c\ne0$ у дроби в левой части неравенства числитель строго увеличится, а знаменатель строго уменьшится, а значит левая часть неравенства минимальна при $c=0$, в то время, как правая не зависит от $c$. Таким образом нам достаточно решить данное неравенство только при $c=0$:}\\
    2\brackets{\frac{1-m}{1+m}} \geqslant \frac{1-2m}{1+2m};\\
    2(1-m)(1+2m) \geqslant (1-2m)(1+m);\\
    2+2m-4m^2 \geqslant 1-m-2m^2;\\
    m+1 \geqslant 2m^2;\\
    \intertext{что равносильно тому, что $m \in \left[-\frac{1}{2}, 1\right]$, что верно, так как по условию $m \in \left(0, \frac{1}{4}\right]$.}
    \intertext{Таким образом, переход индукции доказан и задача решена.}
\end{gather*}

\i Давайте по индукции доказывать неравенства вида
$$\sqrt{2\sqrt{3\sqrt{ \ldots \sqrt{(n-1)\sqrt{(n+1)^2}}}}} \leq 3$$
\textbf{База.} $n = 2: \sqrt{(2+1)^2} = 3$, база верна.\\
\textbf{Переход.} Предположим, что неравентво доказано для $n=k$, докажем, что оно также верно для $n=k+1:$
\begin{gather*}
    \sqrt{2\sqrt{3\sqrt{ \ldots \sqrt{(k-1)\sqrt{(k+1)^2}}}}} = \sqrt{2\sqrt{3\sqrt{ \ldots \sqrt{(k-1)(k+1)}}}} <\\
    < \sqrt{2\sqrt{3\sqrt{ \ldots \sqrt{(k-2)\sqrt{k^2}}}}}.
\end{gather*}    
Что по предположению индукции не превосходит $3$, переход доказан. Тогда справедливо следующее неравенство:
\begin{gather*}
    3 \geqslant \sqrt{2\sqrt{3\sqrt{ \ldots \sqrt{(n-1)\sqrt{(n+1)^2}}}}} > \\
    > \sqrt{2\sqrt{3 \ldots \sqrt{(n-1)\sqrt{n}}}}.
\end{gather*}
Что и требовалось доказать. 

\i Для начала скажем, что мы будем называть отрезок, оба конца которого отмечены, хорошим. Тогда от нас требуется доказать, что в указанной в условии конструкции найдутся хорошие отрезки всех длоин от $1$ до $3^n$. Докажем это по индукции по $n$.\\
\textbf{База.} Для $n = 1$ задача очевидна.\\
\textbf{Переход.} Предположем, что задача решена для $n=k$, докажем, что она также верна для $n=k+1$. После первого разбиения отрезка длины $3^{k+1}$ мы получим 2 отмеченный отрезка длины $3^n$ (далее будем называть их $A$ и $B$), для которых верно предположение инукции, также, при наложении все их отмеченные точки совпадут (что очевидно). В таком случае, мы уже нашли хорошие отрезки всех длин от $1$ до $3^n$ и длин $2\times3^n$ и $3^{n+1}$ (правые точки $A$ и $B$ и сам исходный отрезок). Теперь покажем, как найти все хорошие отрезки с длинами $l: 3^k < l < 2 \times 3^k$. Достаточно выбрать отрезок длины $3^{k} - l$ в $A$ и его правый конец соединить с левым концом аналогичного (совпадающего при наложении) отрезка в $B$. Если же $2\times3^k < l < 3^{k+1}$, то найдём в $A$ отрезок длины $3^k - l$ и ссоединим его левый конец с правым концом аналогичного отрезка из $B$. Легко убедиться, что в обоих случаях мы получим отрезки искомой длины, а значит сможем получить хорошие отрезки всех длин от $1$ до $3^{k+1}$. Таким образом, переход доказан и задача решена.

\i Для решения задачи давайте покажем, что попав в точку $n$, кузнечик смодет за несколько прыжков попасть в точку $n+1$ (или $n-1$, если ищменить направление всех прыжков).\\
Пусть до попадания в точку $n$ кузнечик сделал $k-1$ прыжок. Давайте докажем, что сделав $2^k$ прыжков влево, а затем один правко, кузнечик в итоге попадет в точку $n+1$, то есть:
\begin{gather*}
    \brackets{2^{k+2^k}+1} - \brackets{2^k+1} - \brackets{2^{k+1}+1} - \ldots - \brackets{2^{k+2^k-1}+1} = 1;\\
    \text{что равносильно:}\\
    2^{k+2^k} - \brackets{2^k + 2^{k+1} + \ldots + 2^{k+2^k-1}} + 1 - 2^k = 1;\\
\end{gather*}
Давайте докажем это равенство по индукции по $k$.\\
\textbf{База.} $k=1: 2^{1+2^1} - \brackets{2^1 + 2^{1+1}} + 1 - 2^1 = 1$.\\
\textbf{Переход.} Предположим, что наше предположение верно для $k=l$, докажем, что оно верно для $k=l+1$:\\
\begin{gather*}
    2^{l+2^l} - \brackets{2^l + 2^{l+1} + \ldots + 2^{l+2^l-1}} + 1 - 2^l = 1;\\
    2^{l+2^l} = 2^l + 2^l + 2^{l+1} + \ldots + 2^{l+2^l-1};\\
    2^{l+2^l} \times 2 = \brackets{2^l + 2^l + 2^{l+1} +  \ldots + 2^{l+2^l-1}} + \brackets{2^l + 2^l + 2^{l+1} + \ldots + 2^{l+2^l-1}};\\
    2^{l+2^{l}} + 2^{l+2^{l+1}} = \brackets{2^l + 2^l + 2^{l+1} + \ldots + 2^{l+2^l-1}} + \brackets{2^{l+2^l} + 2^{l+2^l} + 2^{l+1 + 2^l}+ \ldots + 2^{l + 2^{l+1}}};\\
    2^{(l+1)+2^{(l+1)}} - \brackets{2^{(l+1)} + 2^{(l+1)} + 2^{(l+1)+1} + \ldots + 2^{(l+1)+2^{(l+1)}-1}} + 1 -2^{(l+1)} = 1.
\end{gather*}
Таким образом перезод индукции верен, а значит задача решена.


\i Давайте вести индукцию по количеству лапмочеку на столе ($n$).\\
\textbf{База.} $n=1$, очевидна.\\
\textbf{Переход.} Предположим, что задача решена для $n \leq k$, докажем её для $n=k+1$. Давайте на время забудем про лампочку с номером $i$, тогда по предположению индукции, у нас получилось погасить все лампочки, кроме неё, сделав при этом последовательность операций, которую мы будет обозначать как $A_i$. Если при этом погасла и лампочка $i$, то переход доказан, поэтому будем считать что для любого $i$ последовательность $A_i$ гасит все лампочки, кроме $i$. Очевидно, что операция нажимания на кнопки коммутативна, тогда последовательное нажатие на $A_i$ и $A_j$ изменит состояние всех лампочек, кроме $i$ и $j$, давайте называть такую операцию $B_{i, j}$. Рассмотрим произвольную кнопку $X$, которая соединена с нечетным числом лампочек (такая найдется по условию) и операций типа $A$ погасим все лампочки, соединённые с $X$, кроме одной, а затем нажмём на $X$. После этого на табло останетсяч чётное число горящих лампочк, которые можно будет погасить операциями типа $B$. Таким образом, переход индукции доказан и задача решена.

\i Давайте вести в этой задаче индукцию по $n$.\\
\textbf{База.} $n=2$, очевидна, так как единственный возможный набор для $n=2$ это $(1,1)$.\\
\textbf{Переход.} Предположим, что задача решена для всех $n\leq k$, докажем, что она также решена для $n=k+1$. Рассмотрим $a_{k+1}$. Если $a_{k+1} = a_k$, то применим предположение индукции для $a_1, \ldots a_{k-1}$ (при этом, очевидно, что $n \geqslant 4$, поэтому мы можем применять предположение индукции), после чего к первому набору добивим $a_k$, а ко второму "--- $a_{k+1}$ и получим искомое разбиение. Если $a_{k+1} \ne a_k$, заменим данный в условии набор на $a_1, a_2, \ldots a_{k-1}, \abs{a_k - a_{k+1}}$. Так как в новом наборе на одно число меньше, и $a_k + a_{k+1} \modeq{2} \abs{a_k - a_{k+1}}$ и $\abs{a_k - a_{k+1}} \leq k$ (что очевидно так как $1 \leq a_k \leq k$ и $1 \leq a_{k+1} \leq k+1$), то можно применить предположение индукции. После этого останется только вместо числа $\abs{a_k - a_{k+1}}$, добавить числа $a_k$ и $a_{k+1}$ в соответствующие наборы. Таким образом, переход доказан и задачи решена.