\subsection{Десятое ДЗ}


\i Для начала рассмотрим то, как себя ведут остатки при делении на $10$ у последовательньсти $\{a_i\}$.
\begin{gather*}
    a_0 \modeq{10} 5, \quad a_1 \modeq{10} 8, \quad a_2 \modeq{10} 7, \quad a_3 \modeq{10} 2, \quad a_4 \modeq{10} 7.
\end{gather*}
В силу того, что по последней цифре числа $a_i$ однозначно восстанавливается последняя цифра числа $a_{i+1}$ можно сказать, что остатки зацикливаются с предпериодом 2 и длиной периода 2, следовательно, при чётных $i$ больших 1 число $a_i$ будет заканчиваться на 7, а ни нечётных (так же больших 1) на 2. Следовательно $a_{2015}$ будет заканчиваться на 2.

\i Обозначим вершини нечётной степени за $A_1, A_2, A_3$ и $A_4$. Рассмотрим произвольный путь между парой вершин $A_1$ и $A_2$. Покрасим его в первый цвет и далее будем игнорировать рёбра,  входящие в него. При этом заметим, что часть, покрашенная в первый цвет обходится за 1 путь.\\
В оставшемся графе у всех вершин, степень которых была чётна, она таковой и осталась, а у $A_1$ и $A_2$ чётность изменилась (Если появились изолированные вершины, то просто будем из игнорировать). Осталось понять, что теперь в нашем графе найдётся гамельтонов путь, начинающийся в $A_3$ и заканчивающийся в $A_4$, так как граф связен и степени всех вершин, кроме $A_3$ и $A_4$ чётны, а у них нечётны. Тогда все оставшиеся рёбра мы просто покрасим во второй цвет. Что и требовалось доказать.

\i Да, дайте предъявим искомый путь.
$$\{000\} \rightarrow \{100\} \rightarrow \{110\} \rightarrow \{010\} \rightarrow \{011\} \rightarrow \{111\} \rightarrow \{101\} \rightarrow \{001\}$$
Легко убедитсья, что предъявленный путь проходит по всем вершинам и при этом ялвяется простым.

\i Давайте решать эту задачу по индукции по количеству вершин ($n$).\\
База. $n=3$ "--- очевидно.\\
Переход. Предположим, что задача решена для $n=k$, докажем её для $n=k+1$. Для этого выребем произвольную вершину $X$ и на время забудем про неё, тогда мы получим полный связный граф на $k$ вершинах, а значим для него можно применить предположение индукции, тогда в нём найдётся некоторая вершина $Y$, из которой до любой другой можно добраться не более, чем за 2 хода. Обозначим множество всех вершин, в которые ведут рёбра из $Y$ за $S$.Вернём вершину $X$. Заметим, что если $\exists A \in S: \exists$ ребро из $A$ в $X$, то для исходного графа вершина $Y$ будет являться искомой, аналогично и, если из $Y$ есть ребро в $X$. Следовательно рёбра между $X$ и множеством $S \cup \{Y\}$ ведут из $X$ (иначе мы уже нашли искомую вершину). Осталось заметить, что в таком случае вершина $X$ удовлетворяет условию, что от неё можно добраться до любобй другой не более, чем за 2 хода. В самом деле, по предположению индукции из $S \cup \{Y\}$ можно добраться до любой вершины не из этого множества за 1 ход. Тогда и из $S \cup \{X\}$ можно сделать аналогичное, а до $Y$ из $X$ мы умеем добираться за 1 ход.Таким образом мы доказали переход, а значит решили задачу для произвольного $n$.

\i Докажем это по индукции по числу вершин ($n$).\\
База. $n = 2$ "--- очевидно.\\
Переход. Предположим, что задача решена для $n=k$, докажем её для $n=k+1$. Для этого выберем произвольную вершину $A_0$ и на время выкинем её. Полученный граф по-прежнему является полным ориентированным и в нём ровно $k$ вершин, а значит к нему можно применить предположение индукции. Таким образом в нём найдётся простой ориентированныё путь по всем вершинам, обозначим его как $A_1 A_2 \ldots A_k$. Вернём $A_0$. Применив условие задачи для пары $A_0, A_1$ получим, что между ними есть ориентированное ребро. Если оно идёт от $A_0$, то просто поставим $A_0$ в начало нашего пути. Аналогично, мы сможем поставить $A_0$ в конец нашего пути, если ребро $A_kA_0$ выходит из $A_k$. Рассмотрим наименьшее такое $i$, что ребро $A_iA_0$ выходит из $A_i$ (такое обязательно найдётся так как хотя бы одно такое $i$ существует ($i=k$)). При этом $i \ne 1$. Тогда мы просто сможем построить путь $A_0 \ldots A_{i-1} A_0 A_i \ldots A_k$. Таким образом мы совершили переход, а значит решили задачу для всех $n$.

\i Заметим, что не 2-раскрашиваемость графа равносильна тому, что в нём найдётся какой-то простой цикл нечётной длины. Так как в графе всего чётное число вершин, то найдётся некоторая вершина, которая не входит в найденный цикл. Тогда если её степень не ноль, то мы просто можем удалить любое выходящее их ней ребро, и граф останется не вдудольным (мы просто не изменим рассматриваемый цикл), а значит, граф не является наименьшим. Следовательно в любом минимальном не 2-раскрашиваемом графе найдётся вершина степени 0. Что и требовалось доказать.

\i Построим ориентированный граф состояний, где состоянием будет количество камней в каждой коробочке и указание на коробочку, в которую положили последний камень. Заметим, что всего состояний конечное число. Из вершины будем проводить ребро в те, которые соответствуют состояниям, которые можно получить из того, которому соответствует вершина, из которой проводим ребро. Заметим, что по текущей расстановке однозначно определяется следующая, тогда исходящая степень каждой вершины равна 1. Докажем, что по текущей расстановке можно восстановить предыдущую. Для этого просто сделаем <<обратный ход>>: будем брать по 1 камню из каждой кучки, начиная с той, которая отмечена в состоянии (против часовой стрелки), пока не найдём пустую (возможно мы пройдём по некоторым коробка по несколько раз), а затем сложим в неё все собранные камни. Заметим, что это тот же ход, что описан в условии, но все его операции выполнены в обратном порядке. Таким образом в графе состояний ходящая степень каждой вершины также равна 1, следовательно весь граф состояний "--- набор циклов, а значит когда-то снова встретится стартовое состояние => повторится начальное расположение камней. Что и требовалось доказать.