\subsubsection{Шестнадцатое ДЗ}


\i Пусть $n = p_1^{\alpha_1}p_2^{\alpha_2}\ldots p_k^{\alpha_k}$, заметим, что при этом $(p_i, 2) = 1 => \exists d: 2^d \modeq{p_i^{\alpha_i}} 1 =>$ Можно утверждать, что существует показатель двойки по модулю $p_i^{\alpha_i}$, при это известно, что показатель не превосходит модуль. А занчит $n! \divby deg_{p_i^{\alpha_i}}2 => 2^{n!} \modeq{p_i^{\alpha_i}} 1$. А значит по КТО верно, что $2^{n!} -1 \divby n$. Что и требовалось доказать.

\i $\letus a \notdivby 11$, тогда $a^{10} \modeq{11} 1$. Аналогичное верно и для любой другой переменной из равенства. Тогда кратность 11 выражения достигается только тогда, когда каждой слагаемое делится на 11, а значит и корень 10 степени делится на 11. Таким образом, $abcdsf$ делится на $11^6$. Что и требовалось доказать.

\i Так как $(7, 17) = 1$, по МТФ верно, что $7^{7^{7^7}} \modeq{17} 7^{(7^{7^{7}})(mod 16)}$. По аналогичным приxинам верно, что $7^{7^{7^7}} \modeq{17} 7^{(7^{7^{7}})(mod 16)} \modeq{17} 7^{(7^{(7^7)(mod\ ord_{16}7)})(mod 16)}$. При этом $ord_{16} 7 = 2$. Таким образом $7^{7^{7^7}} \modeq{17} 7^7 \modeq{17} 12$.

\i
\begin{gather*}
    x^2 \modeq{200} 1;\\
    (x-1)(x+1) \modeq{200} 0;
\end{gather*}
Следовательно или $(x-1) \modeq{25} 0$, или $(x+1) \modeq{25} 0$, при этом обе скобки должны быть сравнимы с 0 по модулю 2. Переберём эти варианты отдельно.
\begin{enumerate}
    \item $(x-1) \divby 25$:
    \begin{enumerate}
        \item $x \modeq{200} 1$ "--- такой вариант нас устраивает.
        \item $x \modeq{200} 51$ "--- такой вариант нас устраивает.
        \item $x \modeq{200} 101$ "--- такой вариант нас устраивает.
        \item $x \modeq{200} 151$ "--- такой вариант нас тоже устраивает.
    \end{enumerate}
    \item $(x+1) \divby 25$:
    \begin{enumerate}
        \item $x \modeq{200} 199$ "--- этот вариант подходит.
        \item $x \modeq{200} 49$ "--- снова подходит.
        \item $x \modeq{200} 99$ "--- подходит.
        \item $x \modeq{200} 149$ "--- опять устраивает.
    \end{enumerate}
\end{enumerate}
Таким образом, $x\ (mod 200) \in \{1, 49, 51, 99, 101, 149, 151, 199\}$.

\i Да, так как $(3, 10^4)$ взаимнопросты, а значит существует такая степень тройки $d$, что $3^d \modeq{1000} 1$, а это именно то, что нам нужно.\\
*Такая степень существует, так как если взять достаточно много степенией, то какой-то остаток повторится, а значит тройка в степени разности степеней, которые соответствуют этим совпавшим остаткам даёт 1 по модулю 10000.

\i
\begin{gather*}
    x^2 \modeq{10^k} x;\\
    x(x-1) \modeq{10^k} 0.
\end{gather*}
Заметим, что из полученных скобок только одна может делиться на $2$ и тольк одна на $5$, а значит верно, что ровно одна скобка делится на $5^k$, а другоая на $2^k$. Тогда мы получим всего 4 системы на то, какие остатки $x$ даёт по модулям $2^k$ и $5^k$, при этом каждая из них будет иметь ровно 1 решение по модулю $10^k$, и все они будут различны (по КТО). Что и требовалось докажать.