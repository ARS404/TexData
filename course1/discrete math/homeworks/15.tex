\subsection{Пятнадцатое ДЗ}

\i
\begin{gather*}
    99^{1000} \modeq{100} (-1)^{1000} \modeq{100} 1.
\end{gather*}

\i
\begin{gather*}
    53x \modeq{42} 1;\\
    11x \modeq{42} 1;\\
    x \modeq{42} \frac{1}{11};\\
    x \modeq{42} 23;
\end{gather*}

\i Рассмотрим данное равенство по 2 модулям (47 и 74), получим:
\begin{gather*}
    \begin{cases}
        27x \modeq{47} 2900;\\
        -27y \modeq{74} 2900,
    \end{cases}
    \begin{cases}
        27x \modeq{47} 33;\\
        27y \modeq{74} 37,
    \end{cases}
    \begin{cases}
        x \modeq{47} 43;\\
        y \modeq{74} 68.
    \end{cases}
\end{gather*}
В силу того, что $x, y \in \ZZ$ и $x \geq 0, y \geq 0$ получим, что $74x + 47y \geq 74\cdot43 + 47\cdot68 = 6507 > 2900$. Следовательно решений нет.

\i Предположим обратное, тогда:
\begin{gather*}
    (n^2-n+1, n^2+1) = a, \quad a > 1, n^2+1 > 1;\\
    (n^2-n+1, n^2+1) = (-n, n^2+1) = (n, n^2+1) = (n, 1) = 1, \text{противоречие.}
\end{gather*}

\i
\begin{gather*}
    (3^{133}-1, 3^{101}-1) = (3^{101}-1, 3^{32}-1) = (3^{32}-1, 3^{5}-1) = (3^{5}-1, 3^{2}-1) = \\ = (3^{2}-1, 3^1-1) = (8, 2) = 2. 
\end{gather*}

\i
\begin{gather*}
    \sum_{i=1}^{p-1}\frac{1}{i} = \frac{\sum_{i=1}^{p-1}\frac{(p-1)!}{i}}{(p-1)!}.
\end{gather*}
Заметим, что $(p-1)! \not\vdots\  p$. Получается, нам остаётся только показать, что $\brackets{\sum_{i=1}^{p-1}\frac{(p-1)!}{i}} \vdots\ p$, докажем это:
\begin{gather*}
    \text{По теореме Вильсона } (p-1)! \modeq{p} 1 => \sum_{i=1}^{p-1}\frac{(p-1)!}{i} = -\sum_{i=1}^{p-1}\frac{1}{i}.
\end{gather*}
Каждому слагаемому вида $\frac{1}{i}$ сопоставим слагаемое $\frac{1}{p-1}$ (легко заметить, что все слагаемые разделятся на пары). Тогда:
\begin{gather*}
    -\sum_{i=1}^{p-1}\frac{1}{i} = -\frac{1}{2}\sum_{i=1}^{p-1}\brackets{\frac{1}{i} + \frac{1}{p-i}} = -\frac{1}{2}\sum_{i=1}^{p-1}\frac{p}{i(p-1)}.
\end{gather*}
Заметим, что в каждом слагаемом числитель по модулю $p$ сравним с нулём, а знаменатель "--- нет. Тогда и вся сумма сравнима с нулём. Деление на 2 также ничего не <<испортит>> так как $(2, p) = 1$ по условию. Что и требовалось доказать.

\i Земетим, что все числа Ферма взаимнопросты (то есть числа вида $2^{2^k} + 1$ или $f_k$). В самом деле, пусть $i > j$, тогда:
\begin{gather*}
    f_i-2 = 2^{2^i}-1 = (2^{2^{i-1}}-1)f_{i-1} = (2^{2^{i-2}}-1)f_{i-2}f_{i-1} = \ldots = (2^{2^j}-1)f_j\ldots f_{i-1}.
\end{gather*}
Следовательно $(f_i, f_j) = (f_j, 2) = 1$. Заметим, что 
\begin{gather*}
    2^{2^n} - 1 = (2^{2^{n-1}}-1)(2^{2^{n-1}}+1) = \ldots = (2^{2^0} - 1)(2^{2^0}+1)\ldots(2^{2^n}+1).
\end{gather*}
В полученном разложении все скобки, которые являются числами Ферма (которых ровно $n$) взаимнопросты, а значит каждая имеет некоторый уникальный простой делитель, а значит их не меньше $n$. Что и требовалось доказать.