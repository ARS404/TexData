\subsection{Двадцатое ДЗ}

Для начала введём такое понятие, как блок дизъюнкций и блок конъюнкций (чтобы в бальнейшем скоретить запись и упростить себе жизнь). Оба эти блока принимают несколько элементов и считают их дизъюнкцию и конъюнкцию за линию от их числа (очевидно как).\\
Так же нам понядобится блок ксор, который будет ксорить все данные своего входа поочерёдно (так же как и первые 2 блока), при этом из материалов лекции мы умеем ксорить 2 значения за константу, а значит весь блок ксор также будет работать за полиномиальное время.\\ 
Вот теперь можно приступить к задачам.\\


\i Построим нашу схему следующим образом:\\
на первом его уровне будут блоки конъюнкций, соответствующие всем возможным треугольникам в графе (всего их $\binom{n}{3}$);\\
На втором уровне будет один блок, дизъюнкций принимает результаты всех вершин второго уровня;\\
На третьем уровне будет одна вершина, которая принимает результат вершины со второго уровня и возвращает отрицание.\\
Таким образом, мы предъявили схему, сложность которой полиномиальна, так как в ней всего 3 уровня, и каждый из них работает за полиномиальное время, а значит сложность всей схемы так же полиномиальна. Осталось понять, почему она работает. Это очевидно, так как если хотя бы какой-то треугольник в графе существует, то на последней уровень придёт 1, а вернётся 0, и наоборот.

\i Просто предьявим нужныую схему (то, что она работает очевидно):\\
На первом уровне $n$ вершин, $i$-тая из которых принимает $x_i$ и $y_i$ и возвращает $x_i\ xor\ y_i$;\\
На втором уровне один блок дизъюнкций, который принимает значения со всех вершин первого уровня;\\
на третьем уровне будет отрицание второго.\\
Снова в нешей схеме конечное число уровней, каждый из которых работает за полиномиальное время, а значит и вся схема тоже.

\i  мы уже знаем, что посчитать суммы $n$ переменных мы можем за полиномиальное время, так что давайте сохраним в сумму переменных в $\roof{\log n}$ битах, и в стольки же сохраним сумму их отрицаний. Нам осталось научится сравнивать эти 2 числа за полиномиальное время и вернуть 1 тогда и только тогда, когда первое больше. Заметим, что неравенство из 2 двоичных чисел (длины $m$) можно записать следующим оброзом:
$$(a_m != b_m) \vee (a_m = b_m \wedge a_{m-1} != b_{m-1}) \vee \ldots \vee (a_m = b_m \wedge \ldots \wedge a_2 = b_2 \wedge a_1 != b_1).$$
При этом, мы хотим, чтобы неравенство возвращало 1 тогда и только тогда, когда первое число больше, так что его можно переписать следующим образом:
$$x = y <=> \neg (x \oplus y);$$
$$x != y <=> x \wedge (x \oplus y).$$
Таким образом, мы получим полиномиальную схему для сравнения чисел, что вместе со схемой сложения даст нам желаемый результат.

\i

\i Заметим, что если у нас получится посчитать выражение <<((есть 1) и (нет пары)) или (все 1)>>, то мы победим, так как именно в этих случаях нам надо вернуть 1. Первую скобку (<<есть 1>>) мы посчитаем как блок дизъюнкций от 3 данных переменных, а третью (<<все 1>>) мы найдём как блок конъюнкций от тех же переменных. При этом мы не использовали отрицания. Вторую скобку мы получим как отрицание блока дизъюнкций от результатов попарных конъюнкций входных переменны (при этом мы используем единственное отрицание). Осталось посчитать дизъюнкцию первых 2 скобок и результат передать в конъюнкцию с третей, при этом мы не воспользуемся ни одним отрицанием, а значит прдъявим искомую схему.

\i Разделим нашу схему на 2 части. Первая будет проверять свзяность графа (это мы умеем делать согласно материалу лекции (через возведение матрицы смежности с петлями в степень)), на это нам потребудется полиномаильная сложность.\\
Вторая часть схемы будет возвращать 1 только в том случае, если степени всех вершин чётны (это вместе со связностью является необходимым и достаточным условием того, что в графе найдётся эйлеров цикл).\\
В ней на первом уровне расположим блоки ксор для каждой вершины (всего $n$ штук), каждый из которых будет принимать все рёбра, смежные с данной вершиной. При этом мы каждое ребро будем принимать в 2 блока, а значит всего затратим полиномиальное время на вычисления на этом уровне.\\
На втором уровне расположим блок дизъюнкций, который примет все результаты второго блока и верёт 1 тогда и только тогда, когда для какой-то вершины ксор по рёбрам оказался равен 1 (то есть в исходном графе эта вершина имеет нечётныю степень).\\
На третьем уровне просто расположим отрицание второго.\\
Таким образом, проводя вычисления на каждом из 3 уровней мы будем тратить полиномиальное время, а значит и общая сложность тоже полиномиальна.

\i Давайте решим задачу, где надо построить схему, которая будет возвращать 1 тогда и только тогда, когда граф нельзя правильно раскрасить в 2 цвета (потом достаточно просто сделать отрицание). Заметим, что это равносильно тому, чтобы найти в графе какой-то нечётный цикл. В свою очередь, существование нечётного цикла равносильно тому, что межды какой-то парой вершин есть путь чётной длины и путь нечётной длины, так что давайте решать задачу поиска таких 2 путей.\\
Рассмотрим матрицу смежности (без петель). Нам известоно, что в при возведении её в степень $k$ (рассматриваем матрицу и её степени как матрицу над булевыми переменными) на пересечении строки и стольбца будет стоять 1 тогда и только тогда, когда между соответствующими строке и столбцу вершинам есть путь длины $k$. Зная это посчитаем все степени матрицы от первой до $n-1$-ой (каждая считается за полином, значит и на вычисление всех тоже уйдёт полином), такой степени достаточно так как длина простого пути в графе на $n$ вершинах не может превышать $n-1$. Осталось лишь убедится в том, найти ячейку, в которой одновременно стоит 1 в матрице чётной степени и в матрице нечётной степени.\\
На первом уровне расположим блоки дизъюнкций, которые будут принимать значения в выбранной ячейке только в чётных степенях и только в нечётных степенях матрицы (по 2 на каждую ячейку). Так как количество ячеек полиномаильно, как и стоимость подсчёта блока дизъюнкций, на вычисления на этом уровне уйдёт полиномиальное время.\\
На втором уровне расположим $\binom{n}{2}$ конъюнкций, которые будут принимать значения из блоков предыдущего уровня (каждый будет смотреть на то, если ли путь чётной длины между парой и путь неётной длины между этой же парой), соответственно каждый блок вернёт 1 тогда и только тогда, когда между выбранной парой вершин есть путь чётной и нечётной длины.\\
На третьем уровне расположим блок дизъюнкций, который примет результаты предыдущего блока (которых $\binom{n}{2}$, и за полиномиальное время вернёт 1 тогда и только тогда, когда нашёлся цикл нечётной длины (по вышедоказанному).\\
Осталось не звбыть сделать отрицание результата третьего уровня на четвёртом.\\
Таким образом, мы предъявили схему, которая имеет полиномиальный <<предподсчёт>> и сама работает за полиномиальное время, и при этом решает задачу.