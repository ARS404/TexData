\subsubsection{Бонус к пятому ДЗ}


Начнём с того, что обозначим множество всех бесконечных арифметических прогрессий за $\Theta$, докажем, что $\Theta$ счётно. Для этого сопоставим каждой бесконечной арифметической прогресии пару из её первого элемента и шага, очевидно, что такое сопоставление является биекцией. Тогда нам просто надо пронумеровать все возможные пары натуральных чисел. Сделаем это по следующему алгоритму:\\
на $i$-той итерации нашего алгоритма пронумеруем пары $(k, i), k \in \{1, 2, \ldots, i\}$. Тогда очевидно, что каждая пара получит номер и мы построим биекцию с $\NN$.\\
Теперь построим множества $A$ и $B$ ледующим образом:
\begin{itemize}
    \item изначально все натуральные числа лежат в $B$;
    \item на $i$-той операции будем <<удалять>> из $B$ $i$-тую арифметическую прогрессию (елси она еще не удалена, иначе пропускаем итерацию). При этом под <<удалением>> прогресии мы подразумеваем перенос хотя бы одного её члена в множество $A$. Покажем, что мы всегда можем так сделать. В любой прогресси, очевидно, что в рассматриваемой прогрессии найдётся сколь угодно большой элемент. Давайте перенесём в $A$ такой, который больше любого числа, которое уже записано в $A$ хотя бы в $239$ раз. Тогда, в $A$ гарантированно не возникнет трехчленных арифметических прогрессий, потому что наше число будет больше любого другого из $A$ больше, чем на наибольшую разность, среди пары элементов $A$ до переноса.
\end{itemize}
Таким образом у нас вышло построить множества $A$ и $B$ такие, что каждое натуральное число попало в одно из них, и при этом в $A$ не т трёхчленных арифметических прогрессий, а в $B$ нет бесконечных. Что и требовалось доказать.