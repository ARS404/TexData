\subsubsection{Восемнадцатое ДЗ}


\i Рассмотрим следующую стратегию игры за перовго игрока. Пусть оп в самом начале возьмёт 4 спички, а затем каждым своим ходом будет дополнять пердыдущий ход второго до 6 спичек. Тогда будет соблюдаться следующий инвариант "--- после хода первого количество спичек всегда кратно 6, а после хода второго "--- нет. Тогда очевидно, что первый не может проиграть, так как второй не сможет получить 0, а так как игра конечно он выиграет.

\i Докажем, что второй игрок может победить. Пусть первые 9 ходов были сделаны произвольно, докажем, что последним своим ходом второй может гарантировать себе победу не зависимо от того, какие ходы были до этого. На данный момент число выглядит как $\overline{a_1a_2\ldots a_9\bullet}$, из него можно получить слеюудщий числа: $\overline{a_1a_2\ldots a_9 0}, \overline{a_1a_2\ldots a_9 1}, \ldots$, $\overline{a_1a_2\ldots a_9 9}$. Эти числа "--- 9 последовательных чисел, а значит они дают 9 последовательных остатков по модулю 7, тогда, очевидно, что среди них найдётся 0, и второй сможет гарантировать себе победу.

\i Будем доказывать, что первый сможет гарантировать себе победу. Для этого рассмотрим стратегию, когда он каждым ходом из своего числа вычитает произвольный нечётный делитель (который всегда найдётся,  так как всегда можно вычесть 1). В таком случае будет поддерживаться инвариант, что после хода первого игрока число становится нечётным, а значит и второй игрок обязан вычитать нечётный делитель, и после его хода число станет чётным, таким образом, свой ход первый игрок всегда будет начинать с чётным числом и будет получать изз него нечётное, а значит он не может проиграть, тогда в силу конечности игры, он выиграет.

\i Дла начала скажем, что при $n$ равном 1 и 2 победу себе, очевидно, гарантирует первый, при $n$ равном 3 "--- второй, так что далее будем считать, что $n > 3$. Также для удобства будет считать, что минусы расположены в вершинах привильного $n$-угольника, вписанного в окружность $\Omega$ с центром $O$. Предъявим выигрышную стратегию для второго игрока:\\
\textit{Если $n$ чётно}. В таком случае второму игроку достаточно отражать хоы первого относительно $O$ (центральная симметрия). Тогда в силу симметрии, если ход нашёлся у первого игрока, то он нашёлся и у второго, а значит второй не может проиграть. Так как игра конечно, то второй выиграет.\\
\textit{Если $n$ нечётно}. Рассмотрим ситуацию после хода первого игрока:
\begin{enumerate}
    \item Изменён они знак, в таком случае изменим 2 знака на противоположной стороне;
    \item Изменеы два знака на одной стороне нашего $n$-угольника, в таком случае изменим знак в противоположной ему вершине.
\end{enumerate}
В обоих случаях изменены 3 знака, 2 из которых лежат на одной стороне, а третий в противоположной к этой стороне вершине. Тогда проведём диаметр $\Omega$ (далее $l$) через эту вершину и середину стороны. Заметим, что $l$ разбил наш $n$-угольник на две симметричные относительно него части, при это никакой ход не может изменить знаки сразу в обеих частях. После этого второму достаточно отражать ходы первого относительно $l$ (осевая симметрия). Тогда, если ход нашёлся у первого игрока, то симметричный ход возможен и для второго, а значит он ен может проиграть, следовательно выиграет.\\
Таким образом, выигрышная стратегия у перового игрока есть только при $n$ равном 1 или 2 (достаточно просто сразу изменить все знаки).

\i Вспомним тот известный факт, что любую не полную триангуляцию $n$-угольника (то есть такую, в которой проведёно меньше $n-3$ диагоналей, которые не пересекаются) можно дополнить до полной (то есть такой, в которой ровно $n-3$ диагонали). Это можно доказать полной индукцией по количеству вершин. База очевидна, в переходе можо рассмотреть произвольную диагональ (если такой нет, то её надо провести), она разобъёт наш многоугольник на 2 с меньшим числом вершин, и для обоих надо применить предположение индукции. Таким образом, мы поняли, чт оперед нами <<игра-шутка>> так как всегда будет проведено ровно $n-3$ диагонали, а значит выигрышная стратегия у первого игрока есть только при чётных $n$, при нечётных она есть у второго. В обоих случаях она звичит примерно как <<делай любой возможный ход>> (по вышедоказанному это работает).

\i Давайте доказывать, что при любом $n$ первый игрок может победить. Для этого предъявим выигрышную стратегию:\\
Пусть до начала игры он выберет произвольную вершину $S$. После этого он будет каждым своим ходом выбирать некоторую вершину $X$, которой ещё нельзя добраться до $S$ и красить произвольное ребро из $X$ в одну из вершин множества тех, до которых из $S$ можно добраться (далее $R$).\\
Докажем, что стратегия работает. Пусть уже прошло $m$ итераций, тогда в $R$ ровно $m+1$ вершина. При этом второй игрок сделал $m$ ходов, а значит среди рёбер любой вершины $X$, от которой ещё нет пути до $S$ закрашено не более $m$ рёбер, а значит найдётся какое-то не покрашенное ребро до вершины из $R$, так как таких рёбер ровно $m+1$, а значит ход по предложенной стратегии всегда возможен. Также легко понять, что через $n-1$ итерацию мы сможем добраться из $S$ до всех вершин, а значит мы получим связный граф на $n$ вершинах и рёбрах первого цвета, а значит победим.\\
*Осталось заметить, что за это же время второй не сможет нарисовать связный граф. Это очевидно, так как на это ему понадобится не менее, чем $n-1$ ход, однако за то время, которое понядобится на победу первому игроку, он успеет провести только $n-2$ ребра, а значит не получит связного графа.
