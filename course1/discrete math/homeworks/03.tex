\subsection{Третье ДЗ}

\i Мы хотим доказать, что $(A \cup B) \backslash C = (A \backslash C) \cup (B \backslash C)$. Рассмотрим это равенство в случае произвольного элеманта, тогда $a = 1$, если рассматриваемый элемент принадлежит $A$, аналогично определим $b$ и $c$. Тогда мы хотим доказать, что 
\begin{gather*}
    (a \vee b) \wedge \neg c = (a \wedge \neg c) \vee (b \wedge \neg c);\\
    (a \wedge \neg c) \vee (b \wedge \neg c) = (a \wedge \neg c) \vee (b \wedge \neg c).
\end{gather*}
Такой переход возможен, так как в слечае логических операция корректно раскрываётся скобки для логических <<и>> и <<или>> (доказывали на парах). Что и требовалось доказать.

\i Рассмотрим все возможные составы по 9 человек и по 1 человеку. И тех и тех ровно $\binom{10}{1}$. Любая группа из 9 человек должны сыграть ровно 1 раз, при этом она не может играть ни с кем, кроме своего дополнения, а по скольку и оно также сыграло ровно 1 игру, то все составы по 9 и по 1 человеку разбились на пары, в объединении дающие полное множество челенов клуба. После этого рассмотрим составы по 8 и по 2 человека. Их снова поровну - $\binom{10}{2}$. Каждый набор из 8 человек должен сыграть с набором из одного человека или из 2, но все наборы по 1 уже кончились, а значит все группы по 8 человек играют с составами по 2, но они могут сыграть только со своим дополнением, а значит они снова разбились на пары. Аналогичное рассуждение пожно повторить для групп по 7 и 3 и групп по 6 и 4. После этого оснаются только группы по 5, которых $\binom{10}{5}$ - чётное число. Каждый состав из 5 человек может сыграть только с другим составом из 5 человек, а это может быть только тогда, когда он играет со своим дополнением. Тогда в каждой игре будут участвовать 2 команды, такие, что в объединение они дают полный набор участников клуба. Что и требовалось доказать.

\i Давайте заменим ксор на слжение, а конъюнкцию на минимум и будем рассматривать данное выражение не как функцию надо будевыми переменными, а как выражение от целы чисел. В таком случае, нас интересуем чётность его значения. Заметим, что выражение является симметрическим, а значит нас только интересует количествоединиц и нулей. Пусть единиц $k \ (k > 0)$. Также, очевидно, что занулятся все скобки (в том числе и из одного элемента), в которых есть хотя бы один 0. Тогда чётность значения выражения совпадает с чётность количества скобок, в которых только 1. Легко понять, что их будет ровно $\binom{k}{1} + \binom{k}{3} + \ldots$ "--- из прошлых листочков, мы знаем, что это в точности $2^{k-1}$, а значит, если единиц больше одной, то значение функции из условия будет равно 0. Если же единиц 0, то оно тоже очевидно ровно 0.

\i Мы легко можем посчитать количество возможных перестановок из 5 чисел, их ровно $5!$. Давайте посчитаем количество <<плохих>> чисел по формуле включений-исключений:
\begin{center}
    \#с 1 в начале или в середине + \#с 4 и 5 по соседству - \#с 1 в начале или середине и с 4 и 5 по соседству.
\end{center}
перестановок с 1 в начале или середине ровно $4! \times 2$, и с 4 и 5 по соседству столько же (в первом случае строим последовательность без 1 и 2 способамы выбираем, куда ее поставить, во втором случае так же, но без 4). Если же выполняются оба условия, то забудем про 1 и 4, получим $3!$ возможных последовательностей, полсе этого 2 способами вернём 1 и двумя "--- 4, получим $3! \times 4$, однако при этом мы посчитаем также и числа вида $\bullet415\bullet$ и $\bullet514\bullet$ (всего 4 лишних последовательности). Тогда ответом будет $5! - 4! \times 4 + 3! \times 4 - 4 = 44$.

\i Давайте доказывать, что мы не сможем получить константу 0, используя только систему связок $\{\vee, \rightarrow \}$. Будем доказывать это индукцией по длине выражения (количество связок "--- $k$). База $k=0$ "--- очевидна.\\
Переход. Предположим, что константу 0 не получается получить для выражения длины не больше $k$, докажем, что выражением длины $k+1$ её тоже не получится получить. Для этого рассмотрим последнее действие в произвольном выражении длины $k+1$. Если это дизъюнкция, то она даст константу 0 только, если выражения слеваи справа также равны константе 0, но оба по длине не превосходят $k$, а значит, если последняя операция это дизъюнкция, то мы выполнили переход. Если же последняя связка "--- импликация, то для того, чтобы получилась константа 0 необходимо, чтобы справа от импликации была константа 0, что так же невозможно по предположению индукции. Что и требовалось доказать.

\i Пусть $k$ "--- наименьшее натуральное число, такое, что если среди $n$ переменных хотя бы $k$ равны $1$, то функция $MAJ(x_1, \ldots, x_n) = 1$, то есть, $k = \frac{n}{2} + 1$, если $n$ чётно, и $k = \frac{n+1}{2}$, если $n$ нечётно. Давайте докажем, то минимальный зармер ДНФ, задающей функция $MAJ(x_1, \ldots, x_n)$ (далее просто функция) равен $k\binom{n}{k}$.\\
Для начала приведём пример. Рассмотрим ДНФ, в которой ровно $\binom{n}{k}$ скобок, и каждая содержит конъюнкцию уникального набора из $k$ переменных. Очевидно, что у такой ДНФ ровно такая длина, какая нам и нужна. При этом, в любой подстановке переменных, при которой функция равно 1, присутствует не меньше, чем $k$ единиц, а значит в нашей формуле найдётся скобка, которая будет равна 1, а значит и вся ДНФ также будет равна 1. С другой стороны, если функция равна 0, то среди $n$ переменных строго меньше, чем $k$ единиц, а значит никакая скобка не примет значение 1, следовательно всся формула будет равна 0. Таким образом, мы привели пример, осталось доказать, что ДНФ с меньшим числом переменных быть не может.\\
Для этого давайте рассмотрим произвольную ДНФ. Для того, чтобы она принимала истинное значение на каждом наборе из $k$ переменных (переменные из набора равны 1, остальные "--- 0), в ней должна быть скобка, принимающая значение 1 на этом наборе. При этом, для каждой пары наборов эти скобки различны (что очевидно), а длина каждой не меньше, чем $k$, так как если найдётся скобка длины меньше $k$, то мы можем приравнять её к 1, когда среди всех $n$ переменных единиц меньше $k$. Тогда получим, что минимальная длина ДНФ задающей функцию равна $k\binom{n}{k}$. Что и требовалось доказать.