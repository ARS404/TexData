\subsection{Четвёртое ДЗ}


\i $R \comp R = \{(a, c) \in \RR \times \RR: \exists b \in \RR, \frac{a}{b} > 0 \And \frac{b}{c} > 0\}.$ 
\begin{enumerate}
    \item $\frac{a}{b} > 0 \And \frac{b}{c} > 0 => \frac{a}{c} > 0 => R \comp R \subseteq R$,
    \item $\letus \frac{a}{c} > 0 => a \ne 0 \And b \ne 0 => \frac{a}{a} = 1 \And \frac{a}{c} > 0 => \exists b \in \RR: \frac{a}{b} > 0 \And \frac{b}{c} > 0 \ (b = a) => R \subseteq R \comp R.$
\end{enumerate}
Таким образом $R \comp R = R$.

\i Давайте обозначим $(A_1 \backslash A_2) \times (B_1 \backslash B_2)$ за $X$, а $(A_1 \times B_1) \backslash (A_2 \times B_2)$ за $Y$.
\begin{enumerate}
    \item $\letus (a, b) \in X <=> a \in A_1 \backslash A_2 \And b \in B_1 \backslash B_2;$
    \item То есть $a \in A_1 \And a \notin A_2$, аналогичное верно и для $b$;
    \item Таким образом $(a, b) \in Y => X \subseteq Y$.
\end{enumerate}
Осталось разобраться, когда бывает равенство.\\
Заметим, что если $A_1 = A_2$ или $B_1 = B_2$, то оба множества $X$ и $Y$ пустые, а значит достигается равенство, так как $X = Y =$ \o. Так что давайте считать, что $A_1 \ne A_2$ и $B_1 \ne B_2$.\\
$\letus A_1 \cap A_2 \ne$ \o. Тогда давайте рассмотрим $(a, b): a \in A_1 \cap A_2 \And b \in (B_1 \backslash B_2)$. Так как $a \notin A_1 \backslash A_2, \ (a, b) \notin X$. При этом вполне очевидно, что $(a, b) \in Y$. Аналогичное рассуджение верно и для пары $B_1, B_2$. Тогда, если было достигнуто равенство, то $A_1 \cap A_2 =$ \o $\And B_1 \cap B_2 = \o$. Также, достаточно легко убедиться, что этого уловия достаточно, длч того, чтобы $X$ совпадало $Y$, так как оба будут просто равны $A_1 \times B_1$.\\
Таким образом, равество $X$ и $Y$ достиается только, если $A_1 = A_2$ или $B_1 = B_2$ или же для какой-то из пар $A_1, A_2$ или $B_1, B_2$ пересечение пусто.

\i Давайте доказывать, что $f(A \Delta B) \supseteq f(A) \Delta f(B)$.\\
$\letus x \in f(A) \Delta f(B)$. Тогда рассмотрим какой-то $y = f^{-1}(x)$. Очевидно, что, если $y \in A \cap B$, то $f(y) = x \in f(A) \And f(y) = x \in f(B) => f(y) = x \notin f(A) \Delta f(B) \text{ что неверно }=> y \notin A \cap B$, но при этом, очевидно, $y \in A \cup B => y \in A \Delta B$.\\ 
Таким образом, мы доказали, что $f(A \Delta B) \supseteq f(A) \Delta f(B)$, однако, возможно, что $f(A \Delta B) = f(A) \Delta f(B)$, покажем, что такое не всегда верно, для этого просто приведём пример:\\
$\letus f: \{00, 01, 10\} \mapsto \{kek\}, A = \{00, 01\}, B = \{01, 10\}$, тогда $f(A \Delta B) = f(\{00, 10\}) = \{kek\}$, а $f(A) \Delta f(B) = \{kek\} \Delta \{kek\} =$ \o.

\i Давайте доказывать, что $f^{-1}(A \Delta B) \equiv f^{-1}(A) \Delta f^{-1}(B)$.\\
$\letus x \in f^{-1}(A \Delta B) => f(x) \in A \Delta B => f(x) \in A \And f(x) \notin B$ или $f(x) \notin A \And f(x) \in B$. В силу симметрии можно рассмотреть только первый вариант.\\
$f^{-1}(A) = \{a \in Y: f(a) \in A\}$, аналогично для $f^{-1}(B)$. Тогда из нашего предположения следует, что $x \in f^{-1}(A) \Delta f^{-1}(B)$. Таким образом, $f^{-1}(A \Delta B) \subseteq f^{-1}(A) \Delta f^{-1}(B)$.\\
Теперь покажем, что $f^{-1}(A \Delta B) \supseteq f^{-1}(A) \Delta f^{-1}(B)$. $\letus x \in f^{-1}(A) \Delta f^{-1}(B)$. Без ограничения общности будем считать, что $x \in f^{-1}(A) \And x \notin f^{-1}(B) => f(x) \in A \Delta B => x \in f^{-1}(A \Delta B)$, так как $f^{-1}(A \Delta B) = \{c \in Y: f(c) \in A \Delta B\}$.\\
Тогда $f^{-1}(A \Delta B) \subseteq f^{-1}(A) \Delta f^{-1}(B)\And f^{-1}(A \Delta B) \supseteq f^{-1}(A) \Delta f^{-1}(B) => f^{-1}(A \Delta B) \equiv f^{-1}(A) \Delta f^{-1}(B)$, что и требовалось доказать.

\i Давайте доказывать, что ответ составляет ровно $\binom{n+m-1}{n}$. Для этого обозначим количество соответствующих функция за $\Phi_n^m$, где $n$ "--- мощность множества, на котором определена функция, и $m$ "--- мощность области значений. Тогда мы хотим доказать, что $\Phi_n^m = \binom{n+m-1}{n}$. Для этого докажем, что $\Phi_1^m = m = \binom{m}{1}$. В самом деле, если область определения функции составляет всего один элемент, то вариатов функции ровно $m$, так как все возможные значения подходят. Теперь, если мы докажем, что $\Phi_n^m = \Phi_{n-1}^m + \Phi_{n-1}^{m-1}$, то получим желаемое, так как рекуррентные формулы для $\binom{n+m-1}{n}$ и $\Phi^m_n$ совпадут, а также совпадут начальные значения.\\
Давайте рассмотрим произвольную мнотонную функцию $f: \{1, 2, \ldots, n\} \mapsto \{1, 2, \ldots, m\}$. Если $f(n) = m$, то сопоставим ей функцию $g(x): \{1, 2, \ldots, n-1\} \mapsto \{1, 2, \ldots, m-1\}$, а если же $f(n) \ne m$, то сопоставим $h(x): \{1, 2, \ldots, n-1\} \mapsto \{1, 2, \ldots, m\}$. Очевидно, что $g(x)$ и $h(x)$ попарно различны, и притом монотонны, а значит каждая посчитана в $\Phi_{n-1}^{m-1}$ и $\Phi_{n-1}^{m}$, соответственно. Так же для каждой возможной функции $h(x)$ и $g(x)$ соответствует ровно одна функция $f(x)$, что очевидно, а значит, выполнено рекуррентное соотношение $\Phi_n^m = \Phi_{n-1}^m + \Phi_{n-1}^{m-1}$, а это ровно то, что мы и хотели доказать.\\
Таким образм, для фиксированных $n$ и $m$ число монотонных функция из $\{1, 2, \ldots, n\}$ в $\{1, 2, \ldots, m\}$ составляет ровно $\binom{n+m-1}{n}$. 

\i Давайте рассмотрим функцию $f(x)$, которая натуральное число вида $4k+1$ переведёт в $4(k+1)$, число вида $4k$ "--- в $4(k-1)+2$, $4k+2$ "--- $4k+3$, $4k+3$ "--- $4k+1$. Тогда легко будедиться, что $f(f(x))$ переведёт числа по следующему правилу:
\begin{itemize}
    \item $4k+1$ в $4 + 2$,
    \item $4k+2$ в $4k+1$,
    \item $4k+3$ в $4(k+1)$,
    \item $4k$ в $4(k-1)+3$.
\end{itemize}
При этом легко убедиться, что $f(x)$ является биекцией, а $f(f(x))$ меняет чётность числа, значит такая $f(x)$ подходит. Что и требовалось доказать.

\i Вместо того, чтобы предъявить такую функцию в явном виде, давайте предьявим алгоритм её построение и покажем, что он подходит:
\begin{enumerate}
    \item $f(1) = 1, f(2) = 3, f(3) = 4$;
    \item Далее будем рассматривать все числа в порядке позрастания (начиная с 4) и <<назначать>> нашей функции $f$ значения в этих числах:
    \begin{enumerate}
        \item в очередном числе $x$ наша функция еще не определена, тогда скажем, что $f(x) = k, (x-1)^2 < k < x^2$. Заметим, что такое число точно найдётся, так как, до этого мы назначили только $x-1$ число, а в данном диапозоне их строго больше. После этого назначим $f(k) = x^2$,
        \item в очередном числе наша функция уже определена. Тогда обозначим это число за $x, f(x) = k$. При этом мы знаем, что $k = y^2$, для какого-то $y < x$ (по построению), а занчит просто <<назначим>> $f(y^2) = x^2$.
    \end{enumerate}
\end{enumerate}
Заметим, что в приведённом алгоритме, для каждого $x$ значение $f(x)$ определяется одноззначно (то есть мы получили именно функцию, а не не-пойми-что), и при том $f(f(x)) = x^2$ по построению. Следовательно такая функция удовлетворяет условию. Что и требовалось доказать.