\subsection{Бонус ко второму ДЗ}

\textbf{Бонусная задача 2.}\\
Давайте доказывать, что константа $C=4$ удовлетворяет условию. Доказывать это мы собираемся по индукции по $n$.\\
\textbf{База.} $n=2$ очевидно, так как число единиц не превосходит 3.\\
\textbf{Переход.} Предположим, что задача решена для $n=k-1$, докажем, что оно также решена для $n=k$. Для начала скажем, что мы будем называть плохими прмоугольниками такие 4 клетки, что они стоят в пересечениях 2 столбцов и 2 строк, и в них стоят единицы. Тогда мы ходим доказать, что, если в таблиценет плохих прямоугольников, то всего в таблице не более, чем $4k\sqrt{k}$ единиц.\\
Предположим, что в таблице нашлись такие строка и столбец, что в них в сумме стоит не более $4\sqrt{k}$ единиц. Тогда давайте выкинем из нашей таблицы эти строку и столбец, при этом у нас не возникнет плохих прямоугольникови получатся таблица $(k-1) \times (k-1)$, для которой верно наше предположение индукции, а значит, после выкидывания у нас опсталось не более $4(k-1)\sqrt{k-1}$ единиц, а тогда в исходной таблице их было не более, чем $4((k-1)\sqrt{k-1} + \sqrt{k})$, что не превосходит $4k\sqrt{k}$, таким образом, переход доказан.\\
Если наше предположение оказалось неверно, и в любой строке и любом столбце в сумме стоит больше $4\sqrt{k}$ единиц, то рассторим наименьшую сумму в одной линии (без ограницения общности, будем считать, что это строка). Если эта сумма больше $2\sqrt{k}$, то во всех столбцах и одной строке количество единиц тоже больше $2\sqrt{k}$. Если же рассматриваемая сумма не превосходит $2\sqrt{k}$, то во всех столбцах строго больше $2\sqrt{k}$ единиц, и по принципу Дирихле найдется строка, в которой их тоже больше $2\sqrt{k}$. В обоих случаях назовём эту строку $S$, а множество столбцов, в пересецении которых с $S$ назовём $A$, при этом мы знаем, что $\abs{A} > \sqrt{k}$. В каждом столбце из $A$ стоит больше $2\sqrt{k} - 1$ единица (не считая ту, что в пересечении с $S$). Тогда всего единиц будет строго больше, чем $4\sqrt{n}(\sqrt{n}-1)$, и при этом по принципу Дирихе какие-то 2 окажутся в одной строке (отличной от $S$, которых всего $n-1$). Тогда найдётся плохой прямоугольник, образованный найденными 2 единицами и 2 единицами на пересечении выбранных столбцов с $S$, а значит наше предположение неверно, и переход выполнен.\\
Что и требовалось доказать.