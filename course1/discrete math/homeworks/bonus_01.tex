\subsubsection{Бонус к первому ДЗ}


\textbf{Бонусная задача 1.}\\
Давайте доказывать задачу индукцией по $n$.\\
\textbf{База.} $n=1$, очевидно, что через 1 ход чёрных клеток не останется.\\
\textbf{Переход.} Пердположим, что задача реена для всех $n \leq k$, решим ее для $n=k+1$. Тогда введём оси координат таким образом, чтобы обе координаты каждой клетки из множества тех, что изначально были чёрными (далее $S$), были положительными, и нашлось по одной (возможно больше) клетке из $S$, таких, чтобы у первой $x=1$, а у второй $y=1$. Пусть $S_1 = \{a: a \in S \text{ таких, что их координата по $x$ больше 1}\}$. Тогда $\abs{S_1} \leq k$, и при этом все клетки, с $x = 1$ не могут оказать влияние на покраску клеток у которых $x > 1$. Тогда по предположению индукции (к $S_1$) через не более чем $k$ ходов все покрашенные клетки имеют кординату по $x \leq 1$. С другой стороны, у нас не могло возникнуть покрашенных клеток с $x < 1$. (в самом деле, изначально их не было, тогда можно рассмотреть первую из них, но для её покраски в черный чвет необходима клетка над ней, а тогда рассматриваемая нами клетка не первая, противоречие, значит таких клеток быть не могло). А значит, что после не более, чем $k$ ходов, все покрашенные клетки имеют координату $x = 1$, аналогично доказывается, что их $y = 1$, а тогда у нас осталось не более одной клетки, которая исчезнат за 1 ход, слоедовательно чёрных клеток не останется не позднее, чем через $k+1$ ход, что и требовалось доказать. Таким образом, переход индукции доказан, и задача решена.

