\subsection{Первое БДЗ}

\i(8)\\
Это утверждение верно, так как в качестве $x, y$ можно взять пару $(\sqrt{2}, \sqrt{2})$, и тогда не найдётся такого рацонального $r: r \in [\sqrt{2}, \sqrt{2}]$.\\
Отрицанием к данному выражению будет являться $\forall x \in \mathds{R} \ \forall y \in \mathds{R} \ \exists r \in \mathds{Q}: r \in [x, y].$

\i(6)\\
Давайте предположим обратное, тогда число $\sqrt{7+4\sqrt{3}} - 2\sqrt{3}$ представимо в виде $\frac{p}{q}: p, q \in \mathds{Z}$. Докажем, что такое невозможно:
\begin{gather*}
    \sqrt{7+4\sqrt{3}} - 2\sqrt{3} = \frac{p}{q};\\
    \sqrt{(2+\sqrt{3})^2} - 2\sqrt{3} = \frac{p}{q};\\
    7 - 4\sqrt{3} = \frac{p^2}{q^2};\\
    \sqrt{3} = \frac{7q^2-p^2}{4q^2}.\\
\end{gather*}
При этом мы знаем, что корень из 3 иррационален (доказывали на парах, поэтому в я не стану приводить доказательство этого факта), а вот число справа всегда рационально, а значит наше предположение неверно, таким образом, мы доказали, что число $\sqrt{7+4\sqrt{3}} - 2\sqrt{3}$ иррационально.

\i (1)\\
$X = \{\sum\limits_{k=1}^n\frac{1}{2^k}: n \in \mathds{N}\}$. Для начала начала скажем, что последовательность $\{x_i:\sum\limits_{k=1}^i\frac{1}{2^k}\}$ монотонно возрастает (что очевидно, так как все слагаемые строго больше нуля). При этом каждый член этой последовательности представляет из себя сумму геометрической прогрессии с множителем меньше 1, а начит мы можем посчитать сумму всей прогресси: $\sum\limits_{k=1}^{\infty}\frac{1}{2^k} = \frac{\frac{1}{2}}{1-\frac{1}{2}} = 1.$ При этом, множество членов последовательности $\{x_i\}$ совпадает с $X$, а значит $supX = 1$. Так же, мы знаем, что члены $\{x_i\}$ монотонно возрастают, следовательно $inf\{x_i\} = \{x_1\} = \frac{1}{2}$. Тогда, по выше доказанному, $infX = inf\{x_i\} = \frac{1}{2}$. 

\i (1)\\
$x_n = \sqrt[3]{n^3 + n^{\frac{3}{2}}} - \sqrt[3]{n^3 - n^{\frac{3}{2}}}$. Давайте доказывать, что $\lim\limits_{n\to\infty}x_n = 0$. Это равносильно тому, что $\lim\limits_{n\to\infty}\frac{\sqrt[3]{n^3+n^{\frac{3}{2}}}}{\sqrt[3]{n^3-n^{\frac{3}{2}}}} = 1$ (так можно сказать, так как, начиная с $n=2$ знаменател полученной дроби всегда строго больше $0$). Так как это выражение можно возвести в куб, получим, что утверждение о том, что предел $x_n$ равне $0$ равносиьлно тому, что $\lim\limits_{n\to\infty}\frac{n^3+n^{\frac{3}{2}}}{n^3-n^{\frac{3}{2}}} = 1.$ Докажем последнее утверждение:
\begin{gather*}
    \lim\limits_{n\to\infty}\frac{n^3+n^{\frac{3}{2}}}{n^3-n^{\frac{3}{2}}} = 1;\\
    \lim\limits_{n\to\infty}(1+\frac{2n^{\frac{3}{2}}}{n^3-n^{\frac{3}{2}}}) = 1;\\
    \lim\limits_{n\to\infty}\frac{2n^{\frac{3}{2}}}{n^3-n^{\frac{3}{2}}} = 0;\\
    \lim\limits_{n\to\infty}\frac{2}{n^{\frac{3}{2}} - 1} = 0.
\end{gather*}
Посделнее тожедество является очевидным, так как числитель дроби всегда константа, а при $n>100$ знаменатель строго больше $n$. Таким образом, мы доказали, что предел последовательности $\{x_n\} = 0$. При этом легко убедиться, что все члены нашей поселовательности больше нуля (так как, очевидно, $\sqrt[3]{n^3 + n^{\frac{3}{2}}} > \sqrt[3]{n^3 - n^{\frac{3}{2}}}$). Тогда $inf\{x_n\} = 0$. По выше доказанному мы знаем, что $\{x_n\}$ убывает (так как убывает $\{x_n^3\}$), а значит $sup\{x_n\} = x_1 = \sqrt[3]{2}$.

\i (8)\\
$x_n = \frac{-6n - 1}{4-3n} = \frac{6n+1}{3n-4} = 2 + \frac{9}{3n-4}$. Очевидно, что $\lim\limits_{n\to\infty}\frac{9}{3n-4} = 0$, так как для любого $\epsilon > 0 \ \exists M \ \forall m > M: \abs{\frac{9}{3m-4}} < \epsilon$ (для этого достаточно взять $M > \frac{9-4\epsilon}{3\epsilon}$). Тогда $\lim\limits_{n\to\infty}(2+\frac{9}{3n-4}) = \lim\limits_{n\to\infty}2 + \lim\limits_{n\to\infty}\frac{9}{3n-4} = 2 + 0 = 2$. Осталось найти минимальное $n_0$ такое, что $n > n_0: x_n < 2 + \frac{1}{1000}$. По выше доказанному мы знаем, что $n_0 > \frac{9 - \frac{4}{1000}}{\frac{3}{1000}} => n_0 > 2998\frac{2}{3} => n_0 = 2999$. 