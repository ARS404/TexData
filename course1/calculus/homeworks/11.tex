\subsection{Одиннадцатое БДЗ}

\setcounter{iii}{25}


\i \textbf{9.}
\begin{gather*}
    f(x) = 1 - x;\\
    g(x) = \sqrt{1-x^2};\\
    [a, b] = [0, 1].
\end{gather*}
Для начала посчитаем расстояние между $f$ и $g$ в $L^\infty$. Это будет 
\begin{gather*}
    \sup_{[a,b]}\abs{f(x) - g(x)} = \sqrt{2} - 1.
\end{gather*}
Теперь разберёмся с $L^2$, расстояние будет иметь вид
\begin{gather*}
    \sqrt{\int_0^1|f(x) - g(x)|^2dx} = \sqrt{\frac{5}{3} - \frac{\pi}{2}}.
\end{gather*}


\i \textbf{7.}\\
\begin{gather*}
    f(x) = \abs{x};\\
    g(x) = \frac{\pi}{2} - \frac{4}{\pi}\cos{x} - \frac{4}{9\pi}\cos{3x};\\
    [a, b] = [-\pi, \pi];\\
    \rho(f(x), g(x)) = \sqrt{\int_{-\pi}^{\pi} |f(x) - g(x)|^2dx}.
\end{gather*}
Сперва посчитаем нужный интеграл
\begin{gather*}
    \int_{-\pi}^{\pi} |f(x) - g(x)|^2dx = \int_{-\pi}^{\pi} (f(x) - g(x))^2dx = \\ = 
    \int_{-\pi}^{\pi} (\abs{x} - \frac{\pi}{2} + \frac{4}{\pi}\cos{x} + \frac{4}{9\pi}\cos{3x})^2dx.
\end{gather*}
Он и так выглядит жутко, а тут ещё и модуль. Давайте посчитаем его отдельно на положительных $x$ и на отрицательных.
\begin{gather*}
    \int_{0}^{\pi} (x - \frac{\pi}{2} + \frac{4}{\pi}\cos{x} + \frac{4}{9\pi}\cos{3x})^2dx = \\ =
    \int_0^\pi (x^2 + \frac{\pi^2}{4} + \frac{16\cos(x)^2}{\pi^2} + \frac{16\cos(3x)^2}{81\pi} - x\pi + \frac{8x\cos(x)}{\pi} + \frac{8x\cos(3x)}{9\pi} - 4\cos(x) + \\ + \frac{32\cos(x)\cos(3x)}{9\pi^2} - \frac{4\cos(3x)}{9})dx;
    \intertext{сейчас всю страшную жесть разобём на отдельные интегралы и посчитаем каждый}
    \int_0^\pi x^2dx - \frac{\pi^2}{4}\int_0^\pi 1dx + \frac{16}{\pi^2}\int_0^\pi \cos(x)^2dx + \frac{16}{81\pi}\int_0^\pi \cos(3x)^2dx - \pi\int_0^\pi xdx + \\ + \frac{8}{\pi}\int_0^\pi x\cos(x)dx + \frac{8}{9\pi}\int_0^\pi \cos(3x)dx - 4\int_0^\pi \cos(x)dx + \frac{32}{9\pi^2}\int_0^\pi \cos(x)\cos(3x)dx - \\ - \frac{4}{9}\int_0^\pi \cos(3x)dx = \\ =
    \int_0^\pi x^2dx - \frac{\pi^2}{4}\int_0^\pi 1dx + \frac{16}{\pi^2}\int_0^\pi \cos(x)^2dx + \frac{8}{81\pi} - \pi\int_0^\pi xdx + \\ + \frac{8}{\pi}\int_0^\pi x\cos(x)dx + \frac{8}{9\pi}\int_0^\pi \cos(3x)dx - 4\int_0^\pi \cos(x)dx + \frac{32}{9\pi^2}\int_0^\pi \cos(x)\cos(3x)dx - \\ - \frac{4}{9}\int_0^\pi \cos(3x)dx = \\ =
    \int_0^\pi x^2dx - \frac{\pi^2}{4}\int_0^\pi 1dx + \frac{16}{\pi^2}\int_0^\pi \cos(x)^2dx + \frac{8}{81\pi} - \pi\int_0^\pi xdx + \\ + \frac{8}{\pi}\int_0^\pi x\cos(x)dx + \frac{8}{9\pi}\int_0^\pi \cos(3x)dx - 4\int_0^\pi \cos(x)dx + \frac{16}{9\pi^2}\int_0^\pi (\cos(2x) + \cos(4x))dx - \\ - \frac{4}{9}\int_0^\pi \cos(3x)dx = \\ =
    \frac{8}{81\pi} - \frac{16}{81\pi} + \int_0^\pi x^2dx - \frac{\pi^2}{4}\int_0^\pi 1dx + \frac{16}{\pi^2}\int_0^\pi \cos(x)^2dx - \pi\int_0^\pi xdx + \\ + \frac{8}{\pi}\int_0^\pi x\cos(x)dx + \frac{8}{9\pi}\int_0^\pi \cos(3x)dx - 4\int_0^\pi \cos(x)dx - \\ - \frac{4}{9}\int_0^\pi \cos(3x)dx = \\ = 
    \frac{-8}{81\pi} + \int_0^\pi x^2dx - \frac{\pi^2}{4}\int_0^\pi 1dx + \frac{16}{\pi^2}\int_0^\pi \cos(x)^2dx - \pi\int_0^\pi xdx + \\ + \frac{8}{\pi}\int_0^\pi x\cos(x)dx + \frac{8}{9\pi}\int_0^\pi \cos(3x)dx - 4\int_0^\pi \cos(x)dx - 0 = \\ = 
    \frac{-8}{81\pi} + \int_0^\pi x^2dx - \brackets{\frac{\pi^2}{4} + \frac{8}{\pi^2}}\int_0^\pi 1dx + \frac{8}{\pi^2}\int_0^\pi \cos(2x)dx - \pi\int_0^\pi xdx + \\ + \frac{8}{\pi}\int_0^\pi x\cos(x)dx + \frac{8}{9\pi}\int_0^\pi \cos(3x)dx - 4\int_0^\pi \cos(x)dx = \\ = 
    \frac{-8}{81\pi} + \int_0^\pi x^2dx + \frac{32+\pi^4}{4\pi} - \pi\int_0^\pi xdx + \\ + \frac{8}{\pi}\int_0^\pi x\cos(x)dx + \frac{8}{9\pi}\int_0^\pi \cos(3x)dx - 4\int_0^\pi \cos(x)dx = \\ = 
    \frac{-8}{81\pi} + \int_0^\pi x^2dx + \frac{32+\pi^4}{4\pi} - \pi\int_0^\pi xdx + \\ + \frac{8}{\pi}\int_0^\pi x\cos(x)dx + \frac{8}{9\pi}\int_0^\pi \cos(3x)dx - 4\int_0^\pi \cos(x)dx = 
\end{gather*}
\begin{gather*}
    =-\frac{1304}{81\pi} + \frac{32+\pi^4}{4\pi} - 4\int_0^\pi \cos(x)dx + \int_0^\pi x^2dx - \pi\int_0^\pi xdx = \\ = 
    =-\frac{1304}{81\pi} + \frac{32+\pi^4}{4\pi} + \int_0^\pi x^2dx - \pi\int_0^\pi xdx = \\ = 
    =-\frac{1304}{81\pi} + \frac{32+\pi^4}{4\pi} + \frac{\pi^3}{3} - \pi\int_0^\pi xdx = \\ = 
    =-\frac{1304}{81\pi} + \frac{32+\pi^4}{4\pi} - \frac{\pi^3}{6} = \\ = \frac{\pi^3}{12} - \frac{656}{81\pi}.
\end{gather*}
Ну и кошмар\ldots Но выбор у нас не слишком большой, продолжим и посчитаем нужное выражения для отрицательных $x$
\begin{gather*}
    \int_{-\pi}^{0} (-x - \frac{\pi}{2} + \frac{4}{\pi}\cos{x} + \frac{4}{9\pi}\cos{3x})^2dx;
    \intertext{Поступим как и прошлый разю}
    \int_{-\pi}^0 \bigg(x^2 + \pi x + \frac{16\cos(x)^2}{\pi^2} + \frac{16\cos(3x)^2}{81\pi^2} - \frac{8x\cos(x)}{\pi} - \frac{8x\cos(3x)}{9\pi} - 4\cos(x) +\\+ \frac{32\cos(x)\cos(3x)}{9\pi^2} - \frac{4\cos(3x)}{9} + \frac{\pi^2}{4}\bigg)dx;
    \intertext{Запишем это выражение как сумму интегралов, сразу сократим равные нулю.}
    \frac{16}{81\pi^2}\int_{\pi}^0 \cos(3x)^2dx + \frac{32}{9\pi^2}\int_{\pi}^0\cos(x)\cos(3x)dx - \frac{8}{9\pi}\int_{\pi}^0 x\cos(3x) + \frac{16}{\pi^2}\int_{\pi}^0 \cos(x)^2dx - \\ - \frac{8}{\pi}\int_{\pi}^0 x\cos(x) + \int_{\pi}^0 x^2dx + \pi\int_{\pi}^0 xdx + \frac{\pi^2}{4}\int_{\pi}^0 1dx = \\ =
    \frac{8}{81\pi} + \frac{16}{9\pi^2}\int_{\pi}^0\cos(4x) + \frac{16}{9\pi}\int_{-\pi}^0 \cos(2x)dx - \frac{8}{9\pi}\int_{\pi}^0 x\cos(3x) + \frac{16}{\pi^2}\int_{\pi}^0 \cos(x)^2dx - \\ - \frac{8}{\pi}\int_{\pi}^0 x\cos(x) + \int_{\pi}^0 x^2dx + \pi\int_{\pi}^0 xdx + \frac{\pi^2}{4}\int_{\pi}^0 1dx = \\ = 
    \frac{8}{81\pi} + \frac{8}{27\pi}\int_{\pi}^0 \sin(3x) + \frac{16}{\pi^2}\int_{\pi}^0 \cos(x)^2dx - \\ - \frac{8}{\pi}\int_{\pi}^0 x\cos(x) + \int_{\pi}^0 x^2dx + \pi\int_{\pi}^0 xdx + \frac{\pi^2}{4}\int_{\pi}^0 1dx = \\ = 
    \frac{8}{81\pi} - \frac{16}{81\pi} + \frac{16}{\pi^2}\int_{\pi}^0 \cos(x)^2dx - \\ - \frac{8}{\pi}\int_{\pi}^0 x\cos(x) + \int_{\pi}^0 x^2dx + \pi\int_{\pi}^0 xdx + \frac{\pi^2}{4}\int_{\pi}^0 1dx = 
\end{gather*}
\begin{gather*} 
    =\frac{-8}{81\pi} + \frac{16}{\pi^2}\int_{\pi}^0 \cos(x)^2dx - \frac{8}{\pi}\int_{\pi}^0 x\cos(x) + \int_{\pi}^0 x^2dx + \pi\int_{\pi}^0 xdx + \frac{\pi^2}{4}\int_{\pi}^0 1dx = \\ = 
    \frac{-8}{81\pi} + \frac{8}{\pi^2}\int_{\pi}^0 \cos(2x)dx - \frac{8}{\pi}\int_{\pi}^0 x\cos(x) + \int_{\pi}^0 x^2dx + \pi\int_{\pi}^0 xdx + \brackets{\frac{\pi^2}{4} + \frac{8}{\pi^2}}\int_{\pi}^0 1dx = \\ =
    \frac{-8}{81\pi} - \frac{8}{\pi}\int_{\pi}^0 x\cos(x) + \int_{\pi}^0 x^2dx + \pi\int_{\pi}^0 xdx + \brackets{\frac{\pi^2}{4} + \frac{8}{\pi^2}}\int_{\pi}^0 1dx = \\ =
    \frac{-8}{81\pi} - \frac{8}{\pi}\int_{\pi}^0 x\cos(x) + \int_{\pi}^0 x^2dx + \pi\int_{\pi}^0 xdx + \frac{32 + 4\pi^4}{4\pi} = \\ =
    \frac{-8}{81\pi} + \frac{8}{\pi}\int_{\pi}^0 \sin(x) + \int_{\pi}^0 x^2dx + \pi\int_{\pi}^0 xdx + \frac{32 + 4\pi^4}{4\pi} = \\ =
    \frac{-8}{81\pi} + \frac{-16}{\pi} + \int_{\pi}^0 x^2dx + \pi\int_{\pi}^0 xdx + \frac{32 + 4\pi^4}{4\pi} = \\ =
    \frac{-1304}{81\pi} + \frac{\pi^3}{3} + \frac{-\pi^3}{2} + \frac{32 + 4\pi^4}{4\pi} = \\ =
    \frac{\pi^3}{12} - \frac{656}{81\pi}.
\end{gather*}
Итак, кажется, что второй раз интеграл можно было не считать, если заметить чётнось функции, но уже поздно\ldots
\par Таким образом, мы нашли $\int_{-\pi}^\pi (f(x) - g(x))^2dx$ теперь найдём расстояние
\begin{gather*}
    \rho(f(x) - g(x)) = \sqrt{\abs{\int_{-\pi}^\pi (f(x) - g(x))^2dx}} = \sqrt{\abs{\frac{\pi^3}{6} - \frac{1312}{81\pi}}}
\end{gather*}


\i \textbf{0.}\\
Докажем жедаемое утверждение для двух подмножеств, а далее для произвольного натурального $n$ оно будет верно оп индукции так как $\bigcup_{i = 1}^n A_i = \bigcup_{i = 1}^{n-2} \cup (A_{n-1} \cup A_n)$, где в скобках открытое множество (по предположению индукции), а значит написано преесечение $n-1$ открытого множества, которое открыто по предположению индукции.
\par Осталось доказать утверждение для произвольных открытых $A$ и $B$. Для этого рассмторим произвольный элемент $x \in A \cup B$, заметим, что изначально он содержался в $A$ или $B$, пусть, без ограничения общности, в $A$, тогда там же и лежала его некоторая окрестность, а значит она также целиком содержится и в $A \cup B$. Таким образом, мы докзазали, что для любого элемента из $A \cup B$ некоторая его окрестность также содержится в нашем пересечении, а значит оно открыто. Что и требовалось доказать.


\i \textbf{0.}\\
Решение отдельно.


\i \textbf{4.}\\
Так как ранг матрицы равен 1, её можно представить в следующем виде
$$\begin{pmatrix}
    a&b \\ ak& bk
\end{pmatrix},$$
а в силу того, что $A = A^T$ верно, что $b = ak$, а тогда $det(A) = abk - abk = 0$.
$$f(A) = det(A) + tr(A) = tr(A),$$
при этом по условию $tr(A) \geq 0$, следовательно $f(A) \geq 0$, при этом 0 не достигается,
так как для этого должно выполнятся условие
\begin{gather*}
    \begin{cases}
        ak = b,\\
        a = -bk;
    \end{cases} =>
    a = \frac{b}{k} = -bk =>
    \frac{1}{k} = -k.
\end{gather*}
Это очевидно невозможно. Ещё стоит сказать, что при $k = 0$ матрица будет иметь вид $\begin{pmatrix} a&0 \\ 0&0 \end{pmatrix}$, в этом же случае значение $f(A)$ может быть сколь угодно близким к нулю, и сколь угодно большим. Таким образом, мы получили, что значения $f(A)$ ограничены нулём снизу (при этом 0 не достигается), и не ограничены сверху.