\subsection{Второе БДЗ}


\setcounter{iii}{5}

\i \textbf{(8)}
\begin{align*}
    \lim_{n\mapsto\infty} \{x_n\} &= \lim_{n\mapsto\infty} \frac{\sqrt[n]{8}-1}{\sqrt[n]{16}-1} =\\
    &= \lim_{n\mapsto\infty} \frac{\brackets{\sqrt[n]{2}-1}\brackets{\sqrt[n]{2}^2+\sqrt[n]{2}+1}}{\brackets{\sqrt[n]{2}-1}\brackets{\sqrt[n]{2}+1}\brackets{\sqrt[n]{4}+1}} =\\
    &= \lim_{n\mapsto\infty} \frac{\sqrt[n]{2}^2+\sqrt[n]{2}+1}{\brackets{\sqrt[n]{2}+1}\brackets{\sqrt[n]{4}+1}} = \\
    &= \frac{\lim_{n\mapsto\infty} \sqrt[n]{2}^2+\sqrt[n]{2}+1}{\brackets{\lim_{n\mapsto\infty} \sqrt[n]{2} + 1}\brackets{\lim_{n\mapsto\infty} \sqrt[n]{4} + 1}} = \\
    &= \frac{3}{4}.
\end{align*}

\i \textbf{(1)}\\
Если предел последовательности $\{x_n\}$ существует, то обозначим его за $x$, тогда верно следующее: $5+\frac{6}{x} = x$. Решениями этого уравнения являются только числа $-1$ и $6$. Легко убедиться, что все члены поледовательноси $\{x_n\}$ больше $0$, а значит, если предел существует, то он равен $6$.\\
Тепреь покажем, что предел существует:\\
$\letus x_k = 6+c, \ -1 < c \leq 1 => x_{k+1} = 5 + \frac{6}{6+c}$. Докажем, что $\abs{c} > \abs{6 - \brackets{5+\frac{6}{6+c}}}$.
\begin{gather*}
    \abs{c} > \abs{6 - \brackets{5+\frac{6}{6+c}}};\\
    \abs{c} > \abs{1 - \frac{6}{6+c}};\\
    \abs{6c + c^2} > \abs{c}, \text{что, очевидно, верно при } -1 \leq c \leq 1 (а именно в таком интервале находится) c \text{изначально} => \text{всегда} \abs{c} \leq 1.
\end{gather*}
Таким образом, мы показали, что предел существует, и он равен 6.

\i \textbf{(0)}\\
Отдельно разберём 3 случая:
\begin{enumerate}
    \item $n \not\vdots 2;$
    \item $n \vdots 2 \And n \not\vdots 4;$
    \item $n \vdots 4.$
\end{enumerate}
\par 1. $n \not\vdots 2$. В этом случае пеовое слагаемое примет значение $0$, а второебудет иметь вид $-2\frac{n}{n+1}$, что, очевидно, страмится к $-2$.
\par 2. $n \vdots 2 \And n \not\vdots 4$. При таких уловиях первое слгаемое будет равно $-1$, а второе "--- $2\frac{n}{n+1}$, тогда сумма стремится к $1$.
\par 3. $n \vdots 4$. В этом случае вервое слагаемое равно $1$, а второе "--- $2\frac{n}{n+1}$, а значит, сумма стремится к $3$.
Таким образом, частичными пределами являются значения $-2, 1, 3$. Покажем, что других нет.\\
Предположим, что такие нашлись, рассмотрим произвольный из них. Обозначим его за $x$, а подпоследовательность, сходящюуся к нему за $\{x_i\}$. Так как x отличен от чисел -2, 1 и 3, то $\forall N \in \NN \exists n_1, n_2 \in \NN n_1 > N \And n_2 > N$, такие, что $x_{n_1}$ и $x_{n_2}$ лежат в разных из рассмотренных выше поледовательностей, так как каждое число из исходной последовательности попало в одну из них (иначе предел совпадёт с одним из чисел $-2, 1, 3$, а мы предполагали обратное, но тогда легко убедиться, что последвательность $\{x_i\}$ расходится. Следовательно, никаких иныч частичных пределов быть не может.\\
Таким образом, для исходной последовательности $lim\ sup = 3, \ lim\ inf = -2$.

\i \textbf{(8)}\\
Давайте докажем, отрицание условия Коши, то есть 
$$\exists \epsilon > 0 \ \forall n_0 \in \NN \ \exists n_1, n_2 \in \NN: n_1 > n_0 \And n_2 > n_0 \And \abs{x_{n_1} - x_{n_2}} > \epsilon.$$
Покажем, что это утверждение верно, для этого покажем, что $\epsilon = \frac{1}{32}$ подходит. Множитель $(-1)^{\left[ \sqrt{\log_2{n}} \right]}$ по модулю всегда равен 1, а множитель $\frac{2n-100}{\sqrt{n^2+1}}$ стремится к $2$ (просто возведём в квадрат и убедимся в этом), следовательно, начиная с какого-то момента, все члены последовательности по модулю больше $1$. Также очевидно, что для первого множетелся найдётся сколь угодно большое значение $n$, в котором он равен $+1$, и аналогичное верно и для $-1$, а значит, $\forall n_0 \in \NN \ \exists n_1, n_2 \in \NN: n_1 > n_0 \And n_2 > n_0 \And x_{n_1} > 1 \And x_{n_2} < -1 => \abs{x_{n_1} - x_{n_2}} > 2 > \frac{1}{32}.$ Что и требовалось доказать.

\i \textbf{(6)}
\begin{align*}
    \lim_{n\mapsto\infty} x_n &= \lim_{n\mapsto\infty} \brackets{\frac{n+7}{n+4}}^{\frac{2n^2-1}{n+5}} = \lim_{n\mapsto\infty} \brackets{1 + \frac{3}{n+4}}^{\frac{2n^2-1}{n+5}} =\\
    &= \lim_{n\mapsto\infty} \brackets{\brackets{1 + \frac{3}{n+4}}^{\frac{n+4}{3}}}^{\frac{2n^2-1}{n+5}\cdot\frac{3}{n+4}} = \lim_{n\mapsto\infty}\brackets{\brackets{1+\frac{3}{n+4}}^\frac{n+4}{3}}^{\frac{6n^2-3}{n^2+9n+20}} = \\
    &= \lim_{n\mapsto\infty} \brackets{\brackets{1+\frac{3}{n+4}}^{\frac{n+4}{3}}}^{\lim_{n\mapsto\infty} \frac{6n^2-3}{n^2+9n+20}} = \lim_{n\mapsto\infty} \brackets{\brackets{1+\frac{3}{n+4}}^{\frac{n+4}{3}}}^6;
    \intertext{Выполним замену: $\frac{n+4}{3}$. Тогда заметим, что при $n\mapsto\infty$ верно $t\mapsto\infty$.}
    &= \lim_{t\mapsto\infty} \brackets{\brackets{1+\frac{1}{t}}^t}^6 = e^6.
\end{align*}