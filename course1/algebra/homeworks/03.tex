\subsection{Третье БДЗ}

\i Для начала заметим, что вектор $(2, 2, \ldots, 2)$ будет являться базисом $B$ (что очевидно).\\
Теперь рассмотрим в $A$ множество векторов $(1, 1, \ldots, 1) \cup \{a_i| i \in \{1, 2, \ldots, n-1\}\}$, где в $a_i$ $i$-тый элемент равен 1, а остальные "--- нули. Легко заметить, что для $A$ это множество векторов является порождающим (в самом деле, если из первого вычесть все остальные, то получится стандартный базис). Также это множество является линейно независимым, так как в любой линейной комбинации с ненулевым коэффициентом при $(1, 1, \ldots, 1)$ последний элемент в результате будет отличен от нуля, а в случае, когда этот коэффициент 0, для любого ненулевого коэффицента при $a_i$ верно, что $i$-тый элемент результата будет отличен от 0. Значит выбранное множество векторов является базисом в $A$. При этом $2(1, 1, \ldots, 1)$ является бизисом в $B$, а значит наши базисы согласованы.

\i Заметим, что $\begin{pmatrix}
    f_1 & f_2 & f_3
\end{pmatrix} = \begin{pmatrix}
    e_1 & e_2 & e_3
\end{pmatrix}\begin{pmatrix}
    3 & 9 & -3\\
    3 & 3 & 3\\
    0 & -6 & 6
\end{pmatrix}.$
$$\begin{pmatrix}
    3 & 9 & -3\\
    3 & 3 & 3\\
    0 & -6 & 6
\end{pmatrix} \sim \begin{pmatrix}
    3 & 0 & 0\\
    0 & -6 & 6\\
    0 & -6 & 6
\end{pmatrix} \sim \begin{pmatrix}
    3 & 0 & 0\\
    0 & -6 & 0\\
    0 & 0 & 0
\end{pmatrix}.$$
Тогда по теореме о согласованных базисах $A/B \cong \ZZ_3 \times \ZZ_6 \times \ZZ$.

\i рассмотрим группу $(\QQ, +)$. Как известно это именно абелева группа, и её мощность счётна. Докажем, что эта группа не является конечно порождённой.\\
Предположим обратное, тогда существует некоторое подмножество этой группы $S = \{\frac{a_1}{b_1}, \frac{a_2}{b_2}, \ldots, \frac{a_n}{b_n}\}$ такая, что $\forall J \in \QQ\ \exists \{x_i| i \in \{1, 2, \ldots, m\}\}: J = \sum_{i=0}^{m} \frac{a_{x_i}}{b_{x_i}}$. В силу конечности $S$ и бесконечности числа простых чисел можно найти такое простое $p$, что оно будет взаимнопросто со всеми $b_i$. Тогда, по нашему предположению $\frac{1}{p}$ представима в виде конечной суммы элементов из $S$, однако знаменатель такой суммы может содержать только простые, которые содержатся в $b_i$. Противоречие, а значит $(\QQ, +)$ не является конечно порождённой группой. Что и требовалось доказать.

\i Для начала скажем, что в любом подгруппе $\RR^n$ существует базис над $\RR$ размера $m \leq n$. Очевидно, что для любого вектора с коодринатами $(a_1, a_2, \ldots, a_m)$ найдётся вектор, $(\{a_1\}, \{a_2\}, \ldots, \{a_n\})$, где $\{x\}$ "--- целая часть $x$. Тогда все такие вектора лежат в фиксированном конечном параллелепипеде размера 1, тогда по условие дискретности их конечное число. Заметим, что множество всех таких векторов будет являться порождающим для $L$ над $R$, также в силу его конечности в нём найдётся коненчый базис над $\ZZ$. При этом он также будет являться базисом над $\RR$, а значит, его размер не будет превосходить $n$. Что и требовалось доказать.
