\subsubsection{Восьмое БДЗ}

\i Давайте просто распишем алгоритм Евклида и по нему воостановим нужные многочлены.
\begin{gather*}
    (x^5 + x^3 + x,\ x^4 + x + 1) = (x^4 + x + 1,\ x^3 + x^2) = (x^3 + x^2,\ x^3 + x + 1) = \\ =
    (x^3 + x + 1,\ x^2 + x + 1) = (x^2 + x + 1,\ x^2 + 1) = (x^2 + 1, x) = (x, 1) = 1.
\end{gather*}
*также мы не забывали, что работает над полем $\ZZ_2$.
\par Итак, осталось только найти представление НОД. при помощи обратного алгоритма Евлкида найдём, что это 
$$
    (x^3 + x)(x^5 + x^3 + x) + (x^4 + x + 1)(x^4 + x + 1) = 1.
$$


\i Прокрутим алгоритм Евклида для многочленов $P(X) = x^4 + 3x + 1$ и $Q(X) = x^3 + 1$, вясним, что $(P, Q) = 7$ и 
$$
    (-4x^2 + 2x - 1)(x^4 + 3x + 1) + (4x^3 - 2x^2 + x + 8)(x^3 + 1) = 7.
$$
Тогда верно следующее равенство
$$
    \frac{\alpha^3 + \alpha + 1}{\alpha^3 + 1} \modeq{P} \frac{1}{7}(\alpha^3 + 3\alpha + 1)(4x^3 - 2x^2 + x + 8).
$$
Такое равенство имеет смысл потому, что многочлен $P$ является зануляющим. Осталось посчитать парвую часть и найти её остаток по модулю $P$, получим, что искомый многочлен $f(x) = 4x^3 - 12x^2 + 8x + 8$.


\i Так как $\brackets{\sqrt{3}+\sqrt{5}}^2 = 8 + 2\sqrt{15}$ верно, что $Q(x) = \brackets{x^2 - 8}^2 - 60$ знауляем $x$. теперь покажем, что $Q$ неприводим. Заметим, что 
$$
    Q(x) = (x - x_1)(x-x_2)(x-x_3)(x-x_4),
$$
где $x_i = \pm\sqrt{8 \pm 2\sqrt{15}}$. Тогда минимальный многочлен должен состоять в точности из этих скобок (возможно не из всех), однако легко убедиться, что прозведение любого числа из них (меньше 4) даёт многочлен, который не может существоваться над $\QQ[x]$, следовательно найденный нами многочлен минимален.


\i Решим уравнение и получим, что корнями являются $\pm\frac{1}{2} \pm \frac{\sqrt{3}}{2}i$. А значит разложение над $\QQ$ имеет вид $F = f_0 + \frac{\sqrt{3}}{2}if_1$, и его степень равна 2.


\i Давайте попробуем выразить $y \in F$ через $x \in K$.
\begin{gather*}
    y = \frac{x^2 + 1}{x};\\
    xy = x^2 + 1;\\
    x = \frac{y \pm \sqrt{y^2 - 4}}{2} = \frac{1}{2}y \pm \sqrt{\brackets{\frac{1}{2}y}^2 - 1}.
\end{gather*}
Отсюда мы получаем, что $[F : K] \leq 2$. В самом деле, $[C(y, \sqrt{y^2 - 4}]) : K] = 2$, и при этом мы можем выразить любое $\frac{f(x)}{g(x)},\ f, g \in \CC[x]$. Осталось только понять, что $[F : K] \ne 2$, что очевидно, например, потому, что $F \ne K$.