\subsection{Девятое БДЗ}

\i Для начала разберёмся с константами и многочленами степени 1 (то есть $0$, $1$, $x$, $x+1$). С ними всё понятно, все, корме нуля, неприводимы. Далее считаем, что степень многочлена больше 1.
\par Заметим, что произвольный многочлен $P(X) = p_4x^4 + p_3x^3 + p_2x^2 + p_1x + p_0$ над $\ZZ_2$ неприводим только тогда, когда у него нет корней (при условии, что его степень больше одного). А это значит, что $P(1) = 1$, следовательно $p_0 = 1$. Также мы знаем, что $P(1) \modeq{2} \sum_{i=0}^4p_i$, а значит у всякого неприводимого многочлена над $\ZZ_2$ нечётное число ненулевых коэффицентов. В таком случае, значение $P(x)$ всегда равно 1. Теперь покажем, что любой такой многочлен, кроме $(x^2 + x + 1)^2$ неприводим. Всамом деле, такой многочлен или должен являться произведением 2 таких же, или неприводим. Из-за ограничения на степень, такой многочлен или неприводим, или имеет неприводимый делитель степени не больше 2. При этом такой делитель всего 1, это $x^2 + x + 1$, таким образом, любой многочлен с начётной суммой коэффициентов и свободным членом равным 1 степени не больше 4 неприводим, кроме $x^2 + x + 1$.
\par Теперь разберёмся с многочленами степени 5. Для удобства будем называть крутым многочлен, который не имеет корней. Таких многочленов степени 5 ровно 8 (первый и последний поэффицент фиксированны, среди остальных 4 1 или 3 единицы). Замечательно, теперь посмотрим на те из них, которые приводимы. Любой из них должен имень крутой делитель степени не больше 2, а мы уже знаем, что такой только один, это $x^2 + x + 1$. Это значит, что любой крутой многочлен степени 5 имеет вид $g(x)(x^2 + x + 1)$, где $g(x)$ "--- тоже крутой степени не больше 3. Заметим, что любой крутой степени 3 неприводим, так как иначе должен иметь крутой делитель степени 1, а таких нет. Всего крутых степени 3 ровно 2 ($x^3 + x + 1$ и $x^3 + x^2 + 1$), а значит крутых приводимых степени 5 ровно 2, а значит неприводимых 6. Что и требовалось доказать.


\i По теореме 2 из лекции 9 мы знаем, что такое поле существует и имеет вид поля разложения многочлена $x^{7^3} - x$ над полем $\ZZ_7[x]$. 


\i Сумма всех элементов будет равно нулю (если характеристика поля не 2) так как их можно просто разбить на пары обратных по сложению (при этом каждому, кроме нуляю, сопоставиться элемент, отличный от него самого), а значит и сумма бдет равна нулю. Теперь разберёмся с произведением. Мы знаем, что существует изоморфизм, который переводит наше поле в поле разложения многочлена, где все элементы "--- делители некоторого многочлена, тогда мы знаем их произведение, которое будет равно -1.



\i Заметим, что любой многочлен с нечётным коэффицентов точно не будет делителем нуля. В самом деле, рассмотрим факторгруппу по максимальному идеалу, она будет изоморфна $\ZZ_2[x]$, тогда такой многочлен попадёт в ненулевой класс, а значит никак не может делить ноль. При этом все остальные (то есть те, у которых все коэффиценты чётны) делят ноль. В самом деле, достаточно возвести их в квадрат, тогда так как из исходного многочлена выносилась двойка, из квадрата вынесется четвёрка, а значит он будет нулём.
\par Теперь нильпотенты. Мы уже знаем, что все многочлены с чётныи коэффициентами в квадрате равны нулю, а значит нильпотетны, с другой стороны те, у которых есть хотябы один ненулевой коэффициент не делят ноль, а значит не нильпотетны. Победа!