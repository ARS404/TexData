\subsection{Шестое БДЗ}

\i Из курса линейной алгберы мы знаем, что обратимость матрицы равносильна тому, что она не вырождена (то есть её определитель отличен от нуля). Также из этого самого курса нам известно, что делители нуля совпадают с вырожденными матрицами. Осталось разобраться с нильпотентными матрицами. Во-первых, мы знаем, что её определительно должен быть равен нулю, поэтому далее будем считать, что такая матрица имеет вид 
$$
\begin{pmatrix}
a & b \\
ka & kb
\end{pmatrix}.
$$
Далее нам поможет лемма о стабилизации. Из неё следует, что любая нильпотентная матрица уже во второй степени станет равна нулевой. Тогда
$$
\begin{pmatrix}
    0 & 0 \\ 0 & 0
\end{pmatrix} = \begin{pmatrix}
    a & b \\
    ka & kb
\end{pmatrix}^2 = \begin{pmatrix}
    a^2 + kab & ab + kb^2\\
    ka^2 + k^2ab & kab + k^2b^2
\end{pmatrix};\\
=> a = 0,\ kb = 0.
$$
Таким образом, мы получили, что любая нильпотентная матрица имеет след равный нулю, при этом очевидно, что любая такая матрица нильпотентна.

\i Легко понять, что любая группа $\frac{n}{k}\ZZ_k$, где $k | n$, является идеалом, давайте докажем, что любая другая нам не подходит.
\par Пусть нашолся какой-то идеал $G$, который не соответствует описанным выше критериям. Тогда пусть $g_0$ "--- его минимальный элемент. Алгоритм Евклида позволяет нам утвержадть, что $(n, g_0) = g_0$, так как иначе $g_0$ не будет являться минимальным, следовательно $g_0 | n$. По аналогичным соображениям мы понимаем, что $g_0 | g \forall g \in G$, следовательно $G$ можно описать как $g_0\ZZ_{n/g_0}$, что равносильно условию выше.

\i Давайте рассмотрим множество $H$ всех многочленов с чётным свободным членом. Сперва заметим, что $H$ будет являться кольцом, в силу свойств многочленов. Также очевидно, что $H$ будет являться идеалом, так как свободный член произведения равен произведению свободных членов. Замечательно, теперь нам осталось только доказать, что $H$ будет не главным идеалом.
\par Предположим обратное, то есть $H = \ZZ[x](f)$, где $f$ "--- некоторый многочлен, отличный от нуля (иначе совсем неинтересно). Сразу заметим, что $f$ не может являться константой, так как в этом случае константа должна быть чётной, но тогда в $H$ не попадёт многочлен $x + 2$ (хотя должен). Если же степень $f$ больше нуля, то в $H$ не попадёт $2$. Значит мы пришли к противоречию и $H$ всё-таки является не главным идеалом.

\i Для начала заметим, что любые 2 различным многочлена степени меньше $n$ будут порождать нам разные классы смежности. Это легко понять из того, что на многочленах действует деление с остатком, а значит и алгоритм Евклида, а такие 2 многочлена очевидно будут давать различные остатки. При этом очевидно, что класс смежности, порождённый многочленом $g$ ь=будет совпадать с классном смжености, порождённым многочленом $g\ (mod\ f)$. Таким образом мы получили, что искомая фактор-группа изморфна остаткам по модулю $f$. В этой группе можно выбрать базис $(1, x, x^2, \ldots, x^{n-1})$ (линейная независимость очевидно, как и то, что такое множество порождает любой остаток, так как его степень строго меньше $n$). Таким образом, мы получили, что размерность фактор-группы равна $n$, что и требовалось доказать.

\i Предположим, что нашёлся ненулевой лемент $a$, который перешёл в ноль. Тогда мы знаем, что любой ненёлевой элемент $F$ можно было представить как $x = xa^{-1}a$, тогда гомоморфизм $f$ переведёт $x$ в $f(xa^{-1})f(a) = f(xa^{-1})0 = 0$. Таким образом, все элементы перешли в ноль (так как ноль тоже мог отправиться только туда). Теперь предположим, что нашлись 2 элемента, которые $f$ перевёл в один и тот же элемент. Тогда их разность (которая была отлична от нуля) перешла в ноль, а мы уже знаем, что в таком случае все элементы $F$ отправляются в ноль.
\par Таким образом мы знаем, что у нас все элементы переходят в ноль или же любые 2 переходят в различные. В этом случае гомоморфизм будет инъективен, а значит $F$ и $Im(F)$ будут изоморфны.  