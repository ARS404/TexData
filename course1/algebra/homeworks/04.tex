\subsection{Четвёртое БДЗ}


\i Начнём с того, что найдём порядки элементов в $\ZZ_2, \ZZ_3, \ZZ_4$:
\begin{itemize}
    \item $\ZZ_2$:
    \begin{itemize}
        \item $ord(0) = 1;$
        \item $ord(1) = 2;$
    \end{itemize}
    \item $\ZZ_3$:
    \begin{itemize}
        \item $ord(0) = 1;$
        \item $ord(1) = 3;$
        \item $ord(2) = 3;$
    \end{itemize}
    \item $\ZZ_4$:
    \begin{itemize}
        \item $ord(0) = 1;$
        \item $ord(1) = 4;$
        \item $ord(2) = 2;$
        \item $ord(3) = 4.$
    \end{itemize}
\end{itemize}
Тут мы пользовались известным фактом о том, что в $\ZZ_n$ $ord(x) = \frac{n}{(n, x)}$.
\par Теперь посчитаем количество элементов подряка 2 в $\ZZ_2 \xor \ZZ_3 \xor \ZZ_4$. Для этого и каждой группы надо выбрать элемент порядка 2 или 1, но при этом хотябы один с порядком 2 должен присутствовать, тогда получим, что таких ровно $2^2 - 1 = 3$.
\par Разберёмся с количеством элементов порядка 3. В $\ZZ_2$ и $\ZZ_4$ можно взять только по 1 элементу, а в $\ZZ_3$ "--- 2, тогда общее количество равно 2.
\par Для порядка 4. В $\ZZ_2$ подойдёт любой, в $\ZZ_4$ можем взять 2 элемента (1 или 3), в $\ZZ_3$ "--- только 0, итого получим, что у нас 4 варианта.
\par Для 6. В $\ZZ_2$ подходит любой, в $\ZZ_3$ мы обязаны взять 1 или 2, а в $\ZZ_4$ подходят 1 и 2. Однако мы также посчитали случай, когда в $\ZZ_2$ и $\ZZ_4$ выбраны 1, однако такой случай не подходит, так как итоговый порядок будет равен 3. Итого получим, что вариантов $2^3 - 2 = 6$.
\par Тут мы очень активно пользовлись тем фактом, что порядок <<собираемого>> элемента в итоговой группе будет равен НОД порядков его частей в своих группах.


\i Как известно, любую абелеву конечнопорождённую группу ($A$) можно разложить в прямую сумму вида 
$$A \cong \ZZ_{p_1^{k_1}} \xor \ZZ_{p_2^{k_2}} \xor \ldots \xor \ZZ_{p_n^{k_n}} \xor \ZZ \xor \ldots \xor \ZZ.$$
Согласно этому утверждению данная в условии группа $G$ представляется в виде $\ZZ_3 \xor \ZZ_5 \xor \ZZ_5$ или $\ZZ_3 \xor \ZZ_{5^2}$.
\par Рассмотрим первый случай. Докажем, что если $(p_1, p_2) = 1$, то группа $\ZZ_{p_1} \xor \ZZ_{p_2}$ будет циклической. Для этого просто покажем, что элемент $(1, 1)$ окажется образующим. Для этого просто заметим, что $p_1(1, 1) = (0, p_1)$, а $p_2(1, 1) = (p_2, 0)$. В силу взаимнопростости $p_1$ и $p_2$ получим, что любой элемент вида $(x, y)$ можно получить (по алгоритму Евклида) в виде $a(0, p_1) + b(p_2, 0) = (ap_1 + bp_2)(1, 1)$ (все числа рассматриваем п осоответствующим модулям). Таким образом мы доказали, что элемент $(1, 1)$ будет являться образующим, что в силу конечности группы равносильно тому, что $\ZZ_{p_1} \xor \ZZ_{p_2}$ циклична, но это протеворечит условиям, а значит такой вариант невозможен.
\par Таким образом остался только один вариант в котором $G \cong \ZZ_3 \xor \ZZ_5 \xor \ZZ_5$.
\par Займёмся поиском подгрупп порядка 15. Такая группа будет иметь вид $\ZZ_3 \xor \ZZ_5$, что по выше написанному равносильно тому, что она будет циклична. При этом в $\ZZ_n$ ровно $\phi(n)$ образующих элементов (что очевидно, так как только они имеют порядок $n$). Более того, каждый из таких элементов будет порождать какую-то группу порядка 15 (ровно 1), и в каждой группе будет $\phi(15)$ таких элементов. Отсюда следует, что искомых групп будет ровно число элементов порядка 15 (в $\ZZ_3 \xor \ZZ_5 \xor \ZZ_5)$ делить на $\phi(15)$, что равно $\frac{2 \cdot 5 \cdot 5 - 2}{2 \cdot 4} = \frac{48}{8} = 6.$ Тут число элементов порядка 15 мы посчитали также, как и в предыдущей задаче.
\par Теперь посчитаем количество групп порядка 5. Для начала заметим, что в такой подгруппе могут находится только элементы порядка 1 и 5 (так как порядок элемента должен делить порядок группы). При этом порядок 1 может иметь только нулевой элемент, а значит оставшиеся 4 в каждой группе будут иметь порядок 5. Всего элементов порядка 5 ровно 24 (в $\ZZ_3$ можем взять только 0, в $\ZZ_5 \xor \ZZ_5$ любой, кроме $(0, 0)$). При этом каждый из таких элементов попадёт ровно в 1 из искомых подгрупп. Таким образом подгрупп порядка 5 ровно $\frac{24}{4} = 6$.


\i Обозначим вектор из условия $(32, 31, 0)$ за $v$. Для начала выразим $v$ через базис $B$ в рациональных числах. В качестве базиса можем взять его порождающие из условия (в самом деле, их линейная независимость очевидна). Для самого выражения нам пригодится алгоритм Гаусса.
\begin{gather*}
    \coolMatrix{4 & 8\\
                2 & 10\\
                -100 & 120} {
                32 \\ 31 \\ 0} \sim
    \coolMatrix{0 & -12\\
                2 & 10\\
                0 & 650\\}{
                -30 \\ 31 \\ 1550} \sim
    \coolMatrix{0 & -12\\
                2 & 10\\
                0 & -4}{
                -30 \\ 31 \\ -10} \sim 
    \coolMatrix{0 & 1\\
                1 & 0\\
                0 & 0}{
                3 \\ \frac{5}{2} \\ 0}.
\end{gather*}
Таким образом, мы получили, что $v$ не выражается через базис $B$ в натуральных числах, в то же время $2v$ выражается, а значит порядок смежного класса будет равен 2.


\i Для начала поймём, что ранги групп $A$ и $B$ равны тогда и только тогда, когда определитель матрицы из условия (далее $C$) отличен от нуля. В самом деле, это очевидный факт из линейной алгебры. Далее просто будем считать, что ранги $A$ и $B$ равны и $det(C) \ne 0$.
\par Теперь воспользуемся теоремой со согласованных базисах, из неё получим, что бизас $B$ можно получить, как базис $A$, домноженный на диагональную матрицу с элементами $u_i$ на диагонали (дфлее $D$). Также мы знаем, что
$$A/B \cong \ZZ_{\abs{u_1}} \xor \ZZ_{\abs{u_2}} \xor \ldots \xor \ZZ_{\abs{u_n}}.$$
А значит, $\abs{A/B} = \abs{u_1 u_2 \ldots u_n}$. С другой стороны тамуже самомму равен и определитель матрицы $D$. Осталось только понять, что от произвольного базиса в $A$ можно пререйти к согласованному базису при помощи домножения на некоторую матрицу, с определителем 1, аналогично и для $B$ (в обратную сторону), а тогда $C = XDY$, где $det(X) = det(Y) = 1$, следовательно $det(C) = det(D)$. таким образом мы получили желаемое равенство.