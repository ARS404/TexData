\subsubsection{Второе БДЗ}


\i Давайте рассмотрим следующие группы:
\begin{itemize}
    \item $G$ "--- группа симметрий квадрата.
    \item $H_1 \subset G$ "--- группа, порождённая симметриями квадрата относительно его диагоналей (то есть в группе 4 элемента: две симметрии относительно диагоналей, симметрия относительно центра и тождественное преобразование).
    \item $H_2 \subset H_1$ "--- группа, порождённая симметрией относительно одной из диагоналей квадрата (в этой группе всего 2 элемента: одна симметрия и тождественное преобразование).
\end{itemize}
$H_1$ абелева, поэтому нормальность $H_2$ в $H_1$ очевидна, также несложно убедится в нормальности $H_1$ в $G$. Осталось показать, что $H_1$ является нормальной в $G$. Это очевидно, так как симметрии относительно диагоналей сопряжены в $G$ повотором на $90^{\circ}$.    

\i Для начала скажем, что $h_2\hatch = h_1 h_2 h_1^{-1}$ лежит в $H_2$ (как следствие нормальности $H_2$). Тогда получим:
\begin{gather*}
    h_1 = h_2\hatch h_1 h_2^{-1};\\
    h_1 = h_2\hatch h_2^{-1} h_2 h_1^{-1} h_2^{-1};\\
    h_1 = h_2\hatch h_2^{-1} h_1\hatch;\\
    \text{при этом $h_1\hatch$ пренадлежит $H_1$ (как следствие нормальности $H_1$)};\\
    h_1 h_1\hatch^{-1} = h_2\hatch h_2^{-1}.
\end{gather*}
Легко заметить, что левое слагаемое лежит в $H_1$, а правое "--- в $H_2$. Тогда, из условия становится очевидно, что оба равны $e$, а значит $h_1\hatch = h_1$ и $h_2\hatch = h_2$. Теперь из равенства $h_1 h_2 h_1^{-1} = h_2\hatch = h_2$ напрямую следует желаемое. Что и требовалось доказать.

\i Очевидно, что все диагональные матрицы лежат в центре $GL$. Теперь рассмотрим произвольную недиагональную матрицу $A$ и покажем, что она не лежит в центре.\\
Предположим обратное, то есть, $\forall B \in GL : BA = AB$. Тогда есть такие $i, j\ (i \ne j)$, что в $A$ ячейка $i, j$ ненулевая. Тогда в качестве матрицы $B$ возьмём единичную матрицу с единицей в ячейке $i, j$ (очевидно, что её определитель отличен от нуля). Тогда легко убедиться, что $AB \ne BA$, в самом деле, $AB_{i, j} = A_{i, i} + A_{i, j}$, а $BA_{i, j} = A_{i, i}$. Противоречие, а значит никакая недиагональная матрица не может лежать в центре $GL$. Что и требовалось доказать.

\i Предположим обратное, чтогда найдутся два такие собственные подгруппы $(\ZZ, +)$ $A$ и $B$, что существует некотороый изоморфизм $\phi: \ZZ \seek A \times B$. Для произвольного $z \in \ZZ$ верно $z = \sum_{i=1}^{z} 1$. Тогда, пусть $\phi(1) = (a, b)$, а значит
\begin{gather*}
    \phi(z) = \phi(\sum_{i=1}^{z} 1) = \sum_{i=1}^{z}\phi(1) = \sum_{i=1}^{z} (a, b).
\end{gather*}
Однако, в таком случае образ, что ядро $\phi$ состоит только из элементов вида $(an, bn)$ (для целого $n$), но тогда, если $a \ne 0$, то $(a, b + b) \not\in Im(\phi)$, а, если $b \ne 0$, то $(a + a, b) \not\in Im(\phi)$. Значит, если искомый гомоморфизм существует, то он переводит единицу в $(0, 0)$, а тогда по выше сказанному любой элемент $Im(\phi) = (0, 0)$, но тогда и $A$ и $B$ были несобственными подгруппами, что протеворечит условию, следовательно такое невозможно.