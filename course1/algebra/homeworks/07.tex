\subsubsection{Седьмое БДЗ}

\i Для начала докажем симметричность данного многочлена. это очевидно, так как он раскладывается в произведение всех возможных попарных сумм переменных $x_1, x_2, x_3, x_4$ (все в первой степени). Поэтому никакая транспозиция аргументов его не изменит.
\par Ну а теперь разложим его в элементарные симметрические. Степень нашего многочлена равна 6 (при этом все одночлены будут иметь именно такую степень), так что его можно представить в виде
    $$a_1(\sigma_1\sigma_2\sigma_3) + a_2(\sigma_1^2\sigma_4) + a_3(\sigma_2\sigma_4) + a_4(\sigma_2^3).$$
Давайте искать коэффициенты подстановками:
\begin{itemize}
    \item $P(1, 1, 1, 0):$ \\
        $f(1, 1, 1, 0) = a_1(3\cdot3\cdot1) + a_4 = 8;$
    \item $P(2, 1, 1, 0):$ \\
        $f(2, 1, 1, 0) = a_1(4\cdot5\cdot2) + 4a_4;$
    \item $=> a_1 = 1,\ a_4 = -1;$
    \item $P(1, 1, 1, 1):$ \\
        $f(1, 1, 1, 1) = (4\cdot6\cdot4) + a_2(4\cdot4) + 6a_3 - 4\cdot4 = 64;$
    \item $P(2, 1, 1, 1):$ \\
        $f(2, 1, 1, 1) = (5\cdot9\cdot7) + a_2(5\cdot5\cdot2) + 18a_3 - 49 = 216;$
    \item $=> a_2 = -1, a_3 = 0.$
\end{itemize}
Таким образом, мы получили, что наш многочлен равен $\sigma_1\sigma_2\sigma_3 - \sigma_1^2\sigma_4 - \sigma_3^2.$


\i Итак, пусть корни многочлена это $X = \{x_1, x_2, x_3, x_4\}$, тогда мы знаем, что 
    $$\frac{1}{x_1} + \frac{1}{x_2} + \frac{1}{x_3} + \frac{1}{x_4} = \frac{\sigma_3(X)}{\sigma_4(X)}.$$
С другой стороны, нам знакомы теорема Виета, котора утверждает, что $(-1)^ia_{n-i}/a_n = \sigma_i(X)$, где $a_k$ "--- $k$-тый коэффицент многочлена с корнями $X$. Тогда значение искомого выражения найти очень просто. это будет $-\frac{1/5}{1/5} = -1$.


\i Пусть, как и в предыдущей заначе, $X$ "--- корни данного многочлена (далее $f$), по которым мы хотим построить многочлен $g$ с корнями $Y$. Тут нам на помошь снова придёт теорема Виета. Нам достаточно найти $\sigma_3(Y), \sigma_2(Y), \sigma_1(Y)$, при том, что мы знаем $\sigma_3(X), \sigma_2(X), \sigma_1(X)$ (этого будет достаточно, так как исходный многочлен приведённый, так что никаких проблем со старшим коэффицентом быть не должно). Итак, давайте выражать...
\begin{itemize}
    \item $\sigma_1(Y) = \sigma_1(X)^3 - 3\sigma_2(X)\sigma_1(X) + 3\sigma_3(X) = 0 + 0 - 3 = -3;$
    \item $\sigma_2(Y) = \sigma_2(X)^3 - 3\sigma_1(X)\sigma_2(X)\sigma_3(X) + 3\sigma_2(X)^2 = -1 + 0 + 3 = 2;$
    \item $\sigma_3(Y) = \sigma_3(X)^3 = -1.$
\end{itemize}
Таким образом, искомый многочлен равен $x^3 - 3x^2 + 2x - 1$.

\i Рассмотрим 2 случая.
\begin{enumerate}
    \item Многочлен имеят корень степени 3. Но тогда мы знаем, что сумма всех корней равно 0 (так как коэффициент при $x^2 = 0$), ну а тогда и все корни равны 0, а значит и $\lambda = 0$. Получим многочлен $x^3 - 12x$, который, очевидно, не подходит, так как его корни это $0$ и $\pm\sqrt{12}$.
    \item Многочлен имеет вид $(x-x_1)^2(x-x_2)$, тогда на его коэффициенты можно составить слеюующую систему уравнений:
    $$\begin{cases}
        2x_1 + x_2 = 0,\\
        x_1^2 + 2x_1x_2 = -12,\\
        x_1^2x_2 = -\lambda;
    \end{cases} => \begin{cases}
        x_2 = -2x_1,\\
        x_1^2 = 4,\\
        \lambda = 2x_1^3;
    \end{cases}$$
    следовательно, $\lambda = \pm16$.
\end{enumerate}


\i По сути, нас просят доказать, что нет бесконечно убывающей последовательности одночленов. Будем доказывать это по индукции по количеству переменных ($n$)
\par База: $n = 1$ "--- очевидно, что бесконечной убывающей последовательности при таких условиях быть не может.
\par Переход от $n = k$ к $n = k+1$. Предположим обратное, тогда на $k+1$ прееменной существует бесконечно убывающая последовательность, в то время как на $k$ её нет. Рассмотрим эту последовательность. Заметим, что степень при первой переменной в такой последовательнсоти одночленов не может убывать бесконечно (так как она не меньше нуля), а значит с некоторого момента она фиксируется. Заметим, что в таком случае наша последовательность с этого момента будет бесконечно убывающей для $k$ переменных (достаточно из всех одночленов выкинуть первую переменную, что возможно, так как во все из них она входит в равной степени). Таким образом мы получили противоречие, а значит на $k+1$ переменной бесконечной убывающей последовательности тоже быть не может. Мы соврешили переход, а значит и доказали желаемое.