\subsection{Первое БДЗ}

\i Для начала заметим, что по определению $ord(g)$ верно, что $g^{x} = g^{y} <=> x \modeq{m} y$. Действительно, пусть $y = x + am$, тогда $g^{y} = g^{am} \times g^{x} = (g^m)^a \times g^x = g^x$. Не сложно проделать аналогичное и в другую сторону, поэтому далее считаем, что все степени меньше $m$. Также скажем, что $g^{i}$ является обратным (который единственный) к $g^{m-i}$, что очевидно. В силу того, что обратный элемент в группе единственный можно утверждать, что нам нужно найти минимальное такое $n$, что $kn \divby m$. Очевидно, что такое $n$ равно $\frac{m}{(k, m)}$.

\i Сперва скажем, что в силу того, что между элементами $G = \{g^k \ |\ k \in \ZZ_{0+}\}$ сториться биекция или сюрьекция, то $G$ конечна или счётна. Теперь рассмотрим $H \subseteq G$. Найдём в $H$ элемент наименьшей ненулевой степени (в силу вышесказанного это становится очевидно). Пусть это некотороый $h = g^{a}$ ($a$ "--- наименьшее большее нуля). Покажем, что $H = \{h^{x} \ |\ x \in \ZZ_{0+}\}$. Во-первых, для элементов вида $h^x = g^{ax}$ очевидно, что они включены в $H$ (так как включен элемент $h$). Во-вторых, элеменнты вида $g^b \ |\ b \not\divby\ a$ не лежат в $H$. Предположим обратное, тогда для 2 произвольных элементов $g^{l},\ g^{r}$ верно, что элемент $g^{l\%r}$ также лежит в $H$, тогда по элгоритму Евклида верно, что в $H$ лежит $g^{(l, r)}$. Вернувшись к предположению о том, что в $H$ нашёлся такой элемент $g^b \ |\ b \not\divby\ a$, поймём, что тогда в $H$ лежит $g^{(a, b)}$, и при этом $a > (a, b)$, что приводит к противоречию с нашим предположением. Таким образом, мы показали, что любая произвольная подгруппа $H \subseteq G$ также является циклической.

\i В начале заметим, что группа чётного порядка содержит в себе конечное число элементов, среди которых нечётное число отличных от $e$. Тогда можно рассмотреть произвольное разбиение $G$ на пересекающиеся только по $e$ подгруппы образованные как $<$$g_i$$>$, для очередного $g_i$, который ещё не содержится ни в какой подгруппе. Таким образом, мы получим хотя бы одну такую подгруппу $G_k$, что $\abs{G_k} = m \divby 2$. Тогда заметим, что $e = \brackets{g_k^{m/2}}^2 = a^2$ для некоторого $a \in G$. Что и требовалось доказать.

\i Для начала рассмотрим праве смежные классы. Легко понять, что класс однозначно определяется тем, куда попала единица. В самом деле, внутри одного класса единица всегда на одном месте, более того, остальные цифры перебирают все возможные перестановки (так как все они присутствуют в $H$). Таким образом, всего существует $n$ правых смежных классов, каждый из которых однозначно задаётся положением единицы. Абсолютно аналогично определяются и левые смежные классы.